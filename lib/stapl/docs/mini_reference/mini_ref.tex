\documentclass{report}
\begin{document}
\title{STAPL Alpha Release Mini\-Reference -- VERSION 0.5}
\date{\today}
\maketitle

\newenvironment{hashitemize}{%
  \renewcommand{\labelitemi}{\bfseries \#\#}%
        \begin{itemize}}{\end{itemize}}

\newcommand{\stapl}{{\sc STAPL}}
\newcommand{\stl}{{\sc STL}}
\newcommand{\pview}{{\sc PVIEW}}
\newcommand{\pviews}{{\sc PVIEWS}}
\newcommand{\pcontainer}{{\sc PCONTAINER}}
\newcommand{\pcontainers}{{\sc PCONTAINERS}}

\pagestyle{myheadings}
\renewcommand{\chaptermark}[1]{
              \markboth{*** DRAFT 0.5 *** \chaptername\ \thechapter. #1 } {} }
\renewcommand{\sectionmark}[1]{
              \markright{*** DRAFT 0.5 *** \thesection. #1 } {}}

% % % % % % % % % % % % % % % % % % % % % % % % % % % % % % % % % % % % %

\vspace*{8cm}
\begin{center}
Copyright (c) 2014, Texas Engineering Experiment Station (TEES), a
component of the Texas A\&M University System.

All rights reserved.
\end{center}
\pagebreak

% % % % % % % % % % % % % % % % % % % % % % % % % % % % % % % % % % % % %

\chapter{Introduction}

% % % % % % % % % % % % % % % % % % % % % % % % % % % % % % % % % % % % %

\section{What is \stapl?}

\stapl\ is a framework for parallel C++ code development.

What is a \stapl\ program?  It has three parts:
\begin{itemize}
\item Dependence graph
\newline
The dependence graph describes which memory operations must precede others,
and which computations produce values that others consume.
\item Data distribution
\newline
The data distribution describes how data aggregates should be partitioned
among the memories associated with processing elements.
\item Instructions
\newline
The instructions, as in any program, describe the computations.
\end{itemize}

The instructions are written in C++, using templated containers and algorithms
analogous to those in the Standard Template Library (\stl\ ).
The dependence graph is defined by the instructions written in C++.
The data distribution takes on default values
in the absence of an explicit specification.

The \stapl\ system has the following characteristics:

\begin{itemize}

\item Execution Model

The \stapl\ execution model is an asynchronous coarse-grain data flow machine.
\stapl\ enables the programmer to describe the parallelism in the program
at the finest granularity possible.  The \stapl\ system will coarsen the grain
of the described parallelism to fit the execution platform,
both at compile time and at execution time.

\item Memory Model

The \stapl\ memory model is a global address space with automatic data
distribution.  This is often referred to as Partitioned Global
Address Space (PGAS).  Data is never replicated in the memories associated
with processing elements.

\item Extensibility
\newline
The system is extensible because new kinds of each of the major features can
be introduced by composition, inheritance, and even implementation from scratch.
Programmers can create new kinds of views, containers, distributions,
algorithms, schedulers, and skeletons.

\item Engineering Excellence
\newline
The implementation of \stapl\ is engineered to be usable for real-world problems
on the full range of parallel systems, providing robust, high-performance
execution.
\end{itemize}

The following features implement the design principles of \stapl.

\begin{itemize}

\item Execution Model
\begin{itemize}
\item Multi-protocol parallelism:
\newline
Applications built with \stapl\ can employ message-passing on a coarse-grain
basis, and threading on a fine-grain basis, without any effort required
from the programmer.
\end{itemize}

\item Memory Model
\begin{itemize}
\item Transparent data distribution:
Programmers can create applications using a shared memory address space and
rely on \stapl\ to manage the distribution and data movement across distributed
memory configurations.  In addition to the default behavior, the programmer
can also select from a library of data distribution mechanisms, or even write
their own.
\newline
\end{itemize}

\item Extensibility
\begin{itemize}
\item Nested parallelism:
\newline
Programmers can compose parallelism using generic containers
and generic algorithms.
\item New container classes:
\newline
Programmers can use parallel data structures commonly employed in High Performance Computing which do not have serial analogs in the \stl\
%% FOR THE BETA RELEASE
%\item Library Interoperability
%\newline
%Applications built with \stapl\ will run in parallel with numerical libraries commonly used in High Performance Computing, such as LAPACK.
\end{itemize}

\item Engineering Excellence
\begin{itemize}
\item Scalable performance:
\newline
Applications built with \stapl\ will run in parallel on systems from
multi-processor tablets to the largest supercomputers.
\item Platform independence:
\newline
Applications built with \stapl\ can be deployed on both shared memory and distributed memory systems, requiring only a re-compile and re-link.  Execution is virtualized, providing a runtime system that is independent of the number or type of processors.
\end{itemize}

\end{itemize}

% % % % % % % % % % % % % % % % % % % % % % % % % % % % % % % % % % % % %

\chapter{A Complete \stapl\ Example Program}

This chapter presents a complete \stapl\ program, in order to motivate
the material that follows.  It reads numeric input from a file specified
on the command line, computes some simple descriptive statistics, and
displays the results to the user.  Comments in the code indicate the
name of the chapter in which the corresponding construct is discussed.

\vspace{0.2cm}
\noindent
\rule{12cm}{0.5mm}
\vspace{0.2cm}

\small
\begin{verbatim}
#include <cstdlib>
#include <iostream> // @@ C++ FOR PARALLEL PROGRAMMERS

#include <stapl/containers/vector/vector.hpp>
#include <stapl/views/vector_view.hpp>
#include <stapl/skeletons/serial.hpp>
#include <stapl/algorithms/algorithm.hpp>
#include <stapl/algorithms/sorting.hpp>
#include <stapl/stream.hpp>
#include <stapl/runtime.hpp>
#include <stapl/utility/do_once.hpp>

using namespace std; // @@ C++ FOR PARALLEL PROGRAMMERS

typedef stapl::vector<double> vec_dbl_tp; // @@ PARALLEL VIEWS
typedef stapl::vector_view<vec_dbl_tp> vec_dbl_vw_tp; // @@ PARALLEL VIEWS

// functor used to read values from file in parallel

struct get_val_wf { // @@ C++ FOR PARALLEL PROGRAMMERS
private:
  stapl::stream<ifstream> m_zin; // @@ INPUT / OUTPUT
public:
  get_val_wf(stapl::stream<ifstream> const& zin)
    : m_zin(zin)
  { }
  typedef void result_type;
  template<typename Ref>
  result_type operator() (Ref val) { // @@ C++ FOR PARALLEL PROGRAMMERS
    m_zin >> val; // @@ C++ FOR PARALLEL PROGRAMMERS
  }
  void define_type(stapl::typer& t) {
    t.member(m_zin);
  }
};

stapl::exit_code stapl_main(int argc, char ** argv) {

  // validate command line arguments
  if ( argc < 2 ) {
    stapl::do_once( [&]() { // @@ C++ FOR PARALLEL PROGRAMMERS
      cout << "Must specify input file name" << endl;
    } );
    return EXIT_FAILURE;
  }
  stapl::stream<ifstream> zin; // @@ INPUT / OUTPUT
  zin.open(argv[1]);
  if ( zin.is_open() ) {
    stapl::do_once( [&]() { // @@ INPUT / OUTPUT
      cout << "Not a valid input file name: " << argv[1] << endl;
    } );
    return EXIT_FAILURE;
  }
  if ( argc < 3 ) {
    stapl::do_once( [&]() {
      cout << "Must specify input data count" << endl;
    } );
    return EXIT_FAILURE;
  }
  int count = atoi(argv[2]);
  if ( count <= 0 ) {
    stapl::do_once( [&]() {
      cout << "Not a valid input data count: " << argv[2] << endl;
    } );
    return EXIT_FAILURE;
  }

  // read input from file
  vec_dbl_tp data(count), temp(count), diff(count); // @@ PARALLEL CONTAINERS
  vec_dbl_vw_tp data_vw(data), temp_vw(temp), diff_vw(diff); // @@ PARALLEL VIEWS

  stapl::serial_io(get_val_wf(zin), data_vw); // @@ INPUT / OUTPUT

  // compute statistics
  typedef stapl::min<double> min_dbl_wf; // @@ C++ FOR PARALLEL PROGRAMMERS
  typedef stapl::max<double> max_dbl_wf;
  typedef stapl::plus<double> add_dbl_wf;

  double min = stapl::reduce( data_vw, min_dbl_wf()); // @@ PARALLEL ALGORITHMS
  double max = stapl::reduce( data_vw, max_dbl_wf());
  double tot = stapl::reduce( data_vw, add_dbl_wf());
  double mean = tot / count;

  // median
  stapl::copy(data_vw, temp_vw); // @@ PARALLEL ALGORITHMS
  stapl::sort(temp_vw); // @@ PARALLEL ALGORITHMS

  double med = 0.0;
  int ndx = temp.size()/2;
  if ( 0 == temp.size() % 2 ) {
    med = 0.5 * (temp_vw[ndx-1] + temp_vw[ndx]);
  } else {
    med = temp_vw[ ndx + 1 ];
  }

  // standard deviation
  typedef stapl::minus<double> sub_dbl_wf;
  typedef stapl::multiplies<double> mul_dbl_wf;
  stapl::transform(data_vw, stapl::make_repeat_view(mean), diff_vw,
                   sub_dbl_wf() ); // @@ PARALLEL ALGORITHMS
  stapl::transform(diff_vw, diff_vw, temp_vw,
                   mul_dbl_wf() );
  double var = stapl::reduce( temp_vw, add_dbl_wf());
  var /= count - 1;
  double sdev = sqrt(var);

  // display results

  stapl::rmi_fence(); // @@ INPUT / OUTPUT

  stapl::do_once( [&]() {
    cout << "Descriptive statistics N= " << count << endl;
    cout << "min     " << min << "  ";
    cout << "max     " << max << "  ";
    cout << "mean    " << mean << "  ";
    cout << "median  " << med << "  ";
    cout << "std dev " << sdev << endl;
  } );

  return EXIT_SUCCESS;
}
\end{verbatim}
\normalsize


% % % % % % % % % % % % % % % % % % % % % % % % % % % % % % % % % % % % %

\chapter{C++ Concepts for Parallel Programmers}

This chapter briefly describes C++ concepts and features that are
essential for understanding the remainder of this document.

\section{Class Declarations}

A struct is a class in which the members are default public.
Since it is undesirable to have data members which have public access,
we adopt the convention in this document
that classes with no data members are defined using the
\texttt{{\bf struct}}
keyword rather than the
\texttt{{\bf class}}
keyword.

\section{Function Objects}

A function object is a class that defines
\texttt{{\bf operator()}}.
This operator is known as the "function call" or "application" operator.

Besides the application operator, a function object can also have
data members.  This enables the function to have state.

Each function object has its own type, unlike ordinary functions, which
have different types only when they have different signatures.

Because a function object is an object, it can be passed as a argument.
An alternate name for function object is "functor".

C++11 also supports lambdas, which are not used in this document.

\section{Templates}

A template is a class or function that is specified in terms of types
or values, which are treated as parameters of the template.

\section{Standard Template Library}

The Standard Template Library (\stl) is a part of the C++ standard library.
It provides the programmer with the ability to use \emph{generic programming}.
The \stl\ includes collection classes for organizing data, and algorithms
for applying to those collections.  The components of the \stl\ are all
template, so they can be used with any elemental type.

The \stl\ has three main features:
\begin{itemize}
\item
Containers
\newline
Containers provide varying interfaces to collections of objects.
These interfaces implement common Abstract Data Types.
\item
Iterators
\newline
Iterators are used to access the elements in collections of objects.
The chief benefit of iterators is that they provide a simple and common
interface for all of the container types.
\item
Algorithms
\newline
Algorithms use iterators to process the elements of containers.
\end{itemize}

\section{Namespaces}

In the examples in this text, all \stapl\ constructs are qualified with
\texttt{{\bf stapl::}}
to emphasize the \stapl\ functionality, rather than use a
\stapl\ namespace.
This also avoids conflicts with \stl\ entities.  We recommend that users
follow the same convention.  The alternative is to qualify all
constructs with
\texttt{{\bf std::}}.

% % % % % % % % % % % % % % % % % % % % % % % % % % % % % % % % % % % % %

\chapter{Parallelism Concepts for C++ Programmers}

This chapter briefly describes parallelism concepts that are
essential for understanding the remainder of this document.

\section{SPMD}

SPMD (Single Program, Multiple Data)
is a technique employed to achieve parallelism.
In SPMD, multiple autonomous processors simultaneously execute
the same program at independent points.

\section{MPI}

Message Passing Interface (MPI) is a de facto standard that supports portable
message-passing programming in Fortran, C and and C++. It consists of a set
of library routines that provide protocols for point-to-point and collective
communication. Implementations are available on common processor architectures and operating systems.

\section{OpenMP}

Open Multi-Processing (OpenMP) is a standard that supports multi-platform
shared memory multiprocessing programming in Fortran, C, and C++.
It consists of a set of compiler directives (Fortran) or pragmas (C/C++), library routines, and environment variables that influence run-time behavior.
Implementations are available on common processor architectures and operating systems.

% % % % % % % % % % % % % % % % % % % % % % % % % % % % % % % % % % % % %

\chapter{Relationship between \stapl\ and \stl\ }

\stapl\ has functionality similar to the \stl.  \stapl\ executes on uni-
or multi-processor architectures with shared or distributed memory and
can co-exist in the same program with \stl.  \stapl\ top-level components
have well-defined counterparts in \stl.

\begin{itemize}
\item
pContainers - parallel equivalents of \stl\ \emph{containers}.
\item
pAlgorithms - parallel equivalents of \stl\ \emph{algorithms}.
\item
pViews - provide data access operations independent of the underlying
storage, as do \stl\ iterators.
\end{itemize}

Having a basic understanding of the major features of \stl\ will make it
much easier to learn \stapl.

% % % % % % % % % % % % % % % % % % % % % % % % % % % % % % % % % % % % %

\chapter{\stapl\ Concepts}

This chapter briefly describes \stapl\ concepts that are
essential for understanding the remainder of this document.

\section{Parallel Containers}

\stapl\ parallel containers (\pcontainers) are distributed,
thread-safe, concurrent objects.  The are composable and extensible
by C++ inheritance mechanisms.  Most \pcontainers\ have analogs in the
\stl.  There are several additional containers not provided by the \stl\
which are very important for High Performance Computing: the parallel
multi-dimensional array (pMultiarray) and the parallel graph (pGraph).

\pcontainers\ provide methods that correspond to the \stl\ container
methods, where there is an analogous container.  They also provide
additional methods specifically designed for parallel use.

While the data in a \pcontainer\ is distributed, \pcontainers\ provide a
\emph{shared object view}.  This means that they are shared data structures
with a global address space.  Each \pcontainer\ element has a unique
global identifier (GID).  \stapl\ automatically translates the GID so that
it can locate both local and remote elements.

The following containers are supported by \stapl:
\begin{itemize}
\item
Sequence containers: {\tt list, vector }
\item
Associative containers: {\tt set, map }
\item
Indexed containers: {\tt array, matrix, multiarray (vector is also indexed) }
\item
Relational containers: {\tt graph, dynamic\_graph, multi\_graph }
\end{itemize}

\section{Iterators, Ranges, and Views}

The \emph{iterator} is a common abstraction used to represent data
in sequential programming.
Iterators are used in generic algorithms, such as
the \stl\ in C++, to specify the data.
A key function of the iterator is to separate data structures from algorithms.
All accesses to data are performed through iterators, which are implemented
as memory pointers.  Iterators cannot encapsulate other information about the
referenced data, e.g., length of the sequence.
Moreover, the operations associated with iterators are essentially
limited to {\em sequential} traversals of a collection of elements,
and so algorithms using iterators have a linear, flat memory model
that only supports {\em sequential access}.

To allow concurrent access to multiple elements,
some parallel systems provide the \emph{range}.
A range defines a sequence of elements between two
iterators. With the range abstraction, the program can create
sub-ranges and process them in parallel.  Ranges alleviate the
sequential access problem, but still offer only a flat memory model.

In \stapl, a \pview\ encapsulates the information about the collection
of elements it represents, the elements that can be accessed through it,
and provides a set of operations to interact with the data.
Using \pviews, parallel algorithms can be specified using abstract data
types (ADTs).  The \pview\ encodes information about the partition of
the data that allows it to operate in a distributed environment.
The \emph{domain} of a \pview\ is the valid set of identifiers.
The domain specifies which elements of the underlying container
can be accessed through the \pview.

\section{Parallel Views}

Different ADTs provide common sets of operations or concepts.
Views can have any of the following attributes, which control the
operations they can perform.

\texttt{read\_write}:
Views that have this attribute support operations to read and
write values for a given element identifier (i.e., index or key).
Some views have the \emph{read} attribute, but not the \emph{write}
attribute.

\texttt{subscript}:
Views that have this attribute support the square-bracket operator.
This operator returns a reference over the element instead of
returning a copy of the value, as the
\texttt{read}
operation does.

\texttt{sequence}:
Views that have this attribute support the operations
\texttt{begin},
and
\texttt{end}
and the required associated types
required for use with iterators.

\texttt{insert\_erase}:
Views that have this attribute provide the dynamic
operations \emph{insert} and \emph{erase}.

\stapl\ views fall into the following categories:

\begin{itemize}
\item
Model abstract data types:
\newline
array, vector, list, map, multiarray, heap, set, graph
\item
Modify access patterns to the container:
\newline
repeated, overlap, counting, strided, reverse, filter, transform, linear
\item
Modify distribution of the container:
\newline
native, balance, explicit, segmented
\end{itemize}

\section{Work Functions}

A work function is a function object which is applied to each element
of a view by a basic algorithm (see below).
\texttt{{\bf operator()}}
specified by the work function
should have as many arguments as there views passed to the basic algorithm.

A work function can have other members besides
\texttt{{\bf operator()}}.
It can have data members, which hold state.
If the function object has data members, the class will need
a constructor and/or "setter" method to give them values.  It will
also need a
\texttt{{\bf define\_type}}
method (see Chapter 3).
It may also need a "getter" method to retrieve updated state.

\section{Basic Algorithms}

These functions take one or more work functions as arguments,
which are applied to the elements of the specified view(s),
according to the dependence graph which defines the semantics
of the particular basic algorithm.

\texttt{{\bf map\_func }}

When processing a single view, apply the work function to each
item of the view independently.  When processing multiple views,
apply the work function to a corresponding value from each view.
The work function may return a void type, if it will not be used
to create nested parallelism.

The following pseudocode shows how map\_func acts on simple vectors.

\begin{verbatim}
  p <- [2 3 5 7 11]
  f <- [1 1 2 3 5 ]
  map_func( plus, p, f )
3 4 7 10 16
\end{verbatim}

\texttt{{\bf map\_reduce }}

Apply the first work function independently to each item of the view.
Apply the second work function cumulatively to each item of the values from
the first application.

This function is sometimes used with an identity function for
the map operation, thus becoming a simple reduction.
Both of the work functions must return a non-void type.

\begin{verbatim}
  p <- [2 3 5 7 11]
  f <- [1 1 2 3 5 ]
  map_reduce( neg, plus, p, f )
-3 -4 -7 -10 -16
\end{verbatim}

\texttt{{\bf reduce }}

Apply the first work function independently to each item of the view.
The work function must return a non-void type.

\begin{verbatim}
  p <- [2 3 5 7 11]
  reduce( plus, p ]
28
\end{verbatim}

\texttt{{\bf scan }}

Apply the work function cumulatively to each item of the view,
saving the intermediate results of each function application.
The work function must return a non-void type.

\begin{verbatim}
  f <- [1 1 2 3 5 ]
  scan( plus, f )
1 2 4 7 12
\end{verbatim}

\texttt{{\bf serial }}

Apply the work function to each item of the view in serial order.
The work function may return a void type, if it will not be used
in a nested fashion.

\texttt{{\bf serial\_io }}

Apply the work function to each item of the view in serial order,
and execute it on location zero.
The work function may return a void type, if it will not be used
in a nested fashion.

\texttt{{\bf do\_once }}

Execute the work function a single time on location zero.

All \stapl\ constructs are qualified with
\texttt{{\bf stapl::}}
to emphasize the \stapl\ functionality, rather than use a
\stapl\ namespace.
This also avoids conflicts with \stl\ entities.  We recommend that users
follow the same convention.  The alternative is to qualify all
constructs with
\texttt{{\bf std::}}.

\section{Nested Parallelism}

\stapl\ implements parallel algorithms and data structures using standard C++,
which provide an SPMD model of parallel, and which support nested (recursive)
parallelism.  Nested parallelism is important because:

\begin{itemize}
\item
Many large parallel systems have a hierarchical architecture onto which
nested parallelism would naturally map.
\item
Library functions are often used as basic building blocks which themselves
can be nested or incorporated into a larger parallel application.
\end{itemize}

\section{Portability}

\stapl\ provides portability across multiple systems by including its own
runtime system.  This supports high level parallel constructs (e.g. task
graph execution) and their low level implementation.  \stapl\ also provides
its own scheduling and data distribution interfaces.  So, there is no need
to modify user code when porting a \stapl\ application from one system to
another.

% % % % % % % % % % % % % % % % % % % % % % % % % % % % % % % % % % % % %

\chapter{Distributed Containers}

STAPL supports the following distributed containers:
\vspace{0.4cm}
\newline
Array (section \ref{sec-ary-cont}),
Static Array (section \ref{sec-stary-cont}),
\newline
Vector (section \ref{sec-vec-cont}),
List (section \ref{sec-list-cont}),
\newline
Static Graph (section \ref{sec-stgraf-cont}),
Dynamic Graph (section \ref{sec-dygraf-cont}),
\newline
Matrix (section \ref{sec-mat-cont}),
Multiarray (section \ref{sec-multi-cont}),
\newline
Map (section \ref{sec-map-cont}),
Set (section \ref{sec-set-cont}),
\newline
Unordered Map (section \ref{sec-unmap-cont})
Unordered Set (section \ref{sec-unset-cont})
\vspace{0.4cm}

% % % % % % % % % % % % % % % % % % % % % % % % % % % % % % % % % % % % % % %

\section{Introduction} \label{sec-cont-intro}

A distributed container is the parallel equivalent of the STL container and is
backward compatible with STL containers through its ability to provide iterators. Each distributed container provides (semi-) random access to its elements, a prerequisite for efficient parallel processing. Random access to the subsets of a distributed container's data is provided by an internal distribution maintained by the distributed container. The distribution is updated when elements are added or removed from the distributed container, and when migration of elements between locations is requested. The distribution has two primary components.

The container manager maintains the subsets of elements stored on a location. Each subset is referred to as a bContainer. A distributed container instance may have more than one bContainer per location depending on the desired data distribution and independent migration of elements. The second component of the distribution is the directory, which enables any location to determine the location on which an element of the distributed container is stored.

\vspace{0.4cm}
%%\textbf{
%%The purpose of a container is to be constructed and initialized with data.
%%It is sometimes necessary to process an element, or a subset of elements of
%%a container in a sequential manner.  This chapter documents the methods 
%%provided by \stapl\ to do that.
%%
%%In order to perform operations in parallel on the elements of a container,
%%a view should be constructed over the container, and the operations
%%should be applied to the view.  Performing operations directly on the
%%container is the exception, not the normal way of using \stapl.
%%}

\subsection{Implementation}

\textit{WRITE - implementation common to all containers}


% % % % % % % % % % % % % % % % % % % % % % % % % % % % % % % % % % % % % % %

\section{ Array Container} \label{sec-ary-cont}
\index{array!container}
\index{container!array}
\index{array!constructor}

\subsection{Definition}

\textit{EXPAND - Distributed sequence container with fixed size.}

\subsection{Relationship to \stl\ array}

\textit{WRITE}

\subsection{Relationship to other \stapl\ containers}

\textit{WRITE}

\subsection{Implementation}

\textit{WRITE - details specific to this container}

\subsection{Interface} \label{sec-ary-cont-inter}

\subsubsection{Constructors}

\noindent
\textbf{array}%
\texttt{%
(void)
}

\begin{itemize}
\item
Create an array with size 0. Initially places the array in an unusable state. Should be used in conjunction with array::resize.
\end{itemize}
 
\noindent
\textbf{array}%
\texttt{%
(size\_type 
\textit{n}%
) 
}

\begin{itemize}
\item
Create an array with a given size and default construct all elements.
\end{itemize}
 
\noindent
\textbf{array}%
\texttt{%
(size\_type 
\textit{n,}%
value\_type const \& 
\textit{default\_value}%
) 
}

\begin{itemize}
\item
Create an array with a given size and construct all elements with a default value. 
\end{itemize}
 
\noindent
\textbf{array}%
\texttt{%
(size\_type 
\textit{n,}%
mapper\_type const \& 
\textit{mapper}%
)
}

\begin{itemize}
\item
Create an array with a given size and instance of mapper. 
\end{itemize}
 
\noindent
\textbf{array}%
\texttt{%
(partition\_type const \&
\textit{ps}%
)
}

\begin{itemize}
\item
Create an array with a given instance of partition. 
\end{itemize}
 
\noindent
\textbf{array}%
\texttt{%
(partition\_type const \&
\textit{partitioner,}%
mapper\_type const \&
\textit{mapper}%
)
}

\begin{itemize}
\item
Create an array with a given partitioner and a mapper. 
\end{itemize}
 
\noindent
\texttt{%
template<typename DistSpecView >
}
\newline
\textbf{array}%
\texttt{%
(DistSpecView const \&
\textit{dist\_view,}%
typename boost::enable\_if< 
    is\_distribution\_view< DistSpecView > >::type *=0
)
}

\vspace{0.4cm} \emph{fix Doxygen to omit enable\_if, not part of user interface}

\begin{itemize}
\item
Create an array with a distribution that is specified by the dist\_view provided. 
\end{itemize}
 
\noindent
\texttt{%
template<typename DistSpecView >
}
\newline
\textbf{array}%
\texttt{%
(DistSpecView const \&
\textit{dist\_view,}%
value\_type const \&
\textit{default\_value,}%
typename boost::enable\_if< is\_distribution\_view< DistSpecView > >::type *=0
)
}

\vspace{0.4cm} \emph{fix Doxygen to omit enable\_if, not part of user interface}

\begin{itemize}
\item
Create an array with a distribution that is specified by the dist\_view provided, and initialize all elements to default\_value. 
\end{itemize}
 
\noindent
\texttt{%
template<typename DP >
}
\newline
\textbf{array}%
\texttt{%
(size\_type 
\textit{n,}%
value\_type const \&
\textit{default\_value,}%
DP const \&
\textit{dis\_policy}%
)
}

\begin{itemize}
\item
Create an array with a given size and default value where the value type of the container is itself a distributed container. 
\end{itemize}
 
\noindent
\texttt{%
template<typename X, typename Y >
}
\newline
\textbf{array}%
\texttt{%
(boost::tuples::cons< X, Y > dims
)
}

\begin{itemize}
\item
Create an array of arrays with given n-dimensional size. 
\end{itemize}
 
\noindent
\texttt{%
template<typename X , typename Y , typename DP >
}
\newline
\textbf{array}%
\texttt{%
(boost::tuples::cons< X, Y > 
\textit{dims,}%
DP const \&
\textit{dis\_policy}%
)
}

\begin{itemize}
\item
Create an array of arrays with given n-dimensional size. 
\end{itemize}
 
\noindent
\texttt{%
template<typename SizesView >
}
\newline
\textbf{array}%
\texttt{%
(SizesView const \&
\textit{sizes\_view,}%
typename boost::enable\_if< boost::mpl::and\_< boost::is\_same< size\_type, 
typename SizesView::size\_type >, 
boost::mpl::not\_< is\_distribution\_view< SizesView > > > >::type *=0
)
}

\vspace{0.4cm} \emph{fix Doxygen to omit enable\_if, not part of user interface}

\begin{itemize}
\item
Constructor for composed containers. For an m-level composed container, sizes\_view is an m-1 level composed view representing the sizes of the nested containers. 
\end{itemize}

\subsubsection{Element Manipulation}

\noindent
\texttt{%
reference
}
\textbf{operator[]}%
\texttt{%
(index\_type 
\textit{idx}%
)
}

\begin{itemize}
\item
Construct a reference to a specific index of the array.
\end{itemize}
 
\noindent%
\texttt{%
const\_reference
}
\textbf{operator[]}%
\texttt{%
(index\_type 
\textit{idx}%
) const
}

\begin{itemize}
\item
Construct a const\_reference to a specific index of the array.
\end{itemize}
 
\noindent
\texttt{%
reference
}
\textbf{front}%
\texttt{%
(void)
}

\begin{itemize}
\item
Construct a reference to the first element of the array.
\end{itemize}
 
\noindent
\texttt{%
const\_reference
}
\textbf{front}%
\texttt{%
(void) const
}

\begin{itemize}
\item
Construct a const\_reference to the first element of the array.
\end{itemize}
 
\noindent
\texttt{%
reference
}
\textbf{back}%
\texttt{%
(void)
}

\begin{itemize}
\item
Construct a reference to the last element of the array.
\end{itemize}
 
\noindent
\texttt{%
const\_reference
}
\textbf{back}%
\texttt{%
(void) const
}

\begin{itemize}
\item
Construct a const\_reference to the last element of the array.
\end{itemize}
 
\noindent
\texttt{%
reference
}
\textbf{make\_reference}%
\texttt{%
(index\_type const \&
\textit{idx}%
)
}

\begin{itemize}
\item
Construct a reference to a specific index of the array.
\end{itemize}
 
\noindent
\texttt{%
const\_reference
}
\textbf{make\_reference}%
\texttt{%
(index\_type const \&
\textit{idx}%
) const
}

\begin{itemize}
\item
Construct a const\_reference to a specific index of the array.
\end{itemize}
 
\noindent
\texttt{%
iterator
}
\textbf{make\_iterator}%
\texttt{%
(gid\_type const \&
\textit{gid}%
)
}

\begin{itemize}
\item
Construct an iterator to a specific index of the array.
\end{itemize}
 
\noindent
\texttt{%
const\_iterator 
}
\textbf{make\_const\_iterator}%
\texttt{%
(gid\_type const \&
gid
) const
}

\begin{itemize}
\item
Construct a const\_iterator to a specific index of the array.
\end{itemize}
 
\noindent
\texttt{%
iterator
}
\textbf{begin}%
\texttt{%
(void)
}

\begin{itemize}
\item
Construct an iterator to the beginning of the array.
\end{itemize}
 
\noindent
\texttt{%
const\_iterator
}
\textbf{begin}%
\texttt{%
(void) const
}

\begin{itemize}
\item
Construct a const\_iterator to the beginning of the array.
\end{itemize}
 
\noindent
\texttt{%
const\_iterator
}
\textbf{cbegin}%
\texttt{%
(void) const
}

\begin{itemize}
\item
Construct a const\_iterator to the beginning of the array.
\end{itemize}
 
\noindent
\texttt{%
iterator
}
\textbf{end}%
\texttt{%
(void)
}

\begin{itemize}
\item
Construct an iterator to one past the end of the array.
\end{itemize}
 
\noindent
\texttt{%
const\_iterator
}
\textbf{end}%
\texttt{%
(void) const
}

\begin{itemize}
\item
Construct a const\_iterator to one past the end of the array.
\end{itemize}
 
\noindent
\texttt{%
const\_iterator
}
\textbf{cend}%
\texttt{%
(void) const
}

\begin{itemize}
\item
Construct a const\_iterator to one past the end of the array.
\end{itemize}
 
\noindent
\texttt{%
void
}
\textbf{set\_element}%
\texttt{%
(index\_type const \&
\textit{idx,}%
value\_type const \&
\textit{val}%
)
}

\begin{itemize}
\item
Sets the element specified by the index to the provided value.
\end{itemize}
 
\noindent
\texttt{%
value\_type
}
\textbf{get\_element}%
\texttt{%
(index\_type const \&
\textit{idx}%
) const
}

\begin{itemize}
\item
Returns the value of the element specified by the index.
\end{itemize}
 
\noindent
\texttt{%
future< value\_type >
}
\textbf{get\_element\_split}%
\texttt{%
(index\_type const \&
\textit{idx}%
)
}

\begin{itemize}
\item
Returns a stapl::future holding the value of the element specified by the index.
\end{itemize}
 
\noindent
\texttt{%
void
}
\textbf{apply\_set}%
\texttt{%
(gid\_type const \&
\textit{gid,}%
F const \&
\textit{f}%
)
}

\begin{itemize}
\item
Applies a function f to the element specified by the GID.
\end{itemize}
 
\noindent
\texttt{%
F::result\_type
}
\textbf{apply\_get}%
\texttt{%
(gid\_type const \&
\textit{gid,}%
F const \&
\textit{f}%
)
}

\begin{itemize}
\item
Applies a function f to the element specified by the GID, and returns the result.
\end{itemize}
 
\noindent
\texttt{%
F::result\_type
}
\textbf{apply\_get}%
\texttt{%
(gid\_type const \&
\textit{gid,}%
F const \&
\textit{f}%
) const
}

\begin{itemize}
\item
Applies a function f to the element specified by the GID, and returns the result.
\end{itemize}
 
\subsubsection{Memory and Domain Management}

\noindent
\texttt{%
domain\_type
}
\textbf{domain}%
\texttt{%
(void) const
}

\begin{itemize}
\item
Returns the domain of the container.
\end{itemize}
 
\noindent
\texttt{%
void 
}
\textbf{resize}%
\texttt{%
(size\_type 
\textit{n}%
)
}

\begin{itemize}
\item
Destroy the distribution of the container (including all of its elements) and recreate the container with a different size. 
\end{itemize}
 
\noindent
\texttt{%
template<typename DistSpecView >
void
}
\textbf{redistribute}%
\texttt{%
(DistSpecView const \&
}
\textit{dist\_view,}
\texttt{
typename std::enable\_if< is\_distribution\_view< DistSpecView >::value \&\&is\_view\_based< partition\_type >::value \&\&is\_view\_based< mapper\_type >::value >::type *=0)
}

\vspace{0.4cm} \emph{fix Doxygen to omit enable\_if, not part of user interface}

\begin{itemize}
\item
Redistribute the data stored in the container to match the distribution specified by the distribution view provided. 
\end{itemize}
 
\noindent
\texttt{%
void
}
\textbf{migrate}%
\texttt{%
(gid\_type const \&
\textit{gid,}%
location\_type 
\textit{destination}
)
}

\begin{itemize}
\item
Migrates the element specified by the gid to the destination location. 
\end{itemize}
 
\noindent
\texttt{%
size\_type
}
\textbf{size}%
\texttt{%
(void) const
}

\begin{itemize}
\item
Returns the number of elements in the container. 
\end{itemize}
 
\noindent
\texttt{%
bool
}
\textbf{empty}%
\texttt{%
(void) const
}

\begin{itemize}
\item
Returns if the container has no element.
\end{itemize}
 
\noindent
\texttt{%
distribution\_type \& 
}
\textbf{distribution}%
\texttt{%
 (void)
}
 
\noindent
\texttt{%
distribution\_type const \& 
}
\textbf{distribution}%
\texttt{%
 (void) const
}
 
\noindent
\texttt{%
distribution\_type *
}
\textbf{get\_distribution}%
\texttt{%
(void)
}
 
\begin{itemize}
\item
Returns the data distribution of the container.
\end{itemize}

\noindent
\texttt{%
locality\_info 
}
\textbf{locality}%
\texttt{%
 (gid\_type gid)
}

\begin{itemize}
\item
Return locality information about the element specified by the gid. 
\end{itemize}
 
\noindent
\texttt{%
bool
}
\textbf{is\_local}%
\texttt{%
(gid\_type const \&
\textit{gid}%
)
}

\begin{itemize}
\item
Returns true if the element specified by the GID is stored on this location, or false otherwise. 
\end{itemize}

\subsection{Usage Example} \label{sec-ary-cont-use}

The following example shows how to use the \texttt{array container}:
%%\newline\vspace{0.4cm}\newline\rule{12cm}{0.5mm}
\input{cont/array.tex}
%%\noindent\rule{12cm}{0.5mm}\vspace{0.5cm}

\textbf{PLEASE NOTE: 
Initializing a container with a sequential loop, as in this example,
is for illustrative purposes only.
The normal way to do this is applying a parallel algorithm or 
basic parallel construct to a view over the container.  See Section
\ref{sec-ary-vw-use}
for the recommended usage.
}

\vspace{0.4cm} \textit{WRITE -comments on example}

\subsection{Implementation} \label{sec-ary-cont-impl}

\textit{WRITE}

\subsection{Performance} \label{sec-ary-cont-perf}

\begin{itemize}
\item
Fig. \ref{fig:ary-cont-constr-exper}
shows the performance of constructing a \stapl\ array container
whose elements are atomic values, \stl\ containers, or \stapl\ containers.
\item
Fig. \ref{fig:ary-cont-assign-exper}
shows the performance of assigning values to a \stapl\ array container
whose elements are atomic values, \stl\ containers, or \stapl\ containers.
\item
Fig. \ref{fig:ary-cont-access-exper}
shows the performance of accessing values from a \stapl\ array container
whose elements are atomic values, \stl\ containers, or \stapl\ containers.
\end{itemize}

\textit{WRITE - complexity analysis}

\begin{figure}[p]
%%\includegraphics[scale=0.50]{figs/array_cont_constr}
\caption{Construct array Execution Time}
\label{fig:ary-cont-constr-exper}
\end{figure}

\begin{figure}[p]
%%\includegraphics[scale=0.50]{figs/array_cont_assign}
\caption{Assign Values to array Execution Time}
\label{fig:ary-cont-assign-exper}
\end{figure}

\begin{figure}[p]
%%\includegraphics[scale=0.50]{figs/array_cont_access}
\caption{Access Values in array Execution Time}
\label{fig:ary-cont-access-exper}
\end{figure}


% % % % % % % % % % % % % % % % % % % % % % % % % % % % % % % % % % % % % % % 

\section{Static Array Container} \label{sec-stary-cont}
\index{array!static!container}
\index{container!array!static}
\index{array!static!constructor}

\subsection{Definition}

\textit{EXPAND - Distributed sequence container with fixed size. 
This container provides fast access, but has limits: it cannot be migrated and, the distribution is fixed to a balanced partition with one base container per location.}

\subsection{Relationship to \stl\ array}

\textit{WRITE}

\subsection{Relationship to other \stapl\ containers}

\textit{WRITE}

\subsection{Implementation}

\textit{WRITE - details specific to this container}

\subsection{Interface} \label{sec-stary-cont-inter}

\subsubsection{Constructors}

\noindent
\textbf{static\_array}%
\texttt{%
(size\_type
\textit{n}%
)
}

\begin{itemize}
\item
Create an array with a given size and default constructs all elements. 
\end{itemize}
 
\noindent
\textbf{static\_array}%
\texttt{%
(size\_type 
\textit{n,}%
value\_type const \&
\textit{default\_value}%
)
}

\begin{itemize}
\item
Create an array with a given size and constructs all elements with a default value. 
\end{itemize}
 
\noindent
\texttt{%
template<typename DP >
}
\newline
\textbf{static\_array}%
\texttt{%
(size\_type 
\textit{n,}%
value\_type const \&
\textit{default\_value,}%
DP const \&
\textit{dis\_policy}%
)
}

\begin{itemize}
\item
Create an array with a given size and default value where the value type of the container is itself a distributed container. 
\end{itemize}
 
\noindent
\texttt{%
template<typename X , typename Y >
}
\newline
\textbf{static\_array}%
\texttt{%
(boost::tuples::cons< X, Y > dims
)
}

\begin{itemize}
\item
Create an array of arrays with given n-dimensional size. 
\end{itemize}
 
\noindent
\texttt{%
template<typename X , typename Y , typename DP >
}
\newline
\textbf{static\_array}%
\texttt{%
(boost::tuples::cons< X, Y > dims, DP const \&
\textit{dis\_policy}%
)
}

\begin{itemize}
\item
Create an array of arrays with given n-dimensional size. 
\end{itemize}

\subsubsection{Element Manipulation}

\noindent
\texttt{%
reference
}
\textbf{operator[]}%
\texttt{%
(index\_type 
\textit{idx}%
)
}

\begin{itemize}
\item
Construct a reference to a specific index of the array.
\end{itemize}
 
\noindent%
\texttt{%
const\_reference
}
\textbf{operator[]}%
\texttt{%
(index\_type 
\textit{idx}%
) const
}

\begin{itemize}
\item
Construct a const\_reference to a specific index of the array.
\end{itemize}
 
\noindent
\texttt{%
reference
}
\textbf{front}%
\texttt{%
(void)
}

\begin{itemize}
\item
Construct a reference to the first element of the array.
\end{itemize}
 
\noindent
\texttt{%
const\_reference
}
\textbf{front}%
\texttt{%
(void) const
}

\begin{itemize}
\item
Construct a const\_reference to the first element of the array.
\end{itemize}
 
\noindent
\texttt{%
reference
}
\textbf{back}%
\texttt{%
(void)
}

\begin{itemize}
\item
Construct a reference to the last element of the array.
\end{itemize}
 
\noindent
\texttt{%
const\_reference
}
\textbf{back}%
\texttt{%
(void) const
}

\begin{itemize}
\item
Construct a const\_reference to the last element of the array.
\end{itemize}
 
\noindent
\texttt{%
reference
}
\textbf{make\_reference}%
\texttt{%
(index\_type const \&
\textit{idx}%
)
}

\begin{itemize}
\item
Construct a reference to a specific index of the array.
\end{itemize}
 
\noindent
\texttt{%
const\_reference
}
\textbf{make\_reference}%
\texttt{%
(index\_type const \&
\textit{idx}%
) const
}

\begin{itemize}
\item
Construct a const\_reference to a specific index of the array.
\end{itemize}
 
\noindent
\texttt{%
iterator
}
\textbf{make\_iterator}%
\texttt{%
(gid\_type const \&
\textit{gid}%
)
}

\begin{itemize}
\item
Construct an iterator to a specific index of the array.
\end{itemize}
 
%%\noindent
%%\texttt{%
%%const\_iterator 
%%}
%%\textbf{make\_const\_iterator}%
%%\texttt{%
%%(gid\_type const \&
%%gid
%%) const
%%}

%%\begin{itemize}
%%\item
%%Construct a const\_iterator to a specific index of the array.
%%\end{itemize}
 
\noindent
\texttt{%
iterator
}
\textbf{begin}%
\texttt{%
(void)
}

\begin{itemize}
\item
Construct an iterator to the beginning of the array.
\end{itemize}
 
\noindent
\texttt{%
const\_iterator
}
\textbf{begin}%
\texttt{%
(void) const
}

\begin{itemize}
\item
Construct a const\_iterator to the beginning of the array.
\end{itemize}
 
\noindent
\texttt{%
const\_iterator
}
\textbf{cbegin}%
\texttt{%
(void) const
}

\begin{itemize}
\item
Construct a const\_iterator to the beginning of the array.
\end{itemize}
 
\noindent
\texttt{%
iterator
}
\textbf{end}%
\texttt{%
(void)
}

\begin{itemize}
\item
Construct an iterator to one past the end of the array.
\end{itemize}
 
\noindent
\texttt{%
const\_iterator
}
\textbf{end}%
\texttt{%
(void) const
}

\begin{itemize}
\item
Construct a const\_iterator to one past the end of the array.
\end{itemize}
 
\noindent
\texttt{%
const\_iterator
}
\textbf{cend}%
\texttt{%
(void) const
}

\begin{itemize}
\item
Construct a const\_iterator to one past the end of the array.
\end{itemize}
 
\noindent
\texttt{%
void
}
\textbf{set\_element}%
\texttt{%
(index\_type const \&
\textit{idx,}%
value\_type const \&
\textit{val}%
)
}

\begin{itemize}
\item
Sets the element specified by the index to the provided value.
\end{itemize}
 
\noindent
\texttt{%
value\_type
}
\textbf{get\_element}%
\texttt{%
(index\_type const \&
\textit{idx}%
) const
}

\begin{itemize}
\item
Returns the value of the element specified by the index.
\end{itemize}
 
\noindent
\texttt{%
future< value\_type >
}
\textbf{get\_element\_split}%
\texttt{%
(index\_type const \&
\textit{idx}%
)
}

\begin{itemize}
\item
Returns a stapl::future holding the value of the element specified by the index.
\end{itemize}
 
\noindent
\texttt{%
void
}
\textbf{apply\_set}%
\texttt{%
(gid\_type const \&
\textit{gid,}%
F const \&
\textit{f}%
)
}

\begin{itemize}
\item
Applies a function f to the element specified by the GID.
\end{itemize}
 
\noindent
\texttt{%
F::result\_type
}
\textbf{apply\_get}%
\texttt{%
(gid\_type const \&
\textit{gid,}%
F const \&
\textit{f}%
)
}

\begin{itemize}
\item
Applies a function f to the element specified by the GID, and returns the result.
\end{itemize}
 
\noindent
\texttt{%
F::result\_type
}
\textbf{apply\_get}%
\texttt{%
(gid\_type const \&
\textit{gid,}%
F const \&
\textit{f}%
) const
}

\begin{itemize}
\item
Applies a function f to the element specified by the GID, and returns the result.
\end{itemize}
 

 
\subsubsection{Memory and Domain Management}

\noindent
\texttt{%
domain\_type 	
}
\textbf{domain}%
\texttt{%
(void) const
}

\begin{itemize}
\item
Returns the domain of the container.
\end{itemize}
 
\noindent
\texttt{%
size\_type
}
\textbf{size}%
\texttt{%
(void) const
}

\begin{itemize}
\item
Returns the number of elements in the container. 
\end{itemize}
 
\noindent
\texttt{%
bool
}
\textbf{empty}%
\texttt{%
(void) const
}

\begin{itemize}
\item
Returns if the container has no element.
\end{itemize}
 
\noindent
\texttt{%
distribution\_type \& 
}
\textbf{distribution}%
\texttt{%
 (void)
}
 
\noindent
\texttt{%
distribution\_type const \& 
}
\textbf{distribution}%
\texttt{%
 (void) const
}
 
\noindent
\texttt{%
distribution\_type *
}
\textbf{get\_distribution}%
\texttt{%
(void)
}
 
\begin{itemize}
\item
Returns the data distribution of the container.
\end{itemize}

\noindent
\texttt{%
locality\_info 
}
\textbf{locality}%
\texttt{%
 (gid\_type gid)
}

\begin{itemize}
\item
Return locality information about the element specified by the gid. 
\end{itemize}
 
\noindent
\texttt{%
bool
}
\textbf{is\_local}%
\texttt{%
(gid\_type const \&
\textit{gid}%
)
}

\begin{itemize}
\item
Returns true if the element specified by the GID is stored on this location, or false otherwise. 
\end{itemize}

\subsection{Usage Example} \label{sec-stary-cont-use}

The following example shows how to use the \texttt{static array container}:
%%\newline\vspace{0.4cm}\newline\rule{12cm}{0.5mm}
\input{cont/starray.tex}
%%\noindent\rule{12cm}{0.5mm}\vspace{0.5cm}

\vspace{0.4cm} \textit{WRITE -comments on example}

\subsection{Implementation} \label{sec-stary-cont-impl}

\textit{WRITE}

\subsection{Performance} \label{sec-stary-cont-perf}

\begin{itemize}
\item
Fig. \ref{fig:stary-cont-constr-exper}
shows the performance of constructing a \stapl\ static array container
whose elements are atomic values, \stl\ containers, or \stapl\ containers.
\item
Fig. \ref{fig:stary-cont-assign-exper}
shows the performance of assigning values to a \stapl\ static array container
whose elements are atomic values, \stl\ containers, or \stapl\ containers.
\item
Fig. \ref{fig:stary-cont-access-exper}
shows the performance of accessing values from a \stapl\ static array container
whose elements are atomic values, \stl\ containers, or \stapl\ containers.
\end{itemize}

\textit{WRITE - complexity analysis}

\begin{figure}[p]
%%\includegraphics[scale=0.50]{figs/static array_cont_constr}
\caption{Construct static array Execution Time}
\label{fig:stary-cont-constr-exper}
\end{figure}

\begin{figure}[p]
%%\includegraphics[scale=0.50]{figs/static array_cont_assign}
\caption{Assign Values to static array Execution Time}
\label{fig:stary-cont-assign-exper}
\end{figure}

\begin{figure}[p]
%%\includegraphics[scale=0.50]{figs/static array_cont_access}
\caption{Access Values in static array Execution Time}
\label{fig:stary-cont-access-exper}
\end{figure}


% % % % % % % % % % % % % % % % % % % % % % % % % % % % % % % % % % % % % % % 

\section{Vector Container} \label{sec-vec-cont}
\index{vector!container}
\index{container!vector}
\index{vector!constructor}

\subsection{Definition}

\textit{EXPAND - Distributed sequence container with element insertion and deletion methods.}

\subsection{Relationship to \stl\ vector}

\textit{WRITE}

\subsection{Relationship to other \stapl\ containers}

\textit{WRITE}

\subsection{Implementation}

\textit{WRITE - details specific to this container}

\subsection{Interface} \label{sec-vec-cont-inter}

\subsubsection{Constructors}

\noindent
\textbf{vector}%
\texttt{%
(void)
}
 
\noindent
\textbf{vector}%
\texttt{%
(size\_t
\textit{n}%
)
}

\begin{itemize}
\item
Construct a vector with of size n. 
\end{itemize}
 
\noindent
\textbf{vector}%
\texttt{%
(size\_t
\textit{n,}%
value\_type const \&
\textit{default\_value}%
)
}

\begin{itemize}
\item
Construct a vector with of size n filled with value default\_value. 
\end{itemize}
 
\noindent
\textbf{vector}%
\texttt{%
(partition\_type const \&
\textit{ps}%
)
}

\begin{itemize}
\item
Construct a vector given a partitioner. 
\end{itemize}
 
\noindent
\textbf{vector}%
\texttt{%
(size\_t 
\textit{n,}%
mapper\_type const \&
\textit{mapper}%
)
}

\begin{itemize}
\item
Construct a vector given a mapper. 
\end{itemize}
 
\noindent
\textbf{vector}%
\texttt{%
(partition\_type const \&
\textit{partitioner,}%
mapper\_type const \&
\textit{mapper}%
)
}

\begin{itemize}
\item
Construct a vector given a partitioner and a mapper. 
\end{itemize}
 
\noindent
\texttt{%
template<typename DistSpecView >
}
\newline
\textbf{vector}%
\texttt{%
(DistSpecView const \&
\textit{dist\_view,}%
typename boost::enable\_if< is\_distribution\_view< DistSpecView > >::type *=0
)
}

\vspace{0.4cm} \emph{fix Doxygen to omit enable\_if, not part of user interface}

\begin{itemize}
\item
Create a vector with a distribution that is specified by the dist\_view provided. 
\end{itemize}
 
\noindent
\texttt{%
template<typename DistSpecView >
}
\newline
\textbf{vector}%
\texttt{%
(DistSpecView const \&
\textit{dist\_view,}%
value\_type const \&
\textit{default\_value,}%
typename boost::enable\_if< is\_distribution\_view< DistSpecView > >::type *=0
)
}

\vspace{0.4cm} \emph{fix Doxygen to omit enable\_if, not part of user interface}

\begin{itemize}
\item
Create a vector with a distribution that is specified by the dist\_view provided, and initialize all elements to default\_value. 
\end{itemize}
 
\noindent
\texttt{%
template<typename DP >
}
\newline
\textbf{vector}%
\texttt{%
(size\_t 
\textit{n,}%
value\_type const \&
\textit{default\_value,}%
DP const \&
\textit{dis\_policy}%
)
}

\begin{itemize}
\item
Create a vector with a given size and default value where the value type of the container is itself a distributed container. 
\end{itemize}
 
\noindent
\texttt{%
template<typename X , typename Y >
}
\newline
\textbf{vector}%
\texttt{%
(boost::tuples::cons< X, Y > dims
)
}

\begin{itemize}
\item
Create a vector of vectors with given n-dimensional size. 
\end{itemize}
 
\noindent
\texttt{%
template<typename X , typename Y , typename DP >
}
\newline
\textbf{vector}%
\texttt{%
(boost::tuples::cons< X, Y > 
\textit{dims,}%
DP const \&
\textit{dis\_policy}%
)
}

\begin{itemize}
\item
Create a vector of vectors with given n-dimensional size. 
\end{itemize}
 
\noindent
\texttt{%
template<typename SizesView >
}
\newline
\textbf{vector}%
\texttt{%
(SizesView const \&
\textit{sizes\_view,}%
typename boost::enable\_if< boost::mpl::and\_< boost::is\_same< size\_type, 
typename SizesView::size\_type >, boost::mpl::not\_< is\_distribution\_view< SizesView > > > >::type *=0
)
}

\vspace{0.4cm} \emph{fix Doxygen to omit enable\_if, not part of user interface}

\begin{itemize}
\item
Constructor for composed containers. For an m level composed container, sizes\_view is an m-1 level composed view representing the sizes of the nested containers. 
\end{itemize}

\subsubsection{Element Manipulation}

\noindent
\texttt{%
reference
}
\textbf{operator[]}%
\texttt{%
(index\_type 
\textit{idx}%
)
}

\begin{itemize}
\item
Construct a reference to a specific index of the vector.
\end{itemize}
 
\noindent%
\texttt{%
const\_reference
}
\textbf{operator[]}%
\texttt{%
(index\_type 
\textit{idx}%
) const
}

\begin{itemize}
\item
Construct a const\_reference to a specific index of the vector.
\end{itemize}
 
\noindent
\texttt{%
reference
}
\textbf{front}%
\texttt{%
(void)
}

\begin{itemize}
\item
Construct a reference to the first element of the vector.
\end{itemize}
 
\noindent
\texttt{%
const\_reference
}
\textbf{front}%
\texttt{%
(void) const
}

\begin{itemize}
\item
Construct a const\_reference to the first element of the vector.
\end{itemize}
 
\noindent
\texttt{%
reference
}
\textbf{back}%
\texttt{%
(void)
}

\begin{itemize}
\item
Construct a reference to the last element of the vector.
\end{itemize}
 
\noindent
\texttt{%
const\_reference
}
\textbf{back}%
\texttt{%
(void) const
}

\begin{itemize}
\item
Construct a const\_reference to the last element of the vector.
\end{itemize}
 
%%%%%
 
\noindent
\texttt{%
iterator 	
}
\textbf{insert}%
\texttt{
(iterator const \&
\textit{pos,}%
value\_type const \&
\textit{value}%
)
}

\begin{itemize}
\item
Inserts the given value before the element pointed to by the iterator pos and returns an iterator to the new element.
\end{itemize}
 
\noindent
\texttt{%
iterator 	
}
\textbf{erase}%
\texttt{
(iterator const \&
\textit{pos}%
)
}
 	
\begin{itemize}
\item
Removes the element pointed by the iterator pos and returns an iterator pointing to the next element.
\end{itemize}
 
\noindent
\texttt{%
void 
}
\textbf{clear}%
\texttt{
(void)
}

\begin{itemize}
\item
Remove all elements from the container. 
\end{itemize}
 
\noindent
\texttt{%
void 	
}
\textbf{push\_front}%
\texttt{
(value\_type const \&
\textit{value}%
)
}

\begin{itemize}
\item
Inserts the given value at the beginning of the list.
\end{itemize}
 
\noindent
\texttt{%
void 	
}
\textbf{push\_back}%
\texttt{
(value\_type const \&
\textit{value}%
)
}

\begin{itemize}
\item
Inserts the given value at the end of the list.
\end{itemize}
 
\noindent
\texttt{%
void 	
}
\textbf{add}%
\texttt{
(value\_type const \&
\textit{value}%
)
}

\begin{itemize}
\item
Insert the given value into the list. 
\end{itemize}
 
\noindent
\texttt{%
void 	
}
\textbf{pop\_front}%
\texttt{
(void)
}
 	
\begin{itemize}
\item
Removes the first element of the list.
\end{itemize}
 
\noindent
\texttt{%
void 	
}
\textbf{pop\_back}%
\texttt{
(void)
}
 	
\begin{itemize}
\item
Removes the last element of the list.
\end{itemize}
 
\noindent
\texttt{%
void 	
}
\textbf{splice}%
\texttt{
(iterator 
\textit{pos,}%
list \&
\textit{pl,}%
iterator 
\textit{first,}%
iterator 
\textit{l}%
)
}
 	
\begin{itemize}
\item
Splits the list at the given position pos and inserts between the two split lists the sublist defined from first to last iterator in the specified list pl.
\end{itemize}

%%%%%
 
\noindent
\texttt{%
reference
}
\textbf{make\_reference}%
\texttt{%
(index\_type const \&
\textit{idx}%
)
}

\begin{itemize}
\item
Construct a reference to a specific index of the vector.
\end{itemize}
 
\noindent
\texttt{%
const\_reference
}
\textbf{make\_reference}%
\texttt{%
(index\_type const \&
\textit{idx}%
) const
}

\begin{itemize}
\item
Construct a const\_reference to a specific index of the vector.
\end{itemize}
 
\noindent
\texttt{%
iterator
}
\textbf{make\_iterator}%
\texttt{%
(gid\_type const \&
\textit{gid}%
)
}

\begin{itemize}
\item
Construct an iterator to a specific index of the vector.
\end{itemize}
 
\noindent
\texttt{%
const\_iterator 
}
\textbf{make\_const\_iterator}%
\texttt{%
(gid\_type const \&
gid
) const
}

\begin{itemize}
\item
Construct a const\_iterator to a specific index of the vector.
\end{itemize}
 
\noindent
\texttt{%
iterator
}
\textbf{begin}%
\texttt{%
(void)
}

\begin{itemize}
\item
Construct an iterator to the beginning of the vector.
\end{itemize}
 
\noindent
\texttt{%
const\_iterator
}
\textbf{begin}%
\texttt{%
(void) const
}

\begin{itemize}
\item
Construct a const\_iterator to the beginning of the vector.
\end{itemize}
 
\noindent
\texttt{%
const\_iterator
}
\textbf{cbegin}%
\texttt{%
(void) const
}

\begin{itemize}
\item
Construct a const\_iterator to the beginning of the vector.
\end{itemize}
 
\noindent
\texttt{%
iterator
}
\textbf{end}%
\texttt{%
(void)
}

\begin{itemize}
\item
Construct an iterator to one past the end of the vector.
\end{itemize}
 
\noindent
\texttt{%
const\_iterator
}
\textbf{end}%
\texttt{%
(void) const
}

\begin{itemize}
\item
Construct a const\_iterator to one past the end of the vector.
\end{itemize}
 
\noindent
\texttt{%
const\_iterator
}
\textbf{cend}%
\texttt{%
(void) const
}

\begin{itemize}
\item
Construct a const\_iterator to one past the end of the vector.
\end{itemize}

%%%%%
 
\noindent
\texttt{%
void
}
\textbf{set\_element}%
\texttt{%
(index\_type const \&
\textit{idx,}%
value\_type const \&
\textit{val}%
)
}

\begin{itemize}
\item
Sets the element specified by the index to the provided value.
\end{itemize}
 
\noindent
\texttt{%
value\_type
}
\textbf{get\_element}%
\texttt{%
(index\_type const \&
\textit{idx}%
) const
}

\begin{itemize}
\item
Returns the value of the element specified by the index.
\end{itemize}
 
\noindent
\texttt{%
future< value\_type >
}
\textbf{get\_element\_split}%
\texttt{%
(index\_type const \&
\textit{idx}%
)
}

\begin{itemize}
\item
Returns a stapl::future holding the value of the element specified by the index.
\end{itemize}
 
\noindent
\texttt{%
void
}
\textbf{apply\_set}%
\texttt{%
(gid\_type const \&
\textit{gid,}%
F const \&
\textit{f}%
)
}

\begin{itemize}
\item
Applies a function f to the element specified by the GID.
\end{itemize}
 
\noindent
\texttt{%
F::result\_type
}
\textbf{apply\_get}%
\texttt{%
(gid\_type const \&
\textit{gid,}%
F const \&
\textit{f}%
)
}

\begin{itemize}
\item
Applies a function f to the element specified by the GID, and returns the result.
\end{itemize}
 
\noindent
\texttt{%
F::result\_type
}
\textbf{apply\_get}%
\texttt{%
(gid\_type const \&
\textit{gid,}%
F const \&
\textit{f}%
) const
}

\begin{itemize}
\item
Applies a function f to the element specified by the GID, and returns the result.
\end{itemize}
 
 
\subsubsection{Memory and Domain Management}

\noindent
\texttt{%
domain\_type 	
}
\textbf{domain}%
\texttt{%
(void) const
}

\begin{itemize}
\item
Returns the domain of the container.
\end{itemize}
 
\begin{itemize}
\item
Destroy the distribution of the container (including all of its elements) and recreate the container with a different size. 
\end{itemize}
 
\noindent
\texttt{%
void 
}
\textbf{resize}%
\texttt{%
(size\_type 
\textit{n}%
)
}

\begin{itemize}
\item
Destroy the distribution of the container (including all of its elements) and recreate the container with a different size. 
\end{itemize}
 
\noindent
\texttt{%
template<typename DistSpecView >
void
}
\textbf{redistribute}%
\texttt{%
(DistSpecView const \&
}
\textit{dist\_view,}
\texttt{
typename std::enable\_if< is\_distribution\_view< DistSpecView >::value \&\&is\_view\_based< partition\_type >::value \&\&is\_view\_based< mapper\_type >::value >::type *=0)
}

\vspace{0.4cm} \emph{fix Doxygen to omit enable\_if, not part of user interface}

\begin{itemize}
\item
Redistribute the data stored in the container to match the distribution specified by the distribution view provided. 
\end{itemize}
 
\noindent
\texttt{%
void
}
\textbf{migrate}%
\texttt{%
(gid\_type const \&
\textit{gid,}
location\_type 
\textit{destination}
)
}

\begin{itemize}
\item
Migrates the element specified by the gid to the destination location. 
\end{itemize}
 
\noindent
\texttt{%
size\_type
}
\textbf{size}%
\texttt{%
(void) const
}

\begin{itemize}
\item
Returns the number of elements in the container. 
\end{itemize}
 
\noindent
\texttt{%
bool
}
\textbf{empty}%
\texttt{%
(void) const
}

\begin{itemize}
\item
Returns if the container has no element.
\end{itemize}
 
\noindent
\texttt{%
distribution\_type \& 
}
\textbf{distribution}%
\texttt{%
 (void)
}
 
\noindent
\texttt{%
distribution\_type const \& 
}
\textbf{distribution}%
\texttt{%
 (void) const
}
 
\noindent
\texttt{%
distribution\_type *
}
\textbf{get\_distribution}%
\texttt{%
(void)
}
 
\begin{itemize}
\item
Returns the data distribution of the container.
\end{itemize}

\noindent
\texttt{%
locality\_info 
}
\textbf{locality}%
\texttt{%
 (gid\_type gid)
}

\begin{itemize}
\item
Return locality information about the element specified by the gid. 
\end{itemize}
 
\noindent
\texttt{%
bool
}
\textbf{is\_local}%
\texttt{%
(gid\_type const \&
\textit{gid}%
)
}

\begin{itemize}
\item
Returns true if the element specified by the GID is stored on this location, or false otherwise. 
\end{itemize}


\subsection{Usage Example} \label{sec-vec-cont-use}

The following example shows how to use the \texttt{vector container}:
%%\newline\vspace{0.4cm}\newline\rule{12cm}{0.5mm}
\input{cont/vector.tex}
%%\noindent\rule{12cm}{0.5mm}\vspace{0.5cm}

\textbf{PLEASE NOTE: 
Initializing a container with a sequential loop, as in this example,
is for illustrative purposes only.
The normal way to do this is applying a parallel algorithm or 
basic parallel construct to a view over the container.  See Section
\ref{sec-vec-vw-use}
for the recommended usage.
}

\vspace{0.4cm} \textit{WRITE -comments on example}

\subsection{Implementation} \label{sec-vec-cont-impl}

\textit{WRITE}

\subsection{Performance} \label{sec-vec-cont-perf}

\begin{itemize}
\item
Fig. \ref{fig:vec-cont-constr-exper}
shows the performance of constructing a \stapl\ vector container
whose elements are atomic values, \stl\ containers, or \stapl\ containers.
\item
Fig. \ref{fig:vec-cont-assign-exper}
shows the performance of assigning values to a \stapl\ vector container
whose elements are atomic values, \stl\ containers, or \stapl\ containers.
\item
Fig. \ref{fig:vec-cont-access-exper}
shows the performance of accessing values from a \stapl\ vector container
whose elements are atomic values, \stl\ containers, or \stapl\ containers.
\end{itemize}

\textit{WRITE - complexity analysis}

\begin{figure}[p]
%%\includegraphics[scale=0.50]{figs/vector_cont_constr}
\caption{Construct vector Execution Time}
\label{fig:vec-cont-constr-exper}
\end{figure}

\begin{figure}[p]
%%\includegraphics[scale=0.50]{figs/vector_cont_assign}
\caption{Assign Values to vector Execution Time}
\label{fig:vec-cont-assign-exper}
\end{figure}

\begin{figure}[p]
%%\includegraphics[scale=0.50]{figs/vector_cont_access}
\caption{Access Values in vector Execution Time}
\label{fig:vec-cont-access-exper}
\end{figure}


% % % % % % % % % % % % % % % % % % % % % % % % % % % % % % % % % % % % % % % 

\section{ List Container} \label{sec-list-cont}
\index{list!container}
\index{container!list}
\index{list!constructor}

\subsection{Definition}

\textit{EXPAND - Distributed list container.}

\subsection{Relationship to \stl\ list}

\textit{WRITE}

\subsection{Relationship to other \stapl\ containers}

\textit{WRITE}

\subsection{Implementation}

\textit{WRITE - details specific to this container}

\subsection{Interface} \label{sec-list-cont-inter}

\subsubsection{Constructors}

\noindent
\textbf{list}%
\texttt{%
(void)
}
 
\begin{itemize}
\item
Constructs a list with no elements.
\end{itemize}
 
\noindent
\textbf{list}%
\texttt{%
(size\_t
\textit{n}%
)
}

\begin{itemize}
\item
Constructs a list with n default constructed elements.
\end{itemize}
 
\noindent
\textbf{list}%
\texttt{%
(size\_t 
\textit{n,}%
value\_type const \&
\textit{default\_value}%
)
}

\begin{itemize}
\item
Constructs a list with a given size and default value. 
\end{itemize}
 
\noindent
\textbf{list}%
\texttt{%
(size\_t 
\textit{n,}%
mapper\_type const \&
\textit{mapper}%
)
}

\begin{itemize}
\item
Construct a list given a mapper and a size parameter. 
\end{itemize}
 
\noindent
\textbf{list}%
\texttt{%
(partition\_type const \&
\textit{ps}%
)
}


\begin{itemize}
\item
Constructs a list with a given partitioner. 
\end{itemize}
 
\noindent
\textbf{list}%
\texttt{%
(partition\_type const \&
\textit{partitioner,}%
mapper\_type const \&
\textit{mapper}%
)
}

\begin{itemize}
\item
Constructs a list given a partitioner and a mapper. 
\end{itemize}
 
\noindent
\texttt{%
template<typename DistSpecView >
}
\newline
\textbf{list}%
\texttt{%
(DistSpecView const \&
\textit{dist\_view,}%
typename boost::enable\_if< is\_distribution\_view< DistSpecView > >::type *=0
)
}

\vspace{0.4cm} \emph{fix Doxygen to omit enable\_if, not part of user interface}
 
\begin{itemize}
\item
\emph{fix Doxygen and copy material here}
\end{itemize}
 
\noindent
\texttt{%
template<typename DistSpecView >
}
\newline
\textbf{list}%
\texttt{%
(DistSpecView const \&
\textit{dist\_view,}%
value\_type const \&
\textit{default\_value,}%
typename boost::enable\_if< is\_distribution\_view< DistSpecView > >::type *=0
)
}

\vspace{0.4cm} \emph{fix Doxygen to omit enable\_if, not part of user interface}
 
\noindent
\textbf{list}%
\texttt{%
(list const \&other
)
}

\begin{itemize}
\item
Copy constructs a list from another list container.
\end{itemize}
 
\noindent
\texttt{%
template<typename DP >
}
\newline
\textbf{list}%
\texttt{%
(size\_t 
\textit{n,}%
value\_type const \&
\textit{default\_value,}%
DP const \&
\textit{dis\_policy}%
)
}

\begin{itemize}
\item
Constructs a list with a given size and default value where the value type of the container is itself a distributed container. 
\end{itemize}
 
\noindent
\texttt{%
template<typename X , typename Y >
}
\newline
\textbf{list}%
\texttt{%
(boost::tuples::cons< X, Y > 
\textit{dims}%
)
}

\begin{itemize}
\item
Constructs a list of lists with given n-dimensional size. 
\end{itemize}
 
\noindent
\texttt{%
template<typename X , typename Y , typename DP >
}
\newline
\textbf{list}%
\texttt{%
(boost::tuples::cons< X, Y > 
\textit{dims,}%
DP const \&
\textit{dis\_policy}%
)
}

\begin{itemize}
\item
Constructs a list of lists with given n-dimensional size. 
\end{itemize}

\subsubsection{Element Manipulation}

\begin{itemize}
\item
Construct a reference to the first element of the array.
\end{itemize}
 
\noindent
\texttt{%
const\_reference
}
\textbf{front}%
\texttt{%
(void) const
}

\begin{itemize}
\item
Construct a const\_reference to the first element of the array.
\end{itemize}
 
\noindent
\texttt{%
reference
}
\textbf{back}%
\texttt{%
(void)
}

\begin{itemize}
\item
Construct a reference to the last element of the array.
\end{itemize}
 
\noindent
\texttt{%
const\_reference
}
\textbf{back}%
\texttt{%
(void) const
}

\begin{itemize}
\item
Construct a const\_reference to the last element of the array.
\end{itemize}

%%%%%
 
\noindent
\texttt{%
iterator 	
}
\textbf{insert}%
\texttt{
(iterator const \&
\textit{pos,}%
value\_type const \&
\textit{value}%
)
}

\begin{itemize}
\item
Inserts the given value before the element pointed to by the iterator pos and returns an iterator to the new element.
\end{itemize}
 
\noindent
\texttt{%
iterator 	
}
\textbf{erase}%
\texttt{
(iterator const \&
\textit{pos}%
)
}
 	
\begin{itemize}
\item
Removes the element pointed by the iterator pos and returns an iterator pointing to the next element.
\end{itemize}
 
\noindent
\texttt{%
void 
}
\textbf{clear}%
\texttt{
(void)
}

\begin{itemize}
\item
Remove all elements from the container. 
\end{itemize}
 
\noindent
\texttt{%
void 	
}
\textbf{push\_front}%
\texttt{
(value\_type const \&
\textit{value}%
)
}

\begin{itemize}
\item
Inserts the given value at the beginning of the list.
\end{itemize}
 
\noindent
\texttt{%
void 	
}
\textbf{push\_back}%
\texttt{
(value\_type const \&
\textit{value}%
)
}

\begin{itemize}
\item
Inserts the given value at the end of the list.
\end{itemize}
 
\noindent
\texttt{%
void 	
}
\textbf{add}%
\texttt{
(value\_type const \&
\textit{value}%
)
}

\begin{itemize}
\item
Insert the given value into the list. 
\end{itemize}
 
\noindent
\texttt{%
void 	
}
\textbf{pop\_front}%
\texttt{
(void)
}
 	
\begin{itemize}
\item
Removes the first element of the list.
\end{itemize}
 
\noindent
\texttt{%
void 	
}
\textbf{pop\_back}%
\texttt{
(void)
}
 	
\begin{itemize}
\item
Removes the last element of the list.
\end{itemize}
 
\noindent
\texttt{%
void 	
}
\textbf{splice}%
\texttt{
(iterator 
\textit{pos,}%
list \&
\textit{pl,}%
iterator 
\textit{first,}%
iterator 
\textit{l}%
)
}
 	
\begin{itemize}
\item
Splits the list at the given position pos and inserts between the two split lists the sublist defined from first to last iterator in the specified list pl.
\end{itemize}

%%%%%
 
\noindent
\texttt{%
reference
}
\textbf{make\_reference}%
\texttt{%
(index\_type const \&
\textit{idx}%
)
}

\begin{itemize}
\item
Construct a reference to a specific index of the array.
\end{itemize}
 
\noindent
\texttt{%
const\_reference
}
\textbf{make\_reference}%
\texttt{%
(index\_type const \&
\textit{idx}%
) const
}

\begin{itemize}
\item
Construct a const\_reference to a specific index of the array.
\end{itemize}
 
\noindent
\texttt{%
iterator
}
\textbf{make\_iterator}%
\texttt{%
(gid\_type const \&
\textit{gid}%
)
}

\begin{itemize}
\item
Construct an iterator to a specific index of the array.
\end{itemize}
 
\noindent
\texttt{%
const\_iterator 
}
\textbf{make\_const\_iterator}%
\texttt{%
(gid\_type const \&
gid
) const
}

\begin{itemize}
\item
Construct a const\_iterator to a specific index of the array.
\end{itemize}
 
\noindent
\texttt{%
iterator
}
\textbf{begin}%
\texttt{%
(void)
}

\begin{itemize}
\item
Construct an iterator to the beginning of the array.
\end{itemize}
 
\noindent
\texttt{%
const\_iterator
}
\textbf{begin}%
\texttt{%
(void) const
}

\begin{itemize}
\item
Construct a const\_iterator to the beginning of the array.
\end{itemize}
 
\noindent
\texttt{%
const\_iterator
}
\textbf{cbegin}%
\texttt{%
(void) const
}

\begin{itemize}
\item
Construct a const\_iterator to the beginning of the array.
\end{itemize}
 
\noindent
\texttt{%
iterator
}
\textbf{end}%
\texttt{%
(void)
}

\begin{itemize}
\item
Construct an iterator to one past the end of the array.
\end{itemize}
 
\noindent
\texttt{%
const\_iterator
}
\textbf{end}%
\texttt{%
(void) const
}

\begin{itemize}
\item
Construct a const\_iterator to one past the end of the array.
\end{itemize}
 
\noindent
\texttt{%
const\_iterator
}
\textbf{cend}%
\texttt{%
(void) const
}

\begin{itemize}
\item
Construct a const\_iterator to one past the end of the array.
\end{itemize}
 
\noindent
\texttt{%
void
}
\textbf{set\_element}%
\texttt{%
(index\_type const \&
\textit{idx,}%
value\_type const \&
\textit{val}%
)
}

\begin{itemize}
\item
Sets the element specified by the index to the provided value.
\end{itemize}
 
\noindent
\texttt{%
value\_type
}
\textbf{get\_element}%
\texttt{%
(index\_type const \&
\textit{idx}%
) const
}

\begin{itemize}
\item
Returns the value of the element specified by the index.
\end{itemize}
 
\noindent
\texttt{%
future< value\_type >
}
\textbf{get\_element\_split}%
\texttt{%
(index\_type const \&
\textit{idx}%
)
}

\begin{itemize}
\item
Returns a stapl::future holding the value of the element specified by the index.
\end{itemize}
 
\noindent
\texttt{%
void
}
\textbf{apply\_set}%
\texttt{%
(gid\_type const \&
\textit{gid,}%
F const \&
\textit{f}%
)
}

\begin{itemize}
\item
Applies a function f to the element specified by the GID.
\end{itemize}
 
\noindent
\texttt{%
F::result\_type
}
\textbf{apply\_get}%
\texttt{%
(gid\_type const \&
\textit{gid,}%
F const \&
\textit{f}%
)
}

\begin{itemize}
\item
Applies a function f to the element specified by the GID, and returns the result.
\end{itemize}
 
\noindent
\texttt{%
F::result\_type
}
\textbf{apply\_get}%
\texttt{%
(gid\_type const \&
\textit{gid,}%
F const \&
\textit{f}%
) const
}

\begin{itemize}
\item
Applies a function f to the element specified by the GID, and returns the result.
\end{itemize}
 
 
\subsubsection{Memory and Domain Management}

\noindent
\texttt{%
domain\_type 	
}
\textbf{domain}%
\texttt{%
(void) const
}

\begin{itemize}
\item
Returns the domain of the container.
\end{itemize}
 
\begin{itemize}
\item
Redistribute the data stored in the container to match the distribution specified by the distribution view provided. 
\end{itemize}
 
\noindent
\texttt{%
void
}
\textbf{migrate}%
\texttt{%
(gid\_type const \&
\textit{gid,}
location\_type 
\textit{destination}
)
}

\begin{itemize}
\item
Migrates the element specified by the gid to the destination location. 
\end{itemize}
 
\noindent
\texttt{%
size\_type
}
\textbf{size}%
\texttt{%
(void) const
}

\begin{itemize}
\item
Returns the number of elements in the container. 
\end{itemize}
 
\noindent
\texttt{%
bool
}
\textbf{empty}%
\texttt{%
(void) const
}

\begin{itemize}
\item
Returns if the container has no element.
\end{itemize}
 
\noindent
\texttt{%
distribution\_type \& 
}
\textbf{distribution}%
\texttt{%
 (void)
}
 
\noindent
\texttt{%
distribution\_type const \& 
}
\textbf{distribution}%
\texttt{%
 (void) const
}
 
\noindent
\texttt{%
distribution\_type *
}
\textbf{get\_distribution}%
\texttt{%
(void)
}
 
\begin{itemize}
\item
Returns the data distribution of the container.
\end{itemize}

\noindent
\texttt{%
locality\_info 
}
\textbf{locality}%
\texttt{%
 (gid\_type gid)
}

\begin{itemize}
\item
Return locality information about the element specified by the gid. 
\end{itemize}
 
\noindent
\texttt{%
bool
}
\textbf{is\_local}%
\texttt{%
(gid\_type const \&
\textit{gid}%
)
}

\begin{itemize}
\item
Returns true if the element specified by the GID is stored on this location, or false otherwise. 
\end{itemize}

\subsection{Usage Example} \label{sec-list-cont-use}

The following example shows how to use the \texttt{list container}:
%%\newline\vspace{0.4cm}\newline\rule{12cm}{0.5mm}
\input{cont/list.tex}
%%\noindent\rule{12cm}{0.5mm}\vspace{0.5cm}

\textbf{PLEASE NOTE: 
Initializing a container with a sequential loop, as in this example,
is for illustrative purposes only.
The normal way to do this is applying a parallel algorithm or 
basic parallel construct to a view over the container.  See Section
\ref{sec-list-vw-use}
for the recommended usage.
}

\vspace{0.4cm} \textit{WRITE -comments on example}

\subsection{Implementation} \label{sec-list-cont-impl}

\textit{WRITE}

\subsection{Performance} \label{sec-list-cont-perf}

\begin{itemize}
\item
Fig. \ref{fig:list-cont-constr-exper}
shows the performance of constructing a \stapl\ list container
whose elements are atomic values, \stl\ containers, or \stapl\ containers.
\item
Fig. \ref{fig:list-cont-assign-exper}
shows the performance of assigning values to a \stapl\ list container
whose elements are atomic values, \stl\ containers, or \stapl\ containers.
\item
Fig. \ref{fig:list-cont-access-exper}
shows the performance of accessing values from a \stapl\ list container
whose elements are atomic values, \stl\ containers, or \stapl\ containers.
\end{itemize}

\textit{WRITE - complexity analysis}

\begin{figure}[p]
%%\includegraphics[scale=0.50]{figs/list_cont_constr}
\caption{Construct list Execution Time}
\label{fig:list-cont-constr-exper}
\end{figure}

\begin{figure}[p]
%%\includegraphics[scale=0.50]{figs/list_cont_assign}
\caption{Assign Values to list Execution Time}
\label{fig:list-cont-assign-exper}
\end{figure}

\begin{figure}[p]
%%\includegraphics[scale=0.50]{figs/list_cont_access}
\caption{Access Values in list Execution Time}
\label{fig:list-cont-access-exper}
\end{figure}


% % % % % % % % % % % % % % % % % % % % % % % % % % % % % % % % % % % % % % % 

\section{ Static Graph Container} \label{sec-stgraf-cont}
\index{graph!static!container}
\index{container!graph!static}
\index{graph!static!constructor}

\subsection{Definition}

\textit{EXPAND - Distributed static graph container. Inherits all functionality from either undirected\_graph or directed\_graph.  Static graphs do not allow addition or deletion of vertices. The number of vertices must be known at construction. Edges may be added/deleted. Uses directedness selector to inherit from correct directed/undirected base. }
 
\subsection{Relationship to other \stapl\ containers}

\textit{WRITE}

\subsection{Implementation}

\textit{WRITE - details specific to this container}

\subsection{Interface} \label{sec-stgraf-cont-inter}

\subsubsection{Constructors}

\noindent
\textbf{graph}%
\texttt{%
(void)
}

\begin{itemize}
\item
Creates an empty graph.
\end{itemize}
 
\noindent
\textbf{graph}%
\texttt{%
(size\_t const \&
\textit{n}%
)
}

\begin{itemize}
\item
Creates a graph with a given size.
\end{itemize}
 
\noindent
\textbf{graph}%
\texttt{%
(size\_t const \&
\textit{n,}%
vertex\_property const \&
\textit{default\_value}%
)
}

\begin{itemize}
\item
Creates a graph with a given size and constructs all elements with a default value for vertex property.
\end{itemize}
 
\noindent
\textbf{graph}%
\texttt{%
(partition\_type const \&
\textit{ps,}%
vertex\_property const \&
\textit{default\_value=vertex\_property()}%
)
}

\begin{itemize}
\item
Creates a graph with a given partition and default value for vertex property.
\end{itemize}
 
\noindent
\textbf{graph}%
\texttt{%
(partition\_type const \&
\textit{ps,}%
mapper\_type const \&
\textit{m}%
)
}

\begin{itemize}
\item
Creates a graph with a given partition and mapper.
\end{itemize}
 
\noindent
\textbf{graph}%
\texttt{%
(partition\_type const \&
\textit{ps,}%
mapper\_type const \&
\textit{m,}%
vertex\_property const \&
\textit{default\_value}%
)
}

\begin{itemize}
\item
Creates a graph with a given partition and mapper, with default value for vertex property.
\end{itemize}
 
\noindent
\textbf{graph}%
\texttt{%
(size\_t const \&
\textit{n,}%
mapper\_type const \&
\textit{m,}%
vertex\_property const \&
\textit{default\_value=vertex\_property()}%
)
}

\begin{itemize}
\item
Creates a graph with a given size and mapper, with default value for vertex property. 
\end{itemize}
 
\noindent
\texttt{%
template<typename DistSpecsView >
}
\newline
\textbf{graph}%
\texttt{%
(DistSpecsView const \&
\textit{dist\_view,}%
typename boost::enable\_if< is\_distribution\_view< DistSpecsView > >::type *=0
)
}

\vspace{0.4cm} \emph{fix Doxygen to omit enable\_if, not part of user interface}

\begin{itemize}
\item
\emph{fix Doxygen and copy material here}
\end{itemize}
 
\noindent
\texttt{%
template<typename DistSpecsView >
}
\newline
\textbf{graph}%
\texttt{%
(DistSpecsView const \&
\textit{dist\_view,}%
vertex\_property const \&
\textit{default\_value,}%
typename boost::enable\_if< is\_distribution\_view< DistSpecsView > >::type *=0)
}

\vspace{0.4cm} \emph{fix Doxygen to omit enable\_if, not part of user interface}
 
\noindent
\texttt{%
template<typename DP >
}
\newline
\textbf{graph}%
\texttt{%
(size\_t 
\textit{n,}%
vertex\_property const \&
\textit{default\_value,}%
DP const \&
\textit{dis\_policy}%
)
}

\begin{itemize}
\item
Creates a graph with a given size and default value where the vertex\_property may itself be a distributed container. Required for pC composition. 
\end{itemize}
 
\noindent
\texttt{%
template<typename DP >
}
\newline
\textbf{graph}%
\texttt{%
(size\_t 
\textit{n,}%
DP const \&
\textit{dis\_policy}%
)
}

\begin{itemize}
\item
Creates composed distributed containers with a given distribution policy. Required for pC composition. 
\end{itemize}
 
\noindent
\texttt{%
template<typename X , typename Y >
}
\newline
\textbf{graph}%
\texttt{%
(boost::tuples::cons< X, Y > 
\textit{dims}%
)
}

\begin{itemize}
\item
Creates composed distributed containers with a given size-specifications. Required for pC composition. 
\end{itemize}
 
\noindent
\texttt{%
template<typename X , typename Y , typename DP >
}
\newline
\textbf{graph}%
\texttt{%
(boost::tuples::cons< X, Y > 
\textit{dims,}%
const DP \&
\textit{dis\_policy}%
)
}

\begin{itemize}
\item
Creates composed distributed containers with a given size-specifications. Required for pC composition. 
\end{itemize}
 
\noindent
\texttt{%
template<typename SizesView >
}
\newline
\textbf{graph}%
\texttt{%
(SizesView const \&
\textit{sizes\_view,}%
typename boost::enable\_if< boost::mpl::and\_< boost::is\_same< size\_type, 
typename SizesView::size\_type >, boost::mpl::not\_< is\_distribution\_view< SizesView > > > >::type *=0
)
}

\vspace{0.4cm} \emph{fix Doxygen to omit enable\_if, not part of user interface}

\begin{itemize}
\item
Constructor for composed containers. For an m-level composed container, sizes\_view is an m-1 level composed view representing the sizes of the nested containers. 
\end{itemize}



\subsection{Usage Example} \label{sec-stgraf-cont-use}

The following example shows how to use the \texttt{static graph container}:
%%\newline\vspace{0.4cm}\newline\rule{12cm}{0.5mm}
\input{cont/stgraf.tex}
%%\noindent\rule{12cm}{0.5mm}\vspace{0.5cm}

\textbf{PLEASE NOTE: 
Initializing a container with a sequential loop, as in this example,
is for illustrative purposes only.
The normal way to do this is applying a parallel algorithm or 
basic parallel construct to a view over the container.  See Section
\ref{sec-graf-vw-use}
for the recommended usage.
}

\vspace{0.4cm} \textit{WRITE -comments on example}

\subsection{Implementation} \label{sec-stgraf-cont-impl}

\textit{WRITE}

\subsection{Performance} \label{sec-stgraf-cont-perf}

\begin{itemize}
\item
Fig. \ref{fig:stgraf-cont-constr-exper}
shows the performance of constructing a \stapl\ static graph container
whose elements are atomic values, \stl\ containers, or \stapl\ containers.
\item
Fig. \ref{fig:stgraf-cont-assign-exper}
shows the performance of assigning values to a \stapl\ static graph container
whose elements are atomic values, \stl\ containers, or \stapl\ containers.
\item
Fig. \ref{fig:stgraf-cont-access-exper}
shows the performance of accessing values from a \stapl\ static graph container
whose elements are atomic values, \stl\ containers, or \stapl\ containers.
\end{itemize}

\textit{WRITE - complexity analysis}

\begin{figure}[p]
%%\includegraphics[scale=0.50]{figs/static graph_cont_constr}
\caption{Construct static graph Execution Time}
\label{fig:stgraf-cont-constr-exper}
\end{figure}

\begin{figure}[p]
%%\includegraphics[scale=0.50]{figs/static graph_cont_assign}
\caption{Assign Values to static graph Execution Time}
\label{fig:stgraf-cont-assign-exper}
\end{figure}

\begin{figure}[p]
%%\includegraphics[scale=0.50]{figs/static graph_cont_access}
\caption{Access Values in static graph Execution Time}
\label{fig:stgraf-cont-access-exper}
\end{figure}


% % % % % % % % % % % % % % % % % % % % % % % % % % % % % % % % % % % % % % % 

\section{Dynamic Graph Container} \label{sec-dygraf-cont}
\index{graph!dynamic!container}
\index{container!graph!dynamic}
\index{graph!dynamic!constructor}

\subsection{Definition}

\textit{EXPAND - Distributed dynamic graph that supports addition and deletion of vertices and edges.Inherits from stapl::graph and adds functionality to add/delete vertices.}

\subsection{Relationship to other \stapl\ containers}

\textit{WRITE}

\subsection{Implementation}

\textit{WRITE - details specific to this container}

\subsection{Interface} \label{sec-dygraf-cont-inter}

\subsubsection{Constructors}

\noindent
\textbf{dynamic\_graph}%
\texttt{%
(void)
}
 
\noindent
\textbf{dynamic\_graph}%
\texttt{%
(size\_t const \&
\textit{n}%
)
}

\begin{itemize}
\item
Creates a graph with a given size. 
\end{itemize}
 
\noindent
\textbf{dynamic\_graph}%
\texttt{%
(size\_t const \&
\textit{n,}%
VertexP const \&
\textit{default\_value}%
)
}

\begin{itemize}
\item
Creates a graph with a given size and constructs all elements with a default value for vertex property. 
\end{itemize}
 
\noindent
\textbf{dynamic\_graph}%
\texttt{%
(partition\_type const \&
\textit{ps,}%
VertexP const \& 
\textit{default\_value=VertexP()}%
)
}

\begin{itemize}
\item
Creates a graph with a given partition and default value for vertex property. 
\end{itemize}
 
\noindent
\textbf{dynamic\_graph}%
\texttt{%
(partition\_type const \&
\textit{ps,}%
mapper\_type const \&
\textit{mapper,}%
VertexP const \&
\textit{default\_value=VertexP()}%
)
}

\begin{itemize}
\item
\emph{fix Doxygen and copy material here}
\end{itemize}
 
\noindent
\texttt{%
template<typename DistSpecsView >
}
\newline
\textbf{dynamic\_graph}%
\texttt{%
(DistSpecsView const \&
\textit{dist\_view,}%
typename boost::enable\_if< is\_distribution\_view< DistSpecsView > >::type *=0
)
}

\vspace{0.4cm} \emph{fix Doxygen to omit enable\_if, not part of user interface}
 
\begin{itemize}
\item
\emph{fix Doxygen and copy material here}
\end{itemize}
 
\noindent
\texttt{%
template<typename DistSpecsView >
}
\newline
\textbf{dynamic\_graph}%
\texttt{%
(DistSpecsView const \&
\textit{dist\_view,}%
VertexP const \&
\textit{default\_value,}%
typename boost::enable\_if< is\_distribution\_view< DistSpecsView > >::type *=0
)
}

\vspace{0.4cm} \emph{fix Doxygen to omit enable\_if, not part of user interface}
 
\noindent
\texttt{%
template<typename DP >
}
\newline
\textbf{dynamic\_graph}%
\texttt{%
(size\_t 
\textit{n,}%
vertex\_property const \&
\textit{default\_value,}%
DP const \&
\textit{dis\_policy}%
)
}

\begin{itemize}
\item
Creates a graph with a given size and default value where the vertex\_property may itself be a distributed container. Required for pC composition. 
\end{itemize}
 
\noindent
\texttt{%
template<typename X , typename Y >
}
\newline
\textbf{dynamic\_graph}%
\texttt{%
(boost::tuples::cons< X, Y > 
\textit{dims}%
)
}

\begin{itemize}
\item
Creates composed distributed containers with a given size-specifications. Required for pC composition. 
\end{itemize}
 
\noindent
\texttt{%
template<typename X , typename Y , typename DP >
}
\newline
\textbf{dynamic\_graph}%
\texttt{%
(boost::tuples::cons< X, Y > 
\textit{dims,}%
DP const \&
\textit{dis\_policy}%
)
}

\begin{itemize}
\item
Creates composed distributed containers with a given size-specifications. Required for pC composition. 
\end{itemize}
 
\noindent
\texttt{%
template<typename SizesView >
}
\newline
\textbf{dynamic\_graph}%
\texttt{%
(SizesView const \&
\textit{sizes\_view,}%
typename boost::enable\_if< boost::mpl::and\_< boost::is\_same< size\_type,
 typename SizesView::size\_type >, boost::mpl::not\_< is\_distribution\_view< SizesView > > > >::type *=0
)
}

\begin{itemize}
\item
Constructor for composed containers. For an m-level composed container, sizes\_view is an m-1 level composed view representing the sizes of the nested containers. 
\end{itemize}

HERE

\subsection{Usage Example} \label{sec-dygraf-cont-use}

The following example shows how to use the \texttt{dynamic graph container}:
%%\newline\vspace{0.4cm}\newline\rule{12cm}{0.5mm}
\input{cont/dygraf.tex}
%%\noindent\rule{12cm}{0.5mm}\vspace{0.5cm}

\textbf{PLEASE NOTE: 
Initializing a container with a sequential loop, as in this example,
is for illustrative purposes only.
The normal way to do this is applying a parallel algorithm or 
basic parallel construct to a view over the container.  See Section
\ref{sec-graf-vw-use}
for the recommended usage.
}

\vspace{0.4cm} \textit{WRITE -comments on example}

\subsection{Implementation} \label{sec-dygraf-cont-impl}

\textit{WRITE}

\subsection{Performance} \label{sec-dygraf-cont-perf}

\begin{itemize}
\item
Fig. \ref{fig:dygraf-cont-constr-exper}
shows the performance of constructing a \stapl\ dynamic graph container
whose elements are atomic values, \stl\ containers, or \stapl\ containers.
\item
Fig. \ref{fig:dygraf-cont-assign-exper}
shows the performance of assigning values to a \stapl\ dynamic graph container
whose elements are atomic values, \stl\ containers, or \stapl\ containers.
\item
Fig. \ref{fig:dygraf-cont-access-exper}
shows the performance of accessing values from a \stapl\ dynamic graph container
whose elements are atomic values, \stl\ containers, or \stapl\ containers.
\end{itemize}

\textit{WRITE - complexity analysis}

\begin{figure}[p]
%%\includegraphics[scale=0.50]{figs/dynamic graph_cont_constr}
\caption{Construct dynamic graph Execution Time}
\label{fig:dygraf-cont-constr-exper}
\end{figure}

\begin{figure}[p]
%%\includegraphics[scale=0.50]{figs/dynamic graph_cont_assign}
\caption{Assign Values to dynamic graph Execution Time}
\label{fig:dygraf-cont-assign-exper}
\end{figure}

\begin{figure}[p]
%%\includegraphics[scale=0.50]{figs/dynamic graph_cont_access}
\caption{Access Values in dynamic graph Execution Time}
\label{fig:dygraf-cont-access-exper}
\end{figure}


% % % % % % % % % % % % % % % % % % % % % % % % % % % % % % % % % % % % % % % 

\section { Matrix Container} \label{sec-mat-cont}
\index{matrix!container}
\index{container!matrix}
\index{matrix!constructor}

\subsection{Definition}

\textit{EXPAND - Distributed dense matrix container.}

\subsection{Relationship to other \stapl\ containers}

\textit{WRITE}

\subsection{Implementation}

\textit{WRITE - details specific to this container}

\subsection{Interface} \label{sec-mat-cont-inter}

\subsubsection{Constructors}

\noindent
\textbf{matrix}%
\texttt{%
(size\_type const \&
\textit{n,}%
partition\_type const \&
\textit{ps}%
)
}

\begin{itemize}
\item
Create an matrix with a given size and partition, default constructing all elements. 
\end{itemize}
 
\noindent
\textbf{matrix}%
\texttt{%
(size\_type const \&
\textit{sizes}%
)
}

\begin{itemize}
\item
Create a matrix with a given size that default constructs all elements. 
\end{itemize}
 
\noindent
\textbf{matrix}%
\texttt{%
(size\_type const \&
\textit{sizes,}%
mapper\_type const \&
\textit{mapper}%
)
}

\begin{itemize}
\item
Create a matrix with a given size and mapper. 
\end{itemize}
 
\noindent
\textbf{matrix}%
\texttt{%
(partition\_type const \&
\textit{partitioner,}%
mapper\_type const \&
\textit{mapper}%
)
}

\begin{itemize}
\item
Create a matrix with a given mapper and partition. 
\end{itemize}
 
\noindent
\textbf{%
matrix 
}
\texttt{%
(size\_type const \&
\textit{sizes,}%
value\_type const \&
\textit{default\_value}%
)
}

\begin{itemize}
\item
Create a matrix with a given size and default value default constructing all elements. 
\end{itemize}
 
\noindent
\textbf{matrix}%
\texttt{%
(partition\_type const \&
\textit{ps}%
)
}

\begin{itemize}
\item
Create a matrix with a given partition. 
\end{itemize}

\subsubsection{Element Manipulation}

\noindent
\texttt{%
reference
}
\textbf{operator[]}%
\texttt{%
(index\_type 
\textit{idx}%
)
}

\begin{itemize}
\item
Construct a reference to a specific index of the matrix.
\end{itemize}
 
\noindent%
\texttt{%
const\_reference
}
\textbf{operator[]}%
\texttt{%
(index\_type 
\textit{idx}%
) const
}

\begin{itemize}
\item
Construct a const\_reference to a specific index of the matrix.
\end{itemize}
 
\noindent
\texttt{%
void
}
\textbf{set\_element}%
\texttt{%
(index\_type const \&
\textit{idx,}%
value\_type const \&
\textit{val}%
)
}

\begin{itemize}
\item
Sets the element specified by the index to the provided value.
\end{itemize}
 
\noindent
\texttt{%
value\_type
}
\textbf{get\_element}%
\texttt{%
(index\_type const \&
\textit{idx}%
) const
}

\begin{itemize}
\item
Returns the value of the element specified by the index.
\end{itemize}
 
\noindent
\texttt{%
future< value\_type >
}
\textbf{get\_element\_split}%
\texttt{%
(index\_type const \&
\textit{idx}%
)
}

\begin{itemize}
\item
Returns a stapl::future holding the value of the element specified by the index.
\end{itemize}
 
\noindent
\texttt{%
void
}
\textbf{apply\_set}%
\texttt{%
(gid\_type const \&
\textit{gid,}%
F const \&
\textit{f}%
)
}

\begin{itemize}
\item
Applies a function f to the element specified by the GID.
\end{itemize}
 
\noindent
\texttt{%
F::result\_type
}
\textbf{apply\_get}%
\texttt{%
(gid\_type const \&
\textit{gid,}%
F const \&
\textit{f}%
)
}

\begin{itemize}
\item
Applies a function f to the element specified by the GID, and returns the result.
\end{itemize}
 
\noindent
\texttt{%
F::result\_type
}
\textbf{apply\_get}%
\texttt{%
(gid\_type const \&
\textit{gid,}%
F const \&
\textit{f}%
) const
}

\begin{itemize}
\item
Applies a function f to the element specified by the GID, and returns the result.
\end{itemize}
 
 
\subsubsection{Memory and Domain Management}

\noindent
\texttt{%
dimensions\_type
}
\textbf{dimensions}
\texttt{%
() const
}

\begin{itemize}
\item
Return the size of the matrix in each dimension.
\end{itemize}
 
\noindent
\texttt{%
domain\_type 	
}
\textbf{domain}%
\texttt{%
(void) const
}

\begin{itemize}
\item
Returns the domain of the container.
\end{itemize}
 
\noindent
\texttt{%
void 
}
\textbf{resize}%
\texttt{%
(size\_type 
\textit{n}
)
}

\begin{itemize}
\item
Destroy the distribution of the container (including all of its elements) and recreate the container with a different size.
\end{itemize}
 
\noindent
\texttt{%
template<typename DistSpecView >
void
}
\textbf{redistribute}%
\texttt{%
(DistSpecView const \&
}
\textit{dist\_view,}
\texttt{
typename std::enable\_if< is\_distribution\_view< DistSpecView >::value \&\&is\_view\_based< partition\_type >::value \&\&is\_view\_based< mapper\_type >::value >::type *=0)
}

\vspace{0.4cm} \emph{fix Doxygen to omit enable\_if, not part of user interface}

\begin{itemize}
\item
Redistribute the data stored in the container to match the distribution specified by the distribution view provided. 
\end{itemize}
 
\noindent
\texttt{%
void
}
\textbf{migrate}%
\texttt{%
(gid\_type const \&
\textit{gid,}
location\_type 
\textit{destination}
)
}

\begin{itemize}
\item
Migrates the element specified by the gid to the destination location. 
\end{itemize}
 
\noindent
\texttt{%
size\_type
}
\textbf{size}%
\texttt{%
(void) const
}

\begin{itemize}
\item
Returns the number of elements in the container. 
\end{itemize}
 
\noindent
\texttt{%
bool
}
\textbf{empty}%
\texttt{%
(void) const
}

\begin{itemize}
\item
Returns if the container has no element.
\end{itemize}
 
\noindent
\texttt{%
distribution\_type \& 
}
\textbf{distribution}%
\texttt{%
 (void)
}
 
\noindent
\texttt{%
distribution\_type const \& 
}
\textbf{distribution}%
\texttt{%
 (void) const
}
 
\noindent
\texttt{%
distribution\_type *
}
\textbf{get\_distribution}%
\texttt{%
(void)
}
 
\begin{itemize}
\item
Returns the data distribution of the container.
\end{itemize}

\noindent
\texttt{%
locality\_info 
}
\textbf{locality}%
\texttt{%
 (gid\_type gid)
}

\begin{itemize}
\item
Return locality information about the element specified by the gid. 
\end{itemize}
 
\noindent
\texttt{%
bool
}
\textbf{is\_local}%
\texttt{%
(gid\_type const \&
\textit{gid}%
)
}

\begin{itemize}
\item
Returns true if the element specified by the GID is stored on this location, or false otherwise. 
\end{itemize}

\subsection{Usage Example} \label{sec-mat-cont-use}

The following example shows how to use the \texttt{matrix container}:
%%\newline\vspace{0.4cm}\newline\rule{12cm}{0.5mm}
\input{cont/matrix.tex}
%%\noindent\rule{12cm}{0.5mm}\vspace{0.5cm}

\textbf{PLEASE NOTE: 
Initializing a container with a sequential loop, as in this example,
is for illustrative purposes only.
The normal way to do this is applying a parallel algorithm or 
basic parallel construct to a view over the container.  See Section
\ref{sec-mat-vw-use}
for the recommended usage.
}

\vspace{0.4cm} \textit{WRITE -comments on example}

\subsection{Implementation} \label{sec-mat-cont-impl}

\textit{WRITE}

\subsection{Performance} \label{sec-mat-cont-perf}

\begin{itemize}
\item
Fig. \ref{fig:mat-cont-constr-exper}
shows the performance of constructing a \stapl\ matrix container
whose elements are atomic values, \stl\ containers, or \stapl\ containers.
\item
Fig. \ref{fig:mat-cont-assign-exper}
shows the performance of assigning values to a \stapl\ matrix container
whose elements are atomic values, \stl\ containers, or \stapl\ containers.
\item
Fig. \ref{fig:mat-cont-access-exper}
shows the performance of accessing values from a \stapl\ matrix container
whose elements are atomic values, \stl\ containers, or \stapl\ containers.
\end{itemize}

\textit{WRITE - complexity analysis}

\begin{figure}[p]
%%\includegraphics[scale=0.50]{figs/matrix_cont_constr}
\caption{Construct matrix Execution Time}
\label{fig:mat-cont-constr-exper}
\end{figure}

\begin{figure}[p]
%%\includegraphics[scale=0.50]{figs/matrix_cont_assign}
\caption{Assign Values to matrix Execution Time}
\label{fig:mat-cont-assign-exper}
\end{figure}

\begin{figure}[p]
%%\includegraphics[scale=0.50]{figs/matrix_cont_access}
\caption{Access Values in matrix Execution Time}
\label{fig:mat-cont-access-exper}
\end{figure}


% % % % % % % % % % % % % % % % % % % % % % % % % % % % % % % % % % % % % % % 

\section{ Multiarray Container} \label{sec-multi-cont}
\index{multiarray!container}
\index{container!multiarray}
\index{multiarray!constructor}

\subsection{Definition}

\textit{EXPAND - Distributed multi-dimensional array container. }

\subsection{Relationship to other \stapl\ containers}

\textit{WRITE}

\subsection{Implementation}

\textit{WRITE - details specific to this container}

\subsection{Interface} \label{sec-multi-cont-inter}

\subsubsection{Constructors}

\noindent
\textbf{multiarray}%
\texttt{%
(void)
}

\begin{itemize}
\item
Default constructor creates an empty multiarray with zero size in all dimensions.
\end{itemize}
 
\noindent
\textbf{multiarray}%
\texttt{%
(size\_type const \&
\textit{sizes}%
)
}

\begin{itemize}
\item
Create an multiarray with a given size and default construct all elements. 
\end{itemize}
 
\noindent
\textbf{multiarray}%
\texttt{%
(size\_type const \&
\textit{sizes,}%
value\_type const \&
\textit{default\_value}%
)
}

\begin{itemize}
\item
Create a multiarray with given sizes in each dimension and a default value. 
\end{itemize}
 
\noindent
\textbf{multiarray}%
\texttt{%
(size\_type const \&
\textit{sizes,}%
mapper\_type const \&
\textit{mapper}%
)
}

\begin{itemize}
\item
Create a multiarray with a given sizes in each dimension and a mapper. 
\end{itemize}
 
\noindent
\textbf{multiarray}%
\texttt{%
(partition\_type const \&
\textit{ps}%
)
}

\begin{itemize}
\item
Create an multiarray with a given partition. 
\end{itemize}
 
\noindent
\textbf{multiarray}%
\texttt{%
(partition\_type const \&
\textit{partitioner,}%
mapper\_type const \&
\textit{mapper}%
)
}

\begin{itemize}
\item
Create an multiarray with a given partitioner and mapper, default constructing all elements. 
\end{itemize}
 
\noindent
\texttt{%
template<typename DistSpecsView >
}
\newline
\textbf{multiarray}%
\texttt{%
(DistSpecsView const \&
\textit{dist\_view,}%
typename boost::enable\_if< is\_distribution\_view< DistSpecsView > >::type *=0
)
}

\vspace{0.4cm} \emph{fix Doxygen to omit enable\_if, not part of user interface}
 
\begin{itemize}
\item
\emph{fix Doxygen and copy material here}
\end{itemize}
 
\noindent
\texttt{%
template<typename SizesView >
}
\newline
\textbf{multiarray}%
\texttt{%
(SizesView const \&
\textit{sizes\_view,}%
typename boost::enable\_if< boost::mpl::and\_< boost::is\_same< size\_type, 
typename SizesView::size\_type >, 
boost::mpl::not\_< is\_distribution\_view< SizesView > > > >::type *=0
)
}

\vspace{0.4cm} \emph{fix Doxygen to omit enable\_if, not part of user interface}

\begin{itemize}
\item
Constructor for composed containers. For an m level composed container, sizes\_view is an m-1 level composed view representing the sizes of the nested containers. 
\end{itemize}

\subsubsection{Element Manipulation}

\noindent
\texttt{%
reference
}
\textbf{operator[]}%
\texttt{%
(index\_type 
\textit{idx}%
)
}

\begin{itemize}
\item
Construct a reference to a specific index of the multiarray.
\end{itemize}
 
\noindent%
\texttt{%
const\_reference
}
\textbf{operator[]}%
\texttt{%
(index\_type 
\textit{idx}%
) const
}

\begin{itemize}
\item
Construct a const\_reference to a specific index of the multiarray.
\end{itemize}
 
\noindent
\texttt{%
void
}
\textbf{set\_element}%
\texttt{%
(index\_type const \&
\textit{idx,}%
value\_type const \&
\textit{val}%
)
}

\begin{itemize}
\item
Sets the element specified by the index to the provided value.
\end{itemize}
 
\noindent
\texttt{%
value\_type
}
\textbf{get\_element}%
\texttt{%
(index\_type const \&
\textit{idx}%
) const
}

\begin{itemize}
\item
Returns the value of the element specified by the index.
\end{itemize}
 
\noindent
\texttt{%
future< value\_type >
}
\textbf{get\_element\_split}%
\texttt{%
(index\_type const \&
\textit{idx}%
)
}

\begin{itemize}
\item
Returns a stapl::future holding the value of the element specified by the index.
\end{itemize}
 
\noindent
\texttt{%
void
}
\textbf{apply\_set}%
\texttt{%
(gid\_type const \&
\textit{gid,}%
F const \&
\textit{f}%
)
}

\begin{itemize}
\item
Applies a function f to the element specified by the GID.
\end{itemize}
 
\noindent
\texttt{%
F::result\_type
}
\textbf{apply\_get}%
\texttt{%
(gid\_type const \&
\textit{gid,}%
F const \&
\textit{f}%
)
}

\begin{itemize}
\item
Applies a function f to the element specified by the GID, and returns the result.
\end{itemize}
 
\noindent
\texttt{%
F::result\_type
}
\textbf{apply\_get}%
\texttt{%
(gid\_type const \&
\textit{gid,}%
F const \&
\textit{f}%
) const
}

\begin{itemize}
\item
Applies a function f to the element specified by the GID, and returns the result.
\end{itemize}
 
 
\subsubsection{Memory and Domain Management}

\noindent
\texttt{%
dimensions\_type
}
\textbf{dimensions}
\texttt{%
() const
}

\begin{itemize}
\item
Return the size of the multiarray in each dimension.
\end{itemize}
 
\noindent
\texttt{%
domain\_type 	
}
\textbf{domain}%
\texttt{%
(void) const
}

\begin{itemize}
\item
Returns the domain of the container.
\end{itemize}
 
\noindent
\texttt{%
void 
}
\textbf{resize}%
\texttt{%
(size\_type 
\textit{n}
)
}

\begin{itemize}
\item
Destroy the distribution of the container (including all of its elements) and recreate the container with a different size. 
\end{itemize}
 
\noindent
\texttt{%
template<typename DistSpecView >
void
}
\textbf{redistribute}%
\texttt{%
(DistSpecView const \&
}
\textit{dist\_view,}
\texttt{
typename std::enable\_if< is\_distribution\_view< DistSpecView >::value \&\&is\_view\_based< partition\_type >::value \&\&is\_view\_based< mapper\_type >::value >::type *=0)
}

\vspace{0.4cm} \emph{fix Doxygen to omit enable\_if, not part of user interface}

\begin{itemize}
\item
Redistribute the data stored in the container to match the distribution specified by the distribution view provided. 
\end{itemize}
 
\noindent
\texttt{%
void
}
\textbf{migrate}%
\texttt{%
(gid\_type const \&
\textit{gid,}
location\_type 
\textit{destination}
)
}

\begin{itemize}
\item
Migrates the element specified by the gid to the destination location. 
\end{itemize}
 
\noindent
\texttt{%
size\_type
}
\textbf{size}%
\texttt{%
(void) const
}

\begin{itemize}
\item
Returns the number of elements in the container. 
\end{itemize}
 
\noindent
\texttt{%
bool
}
\textbf{empty}%
\texttt{%
(void) const
}

\begin{itemize}
\item
Returns if the container has no element.
\end{itemize}
 
\noindent
\texttt{%
distribution\_type \& 
}
\textbf{distribution}%
\texttt{%
 (void)
}
 
\noindent
\texttt{%
distribution\_type const \& 
}
\textbf{distribution}%
\texttt{%
 (void) const
}
 
\noindent
\texttt{%
distribution\_type *
}
\textbf{get\_distribution}%
\texttt{%
(void)
}
 
\begin{itemize}
\item
Returns the data distribution of the container.
\end{itemize}

\noindent
\texttt{%
locality\_info 
}
\textbf{locality}%
\texttt{%
 (gid\_type gid)
}

\begin{itemize}
\item
Return locality information about the element specified by the gid. 
\end{itemize}
 
\noindent
\texttt{%
bool
}
\textbf{is\_local}%
\texttt{%
(gid\_type const \&
\textit{gid}%
)
}

\begin{itemize}
\item
Returns true if the element specified by the GID is stored on this location, or false otherwise. 
\end{itemize}


\subsection{Usage Example} \label{sec-multi-cont-use}

The following example shows how to use the \texttt{multiarray container}:
%%\newline\vspace{0.4cm}\newline\rule{12cm}{0.5mm}
\input{cont/multi.tex}
%%\noindent\rule{12cm}{0.5mm}\vspace{0.5cm}

\textbf{PLEASE NOTE: 
Initializing a container with a sequential loop, as in this example,
is for illustrative purposes only.
The normal way to do this is applying a parallel algorithm or 
basic parallel construct to a view over the container.  See Section
\ref{sec-multi-vw-use}
for the recommended usage.
}

\vspace{0.4cm} \textit{WRITE -comments on example}

\subsection{Implementation} \label{sec-multi-cont-impl}

\textit{WRITE}

\subsection{Performance} \label{sec-multi-cont-perf}

\begin{itemize}
\item
Fig. \ref{fig:multi-cont-constr-exper}
shows the performance of constructing a \stapl\ multiarray container
whose elements are atomic values, \stl\ containers, or \stapl\ containers.
\item
Fig. \ref{fig:multi-cont-assign-exper}
shows the performance of assigning values to a \stapl\ multiarray container
whose elements are atomic values, \stl\ containers, or \stapl\ containers.
\item
Fig. \ref{fig:multi-cont-access-exper}
shows the performance of accessing values from a \stapl\ multiarray container
whose elements are atomic values, \stl\ containers, or \stapl\ containers.
\end{itemize}

\textit{WRITE - complexity analysis}

\begin{figure}[p]
%%\includegraphics[scale=0.50]{figs/multiarray_cont_constr}
\caption{Construct multiarray Execution Time}
\label{fig:multi-cont-constr-exper}
\end{figure}

\begin{figure}[p]
%%\includegraphics[scale=0.50]{figs/multiarray_cont_assign}
\caption{Assign Values to multiarray Execution Time}
\label{fig:multi-cont-assign-exper}
\end{figure}

\begin{figure}[p]
%%\includegraphics[scale=0.50]{figs/multiarray_cont_access}
\caption{Access Values in multiarray Execution Time}
\label{fig:multi-cont-access-exper}
\end{figure}


% % % % % % % % % % % % % % % % % % % % % % % % % % % % % % % % % % % % % % % 

\section{ Set Container} \label{sec-set-cont}
\index{set!container}
\index{container!set}
\index{set!constructor}

\subsection{Definition}

\textit{EXPAND - Distributed set container}

\subsection{Relationship to \stl\ set}

\textit{WRITE}

\subsection{Relationship to other \stapl\ containers}

\textit{WRITE}

\subsection{Implementation}

\textit{WRITE - details specific to this container}

\subsection{Interface} \label{sec-set-cont-inter}

\subsubsection{Constructors}

\noindent
\textbf{set}%
\texttt{%
(void)
}

\begin{itemize}
\item
Construct a parallel set where the ownership of keys is determined by a simple block-cyclic distribution of the key space. 
\end{itemize}
 
\noindent
\textbf{set}%
\texttt{%
(typename Partitioner::value\_type const \&
\textit{domain}%
)
}

\begin{itemize}
\item
Create a set with a given domain.
\end{itemize}
 
\noindent
\textbf{set}%
\texttt{%
(Partitioner const \&
\textit{partition}%
)
}

\begin{itemize}
\item
Construct a parallel set where the ownership of keys is determined by a given partition. 
\end{itemize}
 
\noindent
\textbf{set}%
\texttt{%
(Partitioner const \&
\textit{partitioner,}%
Mapper const \&
\textit{mapper}%
)
}

\begin{itemize}
\item
Create a set with a given partitioner and mapper an instance of mapper. More
\end{itemize}

\subsubsection{Element Manipulation}

%%\noindent
%%\texttt{%
%%reference
%%}
%%\textbf{operator[]}%
%%\texttt{%
%%(index\_type 
%%\textit{idx}%
%%)
%%}

%%\begin{itemize}
%%\item
%%Construct a reference to a specific key.
%%\end{itemize}
 
%%\noindent%
%%\texttt{%
%%const\_reference
%%}
%%\textbf{operator[]}%
%%\texttt{%
%%(index\_type 
%%\textit{idx}%
%%) const
%%}

%%\begin{itemize}
%%\item
%%Construct a const\_reference to a specific key;
%%\end{itemize}
 
%%%%%

\noindent
\texttt{%
reference
}
\textbf{make\_reference}%
\texttt{%
(index\_type const \&
\textit{idx}%
)
}

\begin{itemize}
\item
Construct a reference to a specific key.
\end{itemize}
 
\noindent
\texttt{%
const\_reference
}
\textbf{make\_reference}%
\texttt{%
(index\_type const \&
\textit{idx}%
) const
}

\begin{itemize}
\item
Construct a const\_reference to a specific key.
\end{itemize}
 
\noindent
\texttt{%
iterator
}
\textbf{make\_iterator}%
\texttt{%
(gid\_type const \&
\textit{gid}%
)
}

\begin{itemize}
\item
Construct an iterator to a specific key.
\end{itemize}
 
\noindent
\texttt{%
const\_iterator 
}
\textbf{make\_const\_iterator}%
\texttt{%
(gid\_type const \&
gid
) const
}

\begin{itemize}
\item
Construct a const\_iterator to a specific key;
\end{itemize}
 
\texttt{%
iterator
}
\textbf{begin}%
\texttt{%
(void)
}

\begin{itemize}
\item
Construct an iterator to the first element of the domain.
\end{itemize}
 
\noindent
\texttt{%
const\_iterator
}
\textbf{begin}%
\texttt{%
(void) const
}

\begin{itemize}
\item
Construct a const\_iterator to the first element of the domain.
\end{itemize}
 
\noindent
\texttt{%
const\_iterator
}
\textbf{cbegin}%
\texttt{%
(void) const
}

\begin{itemize}
\item
Construct a const\_iterator to the first element of the domain.
\end{itemize}
 
\noindent
\texttt{%
iterator
}
\textbf{end}%
\texttt{%
(void)
}

\begin{itemize}
\item
Construct an iterator to one past the last element of the domain.
\end{itemize}
 
\noindent
\texttt{%
const\_iterator
}
\textbf{end}%
\texttt{%
(void) const
}

\begin{itemize}
\item
Construct a const\_iterator to one past the last element of the domain.
\end{itemize}
 
\noindent
\texttt{%
const\_iterator
}
\textbf{cend}%
\texttt{%
(void) const
}

\begin{itemize}
\item
Construct a const\_iterator to one past the last element of the domain.
\end{itemize}
 
\noindent
\texttt{%
void
}
\textbf{set\_element}%
\texttt{%
(index\_type const \&
\textit{idx,}%
value\_type const \&
\textit{val}%
)
}

\begin{itemize}
\item
Sets the element specified by the index to the provided value.
\end{itemize}
 
\noindent
\texttt{%
value\_type
}
\textbf{get\_element}%
\texttt{%
(index\_type const \&
\textit{idx}%
) const
}

\begin{itemize}
\item
Returns the value of the element specified by the key.
\end{itemize}
 
\noindent
\texttt{%
future< value\_type >
}
\textbf{get\_element\_split}%
\texttt{%
(index\_type const \&
\textit{idx}%
)
}

\begin{itemize}
\item
Returns a stapl::future holding the value of the element specified by the key.
\end{itemize}
 
\noindent
\texttt{%
void
}
\textbf{apply\_set}%
\texttt{%
(gid\_type const \&
\textit{gid,}%
F const \&
\textit{f}%
)
}

\begin{itemize}
\item
Applies a function f to the element specified by the GID.
\end{itemize}
 
\noindent
\texttt{%
F::result\_type
}
\textbf{apply\_get}%
\texttt{%
(gid\_type const \&
\textit{gid,}%
F const \&
\textit{f}%
)
}

\begin{itemize}
\item
Applies a function f to the element specified by the GID, and returns the result.
\end{itemize}
 
\noindent
\texttt{%
F::result\_type
}
\textbf{apply\_get}%
\texttt{%
(gid\_type const \&
\textit{gid,}%
F const \&
\textit{f}%
) const
}

\begin{itemize}
\item
Applies a function f to the element specified by the GID, and returns the result.
\end{itemize}
 
%%%%%

\subsubsection{Memory and Domain Management}

\noindent
\texttt{%
domain\_type 	
}
\textbf{domain}%
\texttt{%
(void) const
}

\begin{itemize}
\item
Returns the domain of the container.
\end{itemize}
 
\noindent
\texttt{%
template<typename DistSpecView >
void
}
\textbf{redistribute}%
\texttt{%
(DistSpecView const \&
}
\textit{dist\_view,}
\texttt{
typename std::enable\_if< is\_distribution\_view< DistSpecView >::value \&\&is\_view\_based< partition\_type >::value \&\&is\_view\_based< mapper\_type >::value >::type *=0)
}

\vspace{0.4cm} \emph{fix Doxygen to omit enable\_if, not part of user interface}

\begin{itemize}
\item
Redistribute the data stored in the container to match the distribution specified by the distribution view provided. 
\end{itemize}
 
\noindent
\texttt{%
void
}
\textbf{migrate}%
\texttt{%
(gid\_type const \&
\textit{gid,}
location\_type 
\textit{destination}
)
}

\begin{itemize}
\item
Migrates the element specified by the gid to the destination location. 
\end{itemize}
 
\noindent
\texttt{%
size\_type
}
\textbf{size}%
\texttt{%
(void) const
}

\begin{itemize}
\item
Returns the number of elements in the container. 
\end{itemize}
 
\noindent
\texttt{%
bool
}
\textbf{empty}%
\texttt{%
(void) const
}

\begin{itemize}
\item
Returns if the container has no element.
\end{itemize}
 
\noindent
\texttt{%
distribution\_type \& 
}
\textbf{distribution}%
\texttt{%
 (void)
}
 
\noindent
\texttt{%
distribution\_type const \& 
}
\textbf{distribution}%
\texttt{%
 (void) const
}
 
\noindent
\texttt{%
distribution\_type *
}
\textbf{get\_distribution}%
\texttt{%
(void)
}
 
\begin{itemize}
\item
Returns the data distribution of the container.
\end{itemize}

\noindent
\texttt{%
locality\_info 
}
\textbf{locality}%
\texttt{%
 (gid\_type gid)
}

\begin{itemize}
\item
Return locality information about the element specified by the gid. 
\end{itemize}
 
\noindent
\texttt{%
bool
}
\textbf{is\_local}%
\texttt{%
(gid\_type const \&
\textit{gid}%
)
}

\begin{itemize}
\item
Returns true if the element specified by the GID is stored on this location, or false otherwise. 
\end{itemize}


\subsection{Usage Example} \label{sec-set-cont-use}

The following example shows how to use the \texttt{set container}:
%%\newline\vspace{0.4cm}\newline\rule{12cm}{0.5mm}
\input{cont/set.tex}
%%\noindent\rule{12cm}{0.5mm}\vspace{0.5cm}

\textbf{PLEASE NOTE: 
Initializing a container with a sequential loop, as in this example,
is for illustrative purposes only.
The normal way to do this is applying a parallel algorithm or 
basic parallel construct to a view over the container.  See Section
\ref{sec-set-vw-use}
for the recommended usage.
}

\vspace{0.4cm} \textit{WRITE -comments on example}

\subsection{Implementation} \label{sec-set-cont-impl}

\textit{WRITE}

\subsection{Performance} \label{sec-set-cont-perf}

\begin{itemize}
\item
Fig. \ref{fig:set-cont-constr-exper}
shows the performance of constructing a \stapl\ set container
whose elements are atomic values, \stl\ containers, or \stapl\ containers.
\item
Fig. \ref{fig:set-cont-assign-exper}
shows the performance of assigning values to a \stapl\ set container
whose elements are atomic values, \stl\ containers, or \stapl\ containers.
\item
Fig. \ref{fig:set-cont-access-exper}
shows the performance of accessing values from a \stapl\ set container
whose elements are atomic values, \stl\ containers, or \stapl\ containers.
\end{itemize}

\textit{WRITE - complexity analysis}

\begin{figure}[p]
%%\includegraphics[scale=0.50]{figs/set_cont_constr}
\caption{Construct set Execution Time}
\label{fig:set-cont-constr-exper}
\end{figure}

\begin{figure}[p]
%%\includegraphics[scale=0.50]{figs/set_cont_assign}
\caption{Assign Values to set Execution Time}
\label{fig:set-cont-assign-exper}
\end{figure}

\begin{figure}[p]
%%\includegraphics[scale=0.50]{figs/set_cont_access}
\caption{Access Values in set Execution Time}
\label{fig:set-cont-access-exper}
\end{figure}


% % % % % % % % % % % % % % % % % % % % % % % % % % % % % % % % % % % % % % % 

\section{ Map Container} \label{sec-map-cont}
\index{map!container}
\index{container!map}
\index{map!constructor}

\subsection{Definition}

\textit{EXPAND - Distributed ordered map container.}

\subsection{Relationship to \stl\ map}

\textit{WRITE}

\subsection{Relationship to other \stapl\ containers}

\textit{WRITE}

\subsection{Implementation}

\textit{WRITE - details specific to this container}

\subsection{Interface} \label{sec-map-cont-inter}

\subsubsection{Constructors}

\noindent
\textbf{map}%
\texttt{%
()
}

\begin{itemize}
\item
Construct a parallel map where the ownership of keys is determined by a simple block-cyclic distribution of the key space. 
\end{itemize}
 
\noindent
\textbf{map}%
\texttt{%
(typename partition\_type::value\_type const \&
\textit{domain}%
)
}

\begin{itemize}
\item
Construct a parallel map where the ownership of keys is determined by a balanced partition of a given domain. 
\end{itemize}
 
\noindent
\textbf{map}%
\texttt{%
(partition\_type const \&
\textit{partition}%
)
}

\begin{itemize}
\item
Construct a parallel map where the ownership of keys is determined by a given partition. 
\end{itemize}
 
\noindent
\textbf{map}%
\texttt{%
(partition\_type const \&
\textit{partitioner,}%
mapper\_type const \&
\textit{mapper}%
)
}

\begin{itemize}
\item
Construct a parallel map where the ownership of keys is determined by a given partition and mapper. 
\end{itemize}

\subsubsection{Element Manipulation}

\noindent
\texttt{%
reference
}
\textbf{operator[]}%
\texttt{%
(index\_type 
\textit{idx}%
)
}

\begin{itemize}
\item
Construct a reference to the mapped value of a specific key.
\end{itemize}
 
\noindent%
\texttt{%
const\_reference
}
\textbf{operator[]}%
\texttt{%
(index\_type 
\textit{idx}%
) const
}

\begin{itemize}
\item
Construct a const\_reference to the mapped value of a specific key.
\end{itemize}
 
\noindent
\texttt{%
reference
}
\textbf{make\_reference}%
\texttt{%
(index\_type const \&
\textit{idx}%
)
}

\begin{itemize}
\item
Construct a reference to the mapped value of a specific key.
\end{itemize}
 
\noindent
\texttt{%
const\_reference
}
\textbf{make\_reference}%
\texttt{%
(index\_type const \&
\textit{idx}%
) const
}

\begin{itemize}
\item
Construct a const\_reference to the mapped value of a specific key.
\end{itemize}
 
\noindent
\texttt{%
iterator
}
\textbf{make\_iterator}%
\texttt{%
(gid\_type const \&
\textit{gid}%
)
}

\begin{itemize}
\item
Construct an iterator to the mapped value of a specific key.
\end{itemize}
 
\noindent
\texttt{%
const\_iterator 
}
\textbf{make\_const\_iterator}%
\texttt{%
(gid\_type const \&
gid
) const
}

\begin{itemize}
\item
Construct a const\_iterator to the mapped value of a specific key.
\end{itemize}

%%%%%
 
\noindent
\texttt{%
iterator 	
}
\textbf{insert}%
\texttt{
(iterator const \&
\textit{pos,}%
value\_type const \&
\textit{value}%
)
}

\begin{itemize}
\item
Inserts the given value before the element pointed to by the iterator pos and returns an iterator to the new element.
\end{itemize}
 
\noindent
\texttt{%
iterator 	
}
\textbf{erase}%
\texttt{
(iterator const \&
\textit{pos}%
)
}
 	
\begin{itemize}
\item
Removes the element pointed by the iterator pos and returns an iterator pointing to the next element.
\end{itemize}
 
\noindent
\texttt{%
void 
}
\textbf{clear}%
\texttt{
(void)
}

%%%%%
 
\noindent
\texttt{%
iterator
}
\textbf{begin}%
\texttt{%
(void)
}

\begin{itemize}
\item
Construct an iterator to the first element of the domain.
\end{itemize}
 
\noindent
\texttt{%
const\_iterator
}
\textbf{begin}%
\texttt{%
(void) const
}

\begin{itemize}
\item
Construct a const\_iterator to the first element of the domain.
\end{itemize}
 
\noindent
\texttt{%
const\_iterator
}
\textbf{cbegin}%
\texttt{%
(void) const
}

\begin{itemize}
\item
Construct a const\_iterator to the first element of the domain.
\end{itemize}
 
\noindent
\texttt{%
iterator
}
\textbf{end}%
\texttt{%
(void)
}

\begin{itemize}
\item
Construct an iterator to one past the last element of the domain.
\end{itemize}
 
\noindent
\texttt{%
const\_iterator
}
\textbf{end}%
\texttt{%
(void) const
}

\begin{itemize}
\item
Construct a const\_iterator to one past the last element of the domain.
\end{itemize}
 
\noindent
\texttt{%
const\_iterator
}
\textbf{cend}%
\texttt{%
(void) const
}

\begin{itemize}
\item
Construct a const\_iterator to one past the last element of the domain.
\end{itemize}
 
\noindent
\texttt{%
void
}
\textbf{set\_element}%
\texttt{%
(index\_type const \&
\textit{idx,}%
value\_type const \&
\textit{val}%
)
}

\begin{itemize}
\item
Sets the element specified by the index to the provided value.
\end{itemize}
 
\noindent
\texttt{%
value\_type
}
\textbf{get\_element}%
\texttt{%
(index\_type const \&
\textit{idx}%
) const
}

\begin{itemize}
\item
Returns the value of the element specified by the index.
\end{itemize}
 
\noindent
\texttt{%
future< value\_type >
}
\textbf{get\_element\_split}%
\texttt{%
(index\_type const \&
\textit{idx}%
)
}

\begin{itemize}
\item
Returns a stapl::future holding the value of the element specified by the index.
\end{itemize}
 
\noindent
\texttt{%
void
}
\textbf{apply\_set}%
\texttt{%
(gid\_type const \&
\textit{gid,}%
F const \&
\textit{f}%
)
}

\begin{itemize}
\item
Applies a function f to the element specified by the GID.
\end{itemize}
 
\noindent
\texttt{%
F::result\_type
}
\textbf{apply\_get}%
\texttt{%
(gid\_type const \&
\textit{gid,}%
F const \&
\textit{f}%
)
}

\begin{itemize}
\item
Applies a function f to the element specified by the GID, and returns the result.
\end{itemize}
 
\noindent
\texttt{%
F::result\_type
}
\textbf{apply\_get}%
\texttt{%
(gid\_type const \&
\textit{gid,}%
F const \&
\textit{f}%
) const
}

\begin{itemize}
\item
Applies a function f to the element specified by the GID, and returns the result.
\end{itemize}
 
\noindent
\texttt{%
F::result\_type
}
\textbf{data\_apply\_async}%
\texttt{%
(gid\_type const \&
\textit{gid,}%
F const \&
\textit{f}%
)
}

\begin{itemize}
\item
Applies a function f to the element specified by the GID, and returns the result.
\end{itemize}
 
\noindent
\texttt{%
F::result\_type
}
\textbf{data\_apply}%
\texttt{%
(gid\_type const \&
\textit{gid,}%
F const \&
\textit{f}%
) const
}

\begin{itemize}
\item
Applies a function f to the element specified by the GID, and returns the result.
\end{itemize}
 
\subsubsection{Memory and Domain Management}

\noindent
\texttt{%
int
}
\textbf{count}%
\texttt{%
(key\_type const \&
\textit{key}%
)
}

\begin{itemize}
\item
Returns 1 if the specified key exists, and otherwise 0.
\end{itemize}
 
\noindent
\texttt{%
iterator 	
}
\textbf{find}%
\texttt{%
(key\_type const \&
\textit{key}%
)
}

\begin{itemize}
\item
Returns the ref of the member if the specified key exists, and otherwise an iterator to the end of the map. 
\end{itemize}
 
\noindent
\texttt{%
domain\_type 	
}
\textbf{domain}%
\texttt{%
(void) const
}

\begin{itemize}
\item
Returns the domain of the container.
\end{itemize}
 
%%void 	update_distribution (std::vector< std::tuple< std::pair< gid_type, gid_type >, cid_type, location_type >> const &updates)

\begin{itemize}
\item
Update the metadata of the distribution with mapping information for elements that will be inserted into the container. More...
\end{itemize}
 
\noindent
\texttt{%
template<typename DistSpecView >
void
}
\textbf{redistribute}%
\texttt{%
(DistSpecView const \&
}
\textit{dist\_view,}
\texttt{
typename std::enable\_if< is\_distribution\_view< DistSpecView >::value \&\&is\_view\_based< partition\_type >::value \&\&is\_view\_based< mapper\_type >::value >::type *=0)
}

\vspace{0.4cm} \emph{fix Doxygen to omit enable\_if, not part of user interface}

\begin{itemize}
\item
Redistribute the data stored in the container to match the distribution specified by the distribution view provided. 
\end{itemize}
 
\noindent
\texttt{%
void
}
\textbf{migrate}%
\texttt{%
(gid\_type const \&
\textit{gid,}
location\_type 
\textit{destination}
)
}

\begin{itemize}
\item
Migrates the element specified by the gid to the destination location. 
\end{itemize}
 
\noindent
\texttt{%
size\_type
}
\textbf{size}%
\texttt{%
(void) const
}

\begin{itemize}
\item
Returns the number of elements in the container. 
\end{itemize}
 
\noindent
\texttt{%
bool
}
\textbf{empty}%
\texttt{%
(void) const
}

\begin{itemize}
\item
Returns if the container has no element.
\end{itemize}
 
\noindent
\texttt{%
distribution\_type \& 
}
\textbf{distribution}%
\texttt{%
 (void)
}
 
\noindent
\texttt{%
distribution\_type const \& 
}
\textbf{distribution}%
\texttt{%
 (void) const
}
 
\noindent
\texttt{%
distribution\_type *
}
\textbf{get\_distribution}%
\texttt{%
(void)
}
 
\begin{itemize}
\item
Returns the data distribution of the container.
\end{itemize}

\noindent
\texttt{%
locality\_info 
}
\textbf{locality}%
\texttt{%
 (gid\_type gid)
}

\begin{itemize}
\item
Return locality information about the element specified by the gid. 
\end{itemize}
 
\noindent
\texttt{%
bool
}
\textbf{is\_local}%
\texttt{%
(gid\_type const \&
\textit{gid}%
)
}

\begin{itemize}
\item
Returns true if the element specified by the GID is stored on this location, or false otherwise. 
\end{itemize}


\subsection{Usage Example} \label{sec-map-cont-use}

The following example shows how to use the \texttt{map container}:
%%\newline\vspace{0.4cm}\newline\rule{12cm}{0.5mm}
\input{cont/map.tex}
%%\noindent\rule{12cm}{0.5mm}\vspace{0.5cm}

\textbf{PLEASE NOTE: 
Initializing a container with a sequential loop, as in this example,
is for illustrative purposes only.
The normal way to do this is applying a parallel algorithm or 
basic parallel construct to a view over the container.  See Section
\ref{sec-map-vw-use}
for the recommended usage.
}

\vspace{0.4cm} \textit{WRITE -comments on example}

\subsection{Implementation} \label{sec-map-cont-impl}

\textit{WRITE}

\subsection{Performance} \label{sec-map-cont-perf}

\begin{itemize}
\item
Fig. \ref{fig:map-cont-constr-exper}
shows the performance of constructing a \stapl\ map container
whose elements are atomic values, \stl\ containers, or \stapl\ containers.
\item
Fig. \ref{fig:map-cont-assign-exper}
shows the performance of assigning values to a \stapl\ map container
whose elements are atomic values, \stl\ containers, or \stapl\ containers.
\item
Fig. \ref{fig:map-cont-access-exper}
shows the performance of accessing values from a \stapl\ map container
whose elements are atomic values, \stl\ containers, or \stapl\ containers.
\end{itemize}

\textit{WRITE - complexity analysis}

\begin{figure}[p]
%%\includegraphics[scale=0.50]{figs/map_cont_constr}
\caption{Construct map Execution Time}
\label{fig:map-cont-constr-exper}
\end{figure}

\begin{figure}[p]
%%\includegraphics[scale=0.50]{figs/map_cont_assign}
\caption{Assign Values to map Execution Time}
\label{fig:map-cont-assign-exper}
\end{figure}

\begin{figure}[p]
%%\includegraphics[scale=0.50]{figs/map_cont_access}
\caption{Access Values in map Execution Time}
\label{fig:map-cont-access-exper}
\end{figure}

% % % % % % % % % % % % % % % % % % % % % % % % % % % % % % % % % % % % % % % 

\section{ Unordered Map Container} \label{sec-unmap-cont}
\index{map!unordered!container}
\index{container!map!unordered}
\index{map!unordered!constructor}

\subsection{Definition}

\textit{EXPAND - Distributed unordered map container.}

\subsection{Relationship to \stl\ unordered map}

\textit{WRITE}

\subsection{Relationship to other \stapl\ containers}

\textit{WRITE}

\subsection{Implementation}

\textit{WRITE - details specific to this container}

\subsection{Interface} \label{sec-unmap-cont-inter}

\subsubsection{Constructors}

\noindent
\textbf{unordered\_map}%
\texttt{%
(hasher const \&
\textit{hash=hasher(),}%
key\_equal const \&
\textit{comp=key\_equal()}%
)
}

\begin{itemize}
\item
Constructs an unordered map container. 
\end{itemize}
 
\noindent
\textbf{unordered\_map}%
\texttt{%
(PS const \&
\textit{part,}%
hasher const \&
\textit{hash=hasher(),}%
key\_equal const \&
\textit{comp=key\_equal()}%
)
}

\begin{itemize}
\item
Constructs an unordered map container given a partition strategy. 
\end{itemize}
 
\noindent
\textbf{unordered\_map}%
\texttt{%
(PS const \&
\textit{partitioner,}%
M const \&
\textit{mapper,}%
hasher const \&
\textit{hash=hasher(),}%
key\_equal const \&
\textit{comp=key\_equal()}%
)
}

\begin{itemize}
\item
Constructs an unordered map container given a partition strategy and a mapper for the distribution. 
\end{itemize}
 
\noindent
\textbf{unordered\_map}%
\texttt{%
(unordered\_map const \&
\textit{other}%
)
}

\begin{itemize}
\item
Constructs an unordered map from another.
\end{itemize}

\subsubsection{Element Manipulation}

\noindent
\texttt{%
reference
}
\textbf{operator[]}%
\texttt{%
(index\_type 
\textit{idx}%
)
}

\begin{itemize}
\item
Construct a reference to the mapped value of a specific key.
\end{itemize}
 
\noindent%
\texttt{%
const\_reference
}
\textbf{operator[]}%
\texttt{%
(index\_type 
\textit{idx}%
) const
}

\begin{itemize}
\item
Construct a const\_reference to the mapped value of a specific key.
\end{itemize}
 
\noindent
\texttt{%
reference
}
\textbf{front}%
\texttt{%
(void)
}

\begin{itemize}
\item
Construct a reference to the first element of the domain.
\end{itemize}
 
\noindent
\texttt{%
const\_reference
}
\textbf{front}%
\texttt{%
(void) const
}

\begin{itemize}
\item
Construct a const\_reference to the first element of the domain.
\end{itemize}
 
\noindent
\texttt{%
reference
}
\textbf{back}%
\texttt{%
(void)
}

\begin{itemize}
\item
Construct a reference to the last element in the domain.
\end{itemize}
 
\noindent
\texttt{%
const\_reference
}
\textbf{back}%
\texttt{%
(void) const
}

\begin{itemize}
\item
Construct a const\_reference to the last element in the domain.
\end{itemize}
 
\noindent
\texttt{%
reference
}
\textbf{make\_reference}%
\texttt{%
(index\_type const \&
\textit{idx}%
)
}

\begin{itemize}
\item
Construct a reference to a specific pair.
\end{itemize}
 
\noindent
\texttt{%
const\_reference
}
\textbf{make\_reference}%
\texttt{%
(index\_type const \&
\textit{idx}%
) const
}

\begin{itemize}
\item
Construct a const\_reference to a specific pair.
\end{itemize}
 
\noindent
\texttt{%
iterator
}
\textbf{make\_iterator}%
\texttt{%
(gid\_type const \&
\textit{gid}%
)
}

\begin{itemize}
\item
Construct an iterator to a specified key.
\end{itemize}
 
\noindent
\texttt{%
const\_iterator 
}
\textbf{make\_const\_iterator}%
\texttt{%
(gid\_type const \&
gid
) const
}

\begin{itemize}
\item
Construct a const\_iterator to a specified key.
\end{itemize}
 
\noindent
\texttt{%
iterator
}
\textbf{begin}%
\texttt{%
(void)
}

\begin{itemize}
\item
Construct an iterator to the first element of the domain.
\end{itemize}
 
\noindent
\texttt{%
const\_iterator
}
\textbf{begin}%
\texttt{%
(void) const
}

\begin{itemize}
\item
Construct a const\_iterator to the first element of the domain.
\end{itemize}
 
\noindent
\texttt{%
const\_iterator
}
\textbf{cbegin}%
\texttt{%
(void) const
}

\begin{itemize}
\item
Construct a const\_iterator to the first element of the domain.
\end{itemize}
 
\noindent
\texttt{%
iterator
}
\textbf{end}%
\texttt{%
(void)
}

\begin{itemize}
\item
Construct an iterator to one past the last element of the domain.
\end{itemize}
 
\noindent
\texttt{%
const\_iterator
}
\textbf{end}%
\texttt{%
(void) const
}

\begin{itemize}
\item
Construct a const\_iterator to one past the last element of the domain.
\end{itemize}
 
\noindent
\texttt{%
const\_iterator
}
\textbf{cend}%
\texttt{%
(void) const
}

\begin{itemize}
\item
Construct a const\_iterator to one past the last element of the domain.
\end{itemize}
 
\noindent
\texttt{%
void
}
\textbf{set\_element}%
\texttt{%
(index\_type const \&
\textit{idx,}%
value\_type const \&
\textit{val}%
)
}

\begin{itemize}
\item
Sets the element specified by the index to the provided value.
\end{itemize}
 
\noindent
\texttt{%
value\_type
}
\textbf{get\_element}%
\texttt{%
(index\_type const \&
\textit{idx}%
) const
}

\begin{itemize}
\item
Returns the value of the element specified by the index.
\end{itemize}
 
\noindent
\texttt{%
future< value\_type >
}
\textbf{get\_element\_split}%
\texttt{%
(index\_type const \&
\textit{idx}%
)
}

\begin{itemize}
\item
Returns a stapl::future holding the value of the element specified by the index.
\end{itemize}
 
\noindent
\texttt{%
void
}
\textbf{apply\_set}%
\texttt{%
(gid\_type const \&
\textit{gid,}%
F const \&
\textit{f}%
)
}

\begin{itemize}
\item
Applies a function f to the element specified by the GID.
\end{itemize}
 
\noindent
\texttt{%
F::result\_type
}
\textbf{apply\_get}%
\texttt{%
(gid\_type const \&
\textit{gid,}%
F const \&
\textit{f}%
)
}

\begin{itemize}
\item
Applies a function f to the element specified by the GID, and returns the result.
\end{itemize}
 
\noindent
\texttt{%
F::result\_type
}
\textbf{apply\_get}%
\texttt{%
(gid\_type const \&
\textit{gid,}%
F const \&
\textit{f}%
) const
}

\begin{itemize}
\item
Applies a function f to the element specified by the GID, and returns the result.
\end{itemize}
 
 
\subsubsection{Memory and Domain Management}

\noindent
\texttt{%
hasher 	
}
\textbf{hash\_function}%
\texttt{%
() const
}

\begin{itemize}
\item
Return an instance of the container's hash function.
\end{itemize}
 
\noindent
\texttt{%
key\_equal
}
\textbf{key\_eq}%
\texttt{%
() const
}

\begin{itemize}
\item
Return an instance of the container's comparator function.
\end{itemize}
 
\noindent
\texttt{%
domain\_type 	
}
\textbf{domain}%
\texttt{%
(void) const
}

\begin{itemize}
\item
Returns the domain of the container.
\end{itemize}
 
\begin{itemize}
\item
Redistribute the data stored in the container to match the distribution specified by the distribution view provided. 
\end{itemize}
 
\noindent
\texttt{%
void
}
\textbf{migrate}%
\texttt{%
(gid\_type const \&
\textit{gid,}
location\_type 
\textit{destination}
)
}

\begin{itemize}
\item
Migrates the element specified by the gid to the destination location. 
\end{itemize}
 
\noindent
\texttt{%
size\_type
}
\textbf{size}%
\texttt{%
(void) const
}

\begin{itemize}
\item
Returns the number of elements in the container. 
\end{itemize}
 
\noindent
\texttt{%
bool
}
\textbf{empty}%
\texttt{%
(void) const
}

\begin{itemize}
\item
Returns if the container has no element.
\end{itemize}
 
\noindent
\texttt{%
distribution\_type \& 
}
\textbf{distribution}%
\texttt{%
 (void)
}
 
\noindent
\texttt{%
distribution\_type const \& 
}
\textbf{distribution}%
\texttt{%
 (void) const
}
 
\noindent
\texttt{%
distribution\_type *
}
\textbf{get\_distribution}%
\texttt{%
(void)
}
 
\begin{itemize}
\item
Returns the data distribution of the container.
\end{itemize}

\noindent
\texttt{%
locality\_info 
}
\textbf{locality}%
\texttt{%
 (gid\_type gid)
}

\begin{itemize}
\item
Return locality information about the element specified by the gid. 
\end{itemize}
 
\noindent
\texttt{%
bool
}
\textbf{is\_local}%
\texttt{%
(gid\_type const \&
\textit{gid}%
)
}

\begin{itemize}
\item
Returns true if the element specified by the GID is stored on this location, or false otherwise. 
\end{itemize}

\subsection{Usage Example} \label{sec-unmap-cont-use}

The following example shows how to use the \texttt{unordered map container}:
%%\newline\vspace{0.4cm}\newline\rule{12cm}{0.5mm}
\input{cont/unordmap.tex}
%%\noindent\rule{12cm}{0.5mm}\vspace{0.5cm}

\textbf{PLEASE NOTE: 
Initializing a container with a sequential loop, as in this example,
is for illustrative purposes only.
The normal way to do this is applying a parallel algorithm or 
basic parallel construct to a view over the container.  See Section
\ref{sec-unmap-vw-use}
for the recommended usage.
}

\vspace{0.4cm} \textit{WRITE -comments on example}

\subsection{Implementation} \label{sec-unmap-cont-impl}

\textit{WRITE}

\subsection{Performance} \label{sec-unmap-cont-perf}

\begin{itemize}
\item
Fig. \ref{fig:unmap-cont-constr-exper}
shows the performance of constructing a \stapl\ unordered map container
whose elements are atomic values, \stl\ containers, or \stapl\ containers.
\item
Fig. \ref{fig:unmap-cont-assign-exper}
shows the performance of assigning values to a \stapl\ unordered map container
whose elements are atomic values, \stl\ containers, or \stapl\ containers.
\item
Fig. \ref{fig:unmap-cont-access-exper}
shows the performance of accessing values from a \stapl\ unordered map container
whose elements are atomic values, \stl\ containers, or \stapl\ containers.
\end{itemize}

\textit{WRITE - complexity analysis}

\begin{figure}[p]
%%\includegraphics[scale=0.50]{figs/unordered map_cont_constr}
\caption{Construct unordered map Execution Time}
\label{fig:unmap-cont-constr-exper}
\end{figure}

\begin{figure}[p]
%%\includegraphics[scale=0.50]{figs/unordered map_cont_assign}
\caption{Assign Values to unordered map Execution Time}
\label{fig:unmap-cont-assign-exper}
\end{figure}

\begin{figure}[p]
%%\includegraphics[scale=0.50]{figs/unordered map_cont_access}
\caption{Access Values in unordered map Execution Time}
\label{fig:unmap-cont-access-exper}
\end{figure}

% % % % % % % % % % % % % % % % % % % % % % % % % % % % % % % % % % % % % % % 

\section{ Unordered Set Container} \label{sec-unset-cont}
\index{set!unordered!container}
\index{container!set!unordered}
\index{set!unordered!constructor}

\subsection{Definition}

\textit{EXPAND - Distributed unordered set container.}

\subsection{Relationship to \stl\ unordered map}

\textit{WRITE}

\subsection{Relationship to other \stapl\ containers}

\textit{WRITE}

\subsection{Implementation}

\textit{WRITE - details specific to this container}

\subsection{Interface} \label{sec-unset-cont-inter}

\subsubsection{Constructors}

%%unordered_set 
%%(hasher const &hash=hasher(), key_equal const &comp=key_equal())
 
%%unordered_set 
%%(partition_type const &part, hasher const &hash=hasher(), key_equal const &comp=key_equal())
 
%%unordered_set 
%%(partition_type const &partitioner, mapper_type const &mapper)
 
%%unordered_set 
%%(unordered_set const &other)
 
\subsubsection{Element Manipulation}

\noindent
\texttt{%
reference
}
\textbf{make\_reference}%
\texttt{%
(index\_type const \&
\textit{idx}%
)
}

\begin{itemize}
\item
Construct a reference to a specified key.
\end{itemize}
 
\noindent
\texttt{%
const\_reference
}
\textbf{make\_reference}%
\texttt{%
(index\_type const \&
\textit{idx}%
) const
}

\begin{itemize}
\item
Construct a const\_reference to a specified key.
\end{itemize}
 
\noindent
\texttt{%
iterator
}
\textbf{make\_iterator}%
\texttt{%
(gid\_type const \&
\textit{gid}%
)
}

\begin{itemize}
\item
Construct an iterator to a specified key.
\end{itemize}
 
\noindent
\texttt{%
const\_iterator 
}
\textbf{make\_const\_iterator}%
\texttt{%
(gid\_type const \&
gid
) const
}

\begin{itemize}
\item
Construct a const\_iterator to a specified key.
\end{itemize}
 
\noindent
\texttt{%
iterator
}
\textbf{begin}%
\texttt{%
(void)
}

\begin{itemize}
\item
Construct an iterator to the first element of the domain.
\end{itemize}
 
\noindent
\texttt{%
const\_iterator
}
\textbf{begin}%
\texttt{%
(void) const
}

\begin{itemize}
\item
Construct a const\_iterator to the first element of the domain.
\end{itemize}
 
\noindent
\texttt{%
const\_iterator
}
\textbf{cbegin}%
\texttt{%
(void) const
}

\begin{itemize}
\item
Construct a const\_iterator to the first element of the domain.
\end{itemize}
 
\noindent
\texttt{%
iterator
}
\textbf{end}%
\texttt{%
(void)
}

\begin{itemize}
\item
Construct an iterator to one past the last element of the domain.
\end{itemize}
 
\noindent
\texttt{%
const\_iterator
}
\textbf{end}%
\texttt{%
(void) const
}

\begin{itemize}
\item
Construct a const\_iterator to one past the last element of the domain.
\end{itemize}
 
\noindent
\texttt{%
const\_iterator
}
\textbf{cend}%
\texttt{%
(void) const
}

\begin{itemize}
\item
Construct a const\_iterator to one past the last element of the domain.
\end{itemize}
 
\noindent
\texttt{%
void
}
\textbf{set\_element}%
\texttt{%
(index\_type const \&
\textit{idx,}%
value\_type const \&
\textit{val}%
)
}

\begin{itemize}
\item
Sets the element specified by the index to the provided value.
\end{itemize}
 
\noindent
\texttt{%
value\_type
}
\textbf{get\_element}%
\texttt{%
(index\_type const \&
\textit{idx}%
) const
}

\begin{itemize}
\item
Returns the value of the element specified by the index.
\end{itemize}
 
\noindent
\texttt{%
future< value\_type >
}
\textbf{get\_element\_split}%
\texttt{%
(index\_type const \&
\textit{idx}%
)
}

\begin{itemize}
\item
Returns a stapl::future holding the value of the element specified by the index.
\end{itemize}
 
\noindent
\texttt{%
void
}
\textbf{apply\_set}%
\texttt{%
(gid\_type const \&
\textit{gid,}%
F const \&
\textit{f}%
)
}

\begin{itemize}
\item
Applies a function f to the element specified by the GID.
\end{itemize}
 
\noindent
\texttt{%
F::result\_type
}
\textbf{apply\_get}%
\texttt{%
(gid\_type const \&
\textit{gid,}%
F const \&
\textit{f}%
)
}

\begin{itemize}
\item
Applies a function f to the element specified by the GID, and returns the result.
\end{itemize}
 
\noindent
\texttt{%
F::result\_type
}
\textbf{apply\_get}%
\texttt{%
(gid\_type const \&
\textit{gid,}%
F const \&
\textit{f}%
) const
}

\begin{itemize}
\item
Applies a function f to the element specified by the GID, and returns the result.
\end{itemize}
 
 
\subsubsection{Memory and Domain Management}

%%hasher 	
%%hash_function 
%%(void) const

%%\begin{itemize}
%%\item
%%Return an instance of the container's hash function.
%%\end{itemize}
 
%%key_equal
%%key_eq 
%%(void) const

%%\begin{itemize}
%%\item
%%Return an instance of the container's comparator function.
%%\end{itemize}
 
\noindent
\texttt{%
domain\_type 	
}
\textbf{domain}%
\texttt{%
(void) const
}

\begin{itemize}
\item
Returns the domain of the container.
\end{itemize}
 
\noindent
\texttt{%
void
}
\textbf{migrate}%
\texttt{%
(gid\_type const \&
\textit{gid,}
location\_type 
\textit{destination}
)
}

\begin{itemize}
\item
Migrates the element specified by the gid to the destination location. 
\end{itemize}
 
\noindent
\texttt{%
size\_type
}
\textbf{size}%
\texttt{%
(void) const
}

\begin{itemize}
\item
Returns the number of elements in the container. 
\end{itemize}
 
\noindent
\texttt{%
bool
}
\textbf{empty}%
\texttt{%
(void) const
}

\begin{itemize}
\item
Returns if the container has no element.
\end{itemize}
 
\noindent
\texttt{%
distribution\_type \& 
}
\textbf{distribution}%
\texttt{%
 (void)
}
 
\noindent
\texttt{%
distribution\_type const \& 
}
\textbf{distribution}%
\texttt{%
 (void) const
}
 
\noindent
\texttt{%
distribution\_type *
}
\textbf{get\_distribution}%
\texttt{%
(void)
}
 
\begin{itemize}
\item
Returns the data distribution of the container.
\end{itemize}

\noindent
\texttt{%
locality\_info 
}
\textbf{locality}%
\texttt{%
 (gid\_type gid)
}

\begin{itemize}
\item
Return locality information about the element specified by the gid. 
\end{itemize}
 
\noindent
\texttt{%
bool
}
\textbf{is\_local}%
\texttt{%
(gid\_type const \&
\textit{gid}%
)
}

\begin{itemize}
\item
Returns true if the element specified by the GID is stored on this location, or false otherwise. 
\end{itemize}


\subsection{Usage Example} \label{sec-unset-cont-use}

The following example shows how to use the \texttt{unordered set container}:
%%\newline\vspace{0.4cm}\newline\rule{12cm}{0.5mm}
\input{cont/unordset.tex}
%%\noindent\rule{12cm}{0.5mm}\vspace{0.5cm}

\textbf{PLEASE NOTE: 
Initializing a container with a sequential loop, as in this example,
is for illustrative purposes only.
The normal way to do this is applying a parallel algorithm or 
basic parallel construct to a view over the container.  See Section
\ref{sec-unset-vw-use}
for the recommended usage.
}

\vspace{0.4cm}
\textit{WRITE -comments on example}

\subsection{Implementation} \label{sec-unset-cont-impl}

\textit{WRITE}

\subsection{Performance} \label{sec-unset-cont-perf}

\begin{itemize}
\item
Fig. \ref{fig:unset-cont-constr-exper}
shows the performance of constructing a \stapl\ unordered set container
whose elements are atomic values, \stl\ containers, or \stapl\ containers.
\item
Fig. \ref{fig:unset-cont-assign-exper}
shows the performance of assigning values to a \stapl\ unordered set container
whose elements are atomic values, \stl\ containers, or \stapl\ containers.
\item
Fig. \ref{fig:unset-cont-access-exper}
shows the performance of accessing values from a \stapl\ unordered set container
whose elements are atomic values, \stl\ containers, or \stapl\ containers.
\end{itemize}

\textit{WRITE - complexity analysis}

\begin{figure}[p]
%%\includegraphics[scale=0.50]{figs/unordered set_cont_constr}
\caption{Construct unordered set Execution Time}
\label{fig:unset-cont-constr-exper}
\end{figure}

\begin{figure}[p]
%%\includegraphics[scale=0.50]{figs/unordered set_cont_assign}
\caption{Assign Values to unordered set Execution Time}
\label{fig:unset-cont-assign-exper}
\end{figure}

\begin{figure}[p]
%%\includegraphics[scale=0.50]{figs/unordered set_cont_access}
\caption{Access Values in unordered set Execution Time}
\label{fig:unset-cont-access-exper}
\end{figure}

% % % % % % % % % % % % % % % % % % % % % % % % % % % % % % % % % % % % % % % 


% % % % % % % % % % % % % % % % % % % % % % % % % % % % % % % % % % % % %

\chapter{Parallel Views}

Parallel views in STAPL are grouped into the following categories:
\vspace{0.4cm}

Model common abstract data types:
\newline
Array (Section \ref{sec-ary-vw}),
Vector (Section \ref{sec-vec-vw}),
List (Section \ref{sec-list-vw}),
Map (Section \ref{sec-map-vw}),
MultiArray (Section \ref{sec-multi-vw}),
Matrix (Section \ref{sec-matrix-vw}),
ArrayRO (Section \ref{sec-aryro-vw}),
Set (Section \ref{sec-set-vw})
\vspace{0.4cm}

Model graph abstract data types:
\newline
Graph (Section \ref{sec-graf-vw}),
\vspace{0.4cm}

Present data without using concrete storage:
\newline
Overlap (Section \ref{sec-over-vw}),
Repeated (Section \ref{sec-rep-vw}),
Counting (Section \ref{sec-cnt-vw}),
Functor (Section \ref{sec-func-vw})
\vspace{0.4cm}

Modify values retrieved from underlying view or container:
\newline
Strided (Section \ref{sec-str-vw}),
Reverse (Section \ref{sec-rev-vw}),
Filter (Section \ref{sec-filt-vw}),
Transform (Section \ref{sec-trans-vw})
\vspace{0.4cm}

Access values from multiarrays in patterns used in numerical computation:
\newline
Banded (Section \ref{sec-band-vw}),
Extended (Section \ref{sec-extend-vw}),
Slices (Section \ref{sec-slices-vw})
Stencil (Section \ref{sec-stencil-vw})
\vspace{0.4cm}

Group elements of underlying container:
\newline
Native (Section \ref{sec-nat-vw}),
Segmented (Section \ref{sec-seg-vw})
\vspace{0.4cm}

Compose views with other views:
\newline
Zip (Section \ref{sec-zip-vw})
Cross (Section \ref{sec-cross-vw})
\vspace{0.4cm}

Enable specification of an arbitrary data distribution:
\newline
System (Section \ref{sec-dist-sys-vw}),
Mapping (Section \ref{sec-dist-map-vw}),
Partitioning (Section \ref{sec-dist-part-vw}),
Distribution specifications (Section \ref{sec-dist-spec-vw})

% % % % % % % % % % % % % % % % % % % % % % % % % % % % % % % % % % % % % % 

\section{Introduction} \label{sec-intro-vw}

\textit{WRITE}

\subsection{Implementation}

\textit{WRITE - implementation common to all views}

% % % % % % % % % % % % % % % % % % % % % % % % % % % % % % % % % % % % % % 

\section{Array View} \label{sec-ary-vw}
\index{array!view}
\index{view!array}

\subsection{Definition}

\textit{EXPAND - A view that provides the interface of an array abstract data type.}

\subsection{Relationship to \stapl\ array container}

\textit{WRITE}

\subsection{Relationship to other \stapl\ views}

\textit{WRITE}

\subsection{Implementation}

\textit{WRITE - details specific to this view}

\subsection{Interface} \label{sec-ary-vw-inter}

\subsubsection{Constructors}

\noindent
\textbf{array\_view}%
\texttt{%
(view\_container\_type *
\textit{vcont,}%
domain\_type const \&
\textit{dom,}%
map\_func\_type
\textit{mfunc=MapFunc()}%
)
}

\begin{itemize}
\item
Constructor used to pass ownership of the container to the view.
\end{itemize}

\noindent
\textbf{array\_view}%
\texttt{%
(view\_container\_type *
\textit{vcont,}%
domain\_type const \&
\textit{dom,}%
map\_func\_type
\textit{mfunc,}%
array\_view const \&
)
}

\begin{itemize}
\item
Constructor used to pass ownership of the container to the view.
\end{itemize}

\noindent
\textbf{array\_view}%
\texttt{%
(view\_container\_type const \&
\textit{vcont,}%
domain\_type const \&
\textit{dom,}%
map\_func\_type
\textit{mfunc=MapFunc()}%
)
}

\begin{itemize}
\item
Constructor that does not takes ownership over the passed container.
\end{itemize}

\noindent
\textbf{array\_view}%
\texttt{%
(view\_container\_type const \&
\textit{vcont,}%
domain\_type const \&
\textit{dom,}%
map\_func\_type
\textit{mfunc,}%
array\_view const \&
)
}

\begin{itemize}
\item
Constructor that does not takes ownership over the passed container.
\end{itemize}

\noindent
\textbf{array\_view}%
\texttt{%
(view\_container\_type *
\textit{vcont}%
)
}

\begin{itemize}
\item
Constructs a view that can reference all the elements of the passed container. The view takes ownership of the container.
\end{itemize}

\noindent
\textbf{array\_view}%
\texttt{%
(view\_container\_type \&
\textit{vcont}%
)
}

\begin{itemize}
\item
Constructs a view that can reference all the elements of the passed container.
\end{itemize}

\noindent
\texttt{%
template<typename Derived1 >
}
\newline
\textbf{array\_view}%
\texttt{%
(array\_view< C, Dom, MapFunc, Derived1 > const \&
\textit{other}%
)
}

\begin{itemize}
\item
Copy constructor when the passed view is not the most derived view.
\end{itemize}

\noindent
\texttt{%
void
}
\newline
\textbf{resize}%
\texttt{%
(size\_t
\textit{size}%
)
}

\begin{itemize}
\item
Update the underlying container to hold the specified number of elements.
\item
All previous information is lost.
\end{itemize}

\subsubsection{View Atrributes}

\noindent
\texttt{%
size\_type
}
\newline
\textbf{size}%
\texttt{%
(void) const
}

\begin{itemize}
\item
Returns the number of elements referenced for the view.
\end{itemize}

\noindent
\texttt{%
bool
}
\newline
\textbf{empty}%
\texttt{%
(void) const
}

\begin{itemize}
\item
Returns if the view does not reference any element.
\end{itemize}

\subsubsection{Element Manipulation}

\noindent
\texttt{%
value\_type
}
\newline
\textbf{get\_element}%
\texttt{%
(index\_t const \&
\textit{index}%
) const
}

\begin{itemize}
\item
Get the element index from the container.
\end{itemize}

\noindent
\texttt{%
template<class Functor >
}
\texttt{%
Functor::result\_type
}
\newline
\textbf{apply\_get}%
\texttt{%
(index\_t const \&
\textit{index,}%
Functor
\textit{f}%
)
}

\begin{itemize}
\item
Applies the provided function to the value referenced for the given index and returns the result of the operation.
\end{itemize}

\noindent
\texttt{%
void
}
\newline
\textbf{set\_element}%
\texttt{%
(index\_t const \&
\textit{index,}%
value\_t const \&
\textit{value}%
)
}

\begin{itemize}
\item
Set the element index in the container to value value.
\end{itemize}

\noindent
\texttt{%
template<class Functor >
}
\texttt{%
void
}
\newline
\textbf{apply\_set}%
\texttt{%
(index\_t const \&
\textit{index,}%
Functor
\textit{f}%
)
}

\begin{itemize}
\item
Applies the provided function to the value referenced for the given index and mutates the element with the resulting value.
\end{itemize}

\noindent
\texttt{%
reference
}
\newline
\textbf{operator[]}%
\texttt{%
(index\_t
\textit{index}%
) const
}

\begin{itemize}
\item
The bracket operator is the basic access method.
\end{itemize}

\noindent
\texttt{%
reference
}
\newline
\textbf{make\_reference}%
\texttt{%
(index\_t
\textit{index}%
) const
}

\subsection{Usage Example} \label{sec-ary-vw-use}

The following example shows how to use the \texttt{array view}:
%%\newline\vspace{0.4cm}\newline\rule{12cm}{0.5mm}
\input{view/array.tex}
%%\noindent\rule{12cm}{0.5mm}\vspace{0.5cm}

\subsection{Implementation} \label{sec-ary-vw-impl}

\textit{WRITE}

\subsection{Performance} \label{sec-ary-vw-perf}

\begin{itemize}
\item
Fig. \ref{fig:ary-vw-assign-exper}
shows the performance of assigning values through a \stapl\ array view
whose elements are atomic values, \stl\ views, or \stapl\ views.
\item
Fig. \ref{fig:ary-vw-access-exper}
shows the performance of accessing values through a \stapl\ array view
whose elements are atomic values, \stl\ views, or \stapl\ views.
\end{itemize}

\begin{figure}[p]
%%\includegraphics[scale=0.50]{figs/array_vw_assign}
\caption{Assign Values through Array View Execution Time}
\label{fig:ary-vw-assign-exper}
\end{figure}

\begin{figure}[p]
%%\includegraphics[scale=0.50]{figs/array_vw_access}
\caption{Access Values through Array View Execution Time}
\label{fig:ary-vw-access-exper}
\end{figure}

\emph{WRITE - reference complexity of underlying container}

% % % % % % % % % % % % % % % % % % % % % % % % % % % % % % % % % % % % % % 

\section{Vector View} \label{sec-vec-vw}
\index{vector!view}
\index{view!vector}

\subsection{Definition}

\textit{EXPAND - A view that provides the interface of a vector abstract data type.}

\subsection{Relationship to \stapl\ vector container}

\textit{WRITE}

\subsection{Relationship to other \stapl\ views}

\textit{WRITE}

\subsection{Implementation}

\textit{WRITE - details specific to this view}

\subsection{Interface} \label{sec-vec-vw-inter}

\subsubsection{Constructors}

\noindent
\textbf{vector\_view}%
\texttt{%
(view\_container\_type
\textit{*vcont,}%
domain\_type const \&
\textit{dom,}%
map\_func\_type
\textit{mfunc=MapFunc()}%
)
}

\begin{itemize}
\item
Constructor used to pass ownership of the container to the view.
\end{itemize}

\noindent
\textbf{vector\_view}%
\texttt{%
(view\_container\_type const \&
\textit{vcont,}%
domain\_type const \&
\textit{dom,}%
map\_func\_type
\textit{mfunc=MapFunc(),}%
vector\_view const \&
\textit{other=vector\_view()}%
)
}

\begin{itemize}
\item
Constructor that does not takes ownership over the passed container.
\end{itemize}

\noindent
\textbf{vector\_view}%
\texttt{%
(view\_container\_type *
\textit{vcont}%
)
}

\begin{itemize}
\item
Constructs a view that can reference all the elements of the passed container. The view takes ownership of the container.
\end{itemize}

\noindent
\textbf{vector\_view}%
\texttt{%
(view\_container\_type \&
\textit{vcont}%
)
}

\begin{itemize}
\item
Constructs a view that can reference all the elements of the passed container.
\end{itemize}

\subsubsection{ Element Manipulation}

\noindent
\texttt{%
void
}
\newline
\textbf{add}%
\texttt{%
(value\_type const \&
\textit{val}%
)
}

\begin{itemize}
\item
Insert a new element in the container locally.
\end{itemize}

\noindent
\texttt{%
void
}
\newline
\textbf{insert}%
\texttt{%
(index\_type const \&
\textit{index,}%
value\_type const \&
\textit{value}%
)
}

\begin{itemize}
\item
Inserts the given value at the position index.
\end{itemize}

\noindent
\texttt{%
void
}
\newline
\textbf{push\_back}%
\texttt{%
(value\_type const \&
\textit{value}%
)
}

\begin{itemize}
\item
Inserts the given value an the end of the underlying container.
\end{itemize}

\noindent
\texttt{%
void
}
\newline
\textbf{pop\_back}%
\texttt{%
(void)
}

\begin{itemize}
\item
Removes the last element in the underlying container.
\end{itemize}

\noindent
\texttt{%
void
}
\newline
\textbf{erase}%
\texttt{%
(index\_type const \&
\textit{index}%
)
}

\begin{itemize}
\item
Removes the element at the position index.
\end{itemize}

\noindent
\texttt{%
void
}
\newline
\textbf{resize}%
\texttt{%
(size\_type
\textit{n}%
)
}

\begin{itemize}
\item
Update the underlying container to hold n elements.
\end{itemize}

\noindent
\texttt{%
void
}
\newline
\textbf{flush}%
\texttt{%
(void)
}

\begin{itemize}
\item
Flushes pending update information of the underlying container.
\end{itemize}

\noindent
\texttt{%
iterator
}
\newline
\textbf{find}%
\texttt{%
(index\_type const \&
\textit{index}%
)
}

\begin{itemize}
\item
Returns and iterator pointing to the element at position index.
\end{itemize}

\noindent
\texttt{%
reference
}
\newline
\textbf{front}%
\texttt{%
(void)
}

\begin{itemize}
\item
Returns a reference to the element at the beginning of the underlying container.
\end{itemize}

\noindent
\texttt{%
reference
}
\newline
\textbf{back}%
\texttt{%
(void)
}

\begin{itemize}
\item
Returns a reference to the element at the end of the underlying container.
\end{itemize}

\subsubsection{View Atrributes}

\noindent
\texttt{%
size\_type
}
\newline
\textbf{size}%
\texttt{%
(void) const
}

\begin{itemize}
\item
Returns the number of elements referenced for the view.
\end{itemize}

\noindent
\texttt{%
bool
}
\newline
\textbf{empty}%
\texttt{%
(void) const
}

\begin{itemize}
\item
Returns if the view does not reference any element.
\end{itemize}

\noindent
\texttt{%
value\_type
}
\newline
\textbf{get\_element}%
\texttt{%
(index\_t const \&
\textit{index}%
) const
}

\begin{itemize}
\item
Get the element index from the container.
\end{itemize}

\noindent
\texttt{%
template<class Functor >
}
\texttt{%
Functor::result\_type
}
\newline
\textbf{apply\_get}%
\texttt{%
(index\_t const \&
\textit{index,}%
Functor
\textit{f}%
)
}

\begin{itemize}
\item
Applies the provided function to the value referenced for the given index and returns the result of the operation.
\end{itemize}

\noindent
\texttt{%
void
}
\newline
\textbf{set\_element}%
\texttt{%
(index\_t const \&
\textit{index,}%
value\_t const \&
\textit{value}%
)
}

\begin{itemize}
\item
Set the element index in the container to value value.
\end{itemize}

\noindent
\texttt{%
template<class Functor >
}
\texttt{%
void
}
\newline
\textbf{apply\_set}%
\texttt{%
(index\_t const \&
\textit{index,}%
Functor
\textit{f}%
)
}

\begin{itemize}
\item
Applies the provided function to the value referenced for the given index and mutates the element with the resulting value.
\end{itemize}

\noindent
\texttt{%
reference
}
\newline
\textbf{operator[]}%
\texttt{%
(index\_t
\textit{index}%
) const
}

\begin{itemize}
\item
The bracket operator is the basic access method.
\end{itemize}

\noindent
\texttt{%
reference
}
\newline
\textbf{make\_reference}%
\texttt{%
(index\_t
\textit{index}%
) const
}

\subsection{Usage Example} \label{sec-vec-vw-use}

The following example shows how to use the \texttt{vector view}:
%%\newline\vspace{0.4cm}\newline\rule{12cm}{0.5mm}
\input{view/vector.tex}
%%\noindent\rule{12cm}{0.5mm}\vspace{0.5cm}

\subsection{Implementation} \label{sec-vec-vw-impl}

\textit{WRITE}

\subsection{Performance} \label{sec-vec-vw-perf}

\begin{itemize}
\item
Fig. \ref{fig:vec-vw-assign-exper}
shows the performance of assigning values through a \stapl\ vector view
whose elements are atomic values, \stl\ views, or \stapl\ views.
\item
Fig. \ref{fig:vec-vw-access-exper}
shows the performance of accessing values through a \stapl\ vector view
whose elements are atomic values, \stl\ views, or \stapl\ views.
\end{itemize}

\begin{figure}[p]
%%\includegraphics[scale=0.50]{figs/vector_vw_assign}
\caption{Assign Values through Vector View Execution Time}
\label{fig:vec-vw-assign-exper}
\end{figure}

\begin{figure}[p]
%%\includegraphics[scale=0.50]{figs/vector_vw_access}
\caption{Access Values through Vector View Execution Time}
\label{fig:vec-vw-access-exper}
\end{figure}

\emph{WRITE - reference complexity of underlying container}

% % % % % % % % % % % % % % % % % % % % % % % % % % % % % % % % % % % % % % 

\section{List View} \label{sec-list-vw}
\index{list!view}
\index{view!list}

\subsection{Definition}

\textit{EXPAND - A view that provides the interface of a list abstract data type.}

\subsection{Relationship to \stapl\ list container}

\textit{WRITE}

\subsection{Relationship to other \stapl\ views}

\textit{WRITE}

\subsection{Implementation}

\textit{WRITE - details specific to this view}

\subsection{Interface} \label{sec-list-vw-inter}

\subsubsection{Constructors}

\noindent
\textbf{list\_view}%
\texttt{%
(view\_container\_type *
\textit{vcont,}%
domain\_type const \&
\textit{dom,}%
    map\_func\_type
\textit{mfunc=MapFunc()}%
)
}

\begin{itemize}
\item
Constructor used to pass ownership of the container to the view.
\end{itemize}

\noindent
\textbf{list\_view}%
\texttt{%
(view\_container\_type const \&
\textit{vcont,}%
domain\_type const \&
\textit{dom,}%
map\_func\_type
\textit{mfunc=MapFunc(),}%
list\_view const \&
\textit{other=list\_view()}%
)
}

\begin{itemize}
\item
Constructor that does not takes ownership over the passed container.
\end{itemize}

\noindent
\textbf{list\_view}%
\texttt{%
(view\_container\_type *
\textit{vcont}%
)
}

\begin{itemize}
\item
Constructs a view that can reference all the elements of the passed container. The view takes ownership of the container.
\end{itemize}

\noindent
\textbf{list\_view}%
\texttt{%
(view\_container\_type \&
\textit{vcont}%
)
}

\begin{itemize}
\item
Constructs a view that can reference all the elements of the passed container.
\end{itemize}

\noindent
\texttt{%
template<typename Derived1 >
}
\newline
\textbf{list\_view}%
\texttt{%
(list\_view< C, Dom, MapFunc, Derived1 > const \&
\textit{other}%
)
}

\begin{itemize}
\item
Copy constructor when the passed view is not the most derived view.
\end{itemize}

\noindent
\textbf{list\_view}%
\texttt{%
(list\_view const \&
\textit{other}%
)
}

\begin{itemize}
\item
\textit{WRITE - fix Doxygen and copy material here}
\end{itemize}

\subsubsection{ Element Manipulation}

\noindent
\texttt{%
iterator
}
\newline
\textbf{insert}%
\texttt{%
(iterator const \&
\textit{pos,}%
value\_type const \&
\textit{value}%
)
}

\begin{itemize}
\item
Inserts the given value at the position index.
\end{itemize}

\noindent
\texttt{%
void
}
\newline
\textbf{push\_back}%
\texttt{%
(value\_type const \&
\textit{value}%
)
}

\begin{itemize}
\item
Inserts the given value an the end of the underlying container.
\end{itemize}

\noindent
\texttt{%
void
}
\newline
\textbf{pop\_back}%
\texttt{%
(void)
}

\begin{itemize}
\item
Removes the last element in the underlying container.
\end{itemize}

\noindent
\texttt{%
void
}
\newline
\textbf{erase}%
\texttt{%
(index\_type const \&
\textit{index}%
)
}

\begin{itemize}
\item
Removes the element at the position index.
\end{itemize}

\noindent
\texttt{%
iterator
}
\newline
\textbf{find}%
\texttt{%
(index\_type const \&
\textit{index}%
)
}

\begin{itemize}
\item
Returns and iterator pointing to the element at position index.
\end{itemize}

\noindent
\texttt{%
reference
}
\newline
\textbf{front}%
\texttt{%
(void)
}

\begin{itemize}
\item
Returns a reference to the element at the beginning of the underlying container.
\end{itemize}

\noindent
\texttt{%
reference
}
\newline
\textbf{back}%
\texttt{%
(void)
}

\begin{itemize}
\item
Returns a reference to the element at the end of the underlying container.
\end{itemize}

\noindent
\texttt{%
void
}
\newline
\textbf{add}%
\texttt{%
(value\_type const \&
\textit{value}%
)
}

\begin{itemize}
\item
Insert the given value into the underlying container.
\end{itemize}

\subsubsection{ View Attributes}

\noindent
\texttt{%
size\_type
}
\newline
\textbf{size}%
\texttt{%
(void) const
}

\begin{itemize}
\item
Returns the number of elements referenced for the view.
\end{itemize}

\noindent
\texttt{%
bool
}
\newline
\textbf{empty}%
\texttt{%
(void) const
}

\begin{itemize}
\item
Returns if the view does not reference any element.
\end{itemize}

\subsection{Usage Example} \label{sec-list-vw-use}

The following example shows how to use the \texttt{list view}:
%%\newline\vspace{0.4cm}\newline\rule{12cm}{0.5mm}
\input{view/list.tex}
%%\noindent\rule{12cm}{0.5mm}\vspace{0.5cm}

\subsection{Implementation} \label{sec-list-vw-impl}

\textit{WRITE}

\subsection{Performance} \label{sec-list-vw-perf}

\begin{itemize}
\item
Fig. \ref{fig:list-vw-assign-exper}
shows the performance of assigning values through a \stapl\ list view
whose elements are atomic values, \stl\ views, or \stapl\ views.
\item
Fig. \ref{fig:list-vw-access-exper}
shows the performance of accessing values through a \stapl\ list view
whose elements are atomic values, \stl\ views, or \stapl\ views.
\end{itemize}

\begin{figure}[p]
%%\includegraphics[scale=0.50]{figs/list_vw_assign}
\caption{Assign Values through List View Execution Time}
\label{fig:list-vw-assign-exper}
\end{figure}

\begin{figure}[p]
%%\includegraphics[scale=0.50]{figs/list_vw_access}
\caption{Access Values through List View Execution Time}
\label{fig:list-vw-access-exper}
\end{figure}

\emph{WRITE - reference complexity of underlying container}

% % % % % % % % % % % % % % % % % % % % % % % % % % % % % % % % % % % % % % 

\section{Map View} \label{sec-map-vw}
\index{map!view}
\index{view!map}

\subsection{Definition}

\textit{EXPAND - A view that maps from keys to values.}

\subsection{Relationship to \stapl\ map container}

\textit{WRITE}

\subsection{Relationship to other \stapl\ views}

\textit{WRITE}

\subsection{Implementation}

\textit{WRITE - details specific to this view}

\subsection{Interface} \label{sec-map-vw-inter}

\subsubsection{Constructors}

\noindent
\textbf{map\_view}%
\texttt{%
(view\_container\_type *
\textit{vcont,}%
domain\_type const \&
\textit{dom,}%
    map\_func\_type
\textit{mfunc=MapFunc()}%
)
}

\begin{itemize}
\item
Constructor used to pass ownership of the container to the view.
\end{itemize}

\noindent
\textbf{map\_view}%
\texttt{%
(view\_container\_type *
\textit{vcont,}%
domain\_type const \&
\textit{dom,}%
map\_func\_type
\textit{mfunc,}%
map\_view const \&
\textit{other}%
)
}

\begin{itemize}
\item
\textit{WRITE - fix Doxygen and copy material here}
\end{itemize}

\noindent
\textbf{map\_view}%
\texttt{%
(view\_container\_type const \&
\textit{vcont,}%
domain\_type const \&
\textit{dom,}%
map\_func\_type
\textit{mfunc=MapFunc(),}%
map\_view const \&
\textit{other=map\_view()}%
)
}

\begin{itemize}
\item
Constructor that does not takes ownership over the passed container.
\end{itemize}

\noindent
\textbf{map\_view}%
\texttt{%
(view\_container\_type *
\textit{vcont}%
)
}

\begin{itemize}
\item
Constructs a view that can reference all the elements of the passed container. The view takes ownership of the container.
\end{itemize}

\noindent
\textbf{map\_view}%
\texttt{%
(view\_container\_type \&
\textit{vcont}%
)
}

\begin{itemize}
\item
Constructs a view that can reference all the elements of the passed container.
\end{itemize}

\noindent
\texttt{%
template<typename Derived1 >
}
\newline
\textbf{map\_view}%
\texttt{%
(map\_view< C, Dom, MapFunc, Derived1 > const \&
\textit{other}%
)
}

\begin{itemize}
\item
\textit{WRITE - fix Doxygen and copy material here}
\end{itemize}

\subsubsection{ Element Manipulation}

\noindent
\texttt{%
bool
}
\newline
\textbf{insert}%
\texttt{%
(value\_type const \&
\textit{value}%
)
}

\begin{itemize}
\item
Inserts the given value if the value does not exist in the container.
\end{itemize}

\noindent
\texttt{%
void
}
\newline
\textbf{insert}%
\texttt{%
(key\_type const \&
\textit{key,}%
mapped\_type const \&
\textit{value}%
)
}

\begin{itemize}
\item
Inserts the given value if the key does not exist in the container.
\end{itemize}

\noindent
\texttt{%
template<typename Functor >
void
}
\newline
\textbf{insert}%
\texttt{%
(value\_type const \&
\textit{val,}%
Functor const \&
\textit{func}%
)
}

\begin{itemize}
\item
Inserts the given <key,value> pair (val) if the key does not exist in the container, otherwise mutates the stored value by applying the given functor func.
\end{itemize}

\noindent
\texttt{%
template<typename Functor >
}
\texttt{%
void
}
\newline
\textbf{insert}%
\texttt{%
(key\_type const \&
\textit{key,}%
mapped\_type const \&
\textit{value,}%
Functor const \&
\textit{on\_failure}%
)
}

\begin{itemize}
\item
Inserts the given value associated with the given key if the key does not exist in the container, otherwise mutates the stored value by applying the given functor func.
\end{itemize}

\noindent
\texttt{%
mapped\_type
}
\newline
\textbf{get}%
\texttt{%
(key\_type const \&
\textit{key}%
)
}

\begin{itemize}
\item
Fetch the value associated with the specified key.
\end{itemize}

\noindent
\texttt{%
size\_t
}
\newline
\textbf{erase}%
\texttt{%
(key\_type const \&
\textit{key}%
)
}

\begin{itemize}
\item
Removes the value associated with the specified key.
\end{itemize}

\noindent
\texttt{%
iterator
}
\newline
\textbf{find}%
\texttt{%
(key\_type const \&
\textit{key}%
)
}

\begin{itemize}
\item
Returns an iterator pointing to the element associated with the given key.
\end{itemize}

\noindent
\texttt{%
int
}
\newline
\textbf{count}%
\texttt{%
(key\_type const \&
\textit{key}%
)
}

\begin{itemize}
\item
Returns 1 if the specified key exists, and otherwise 0.
\end{itemize}

\noindent
\texttt{%
void
}
\newline
\textbf{clear}%
\texttt{%
(void)
}

\begin{itemize}
\item
Removes all the elements stored in the container.
\end{itemize}

\noindent
\texttt{%
value\_type
}
\newline
\textbf{get\_element}%
\texttt{%
(index\_t const \&
\textit{index}%
) const
}

\begin{itemize}
\item
Get the element index from the container.
\end{itemize}

\noindent
\texttt{%
template<class Functor >
}
\texttt{%
Functor::result\_type
}
\newline
\textbf{apply\_get}%
\texttt{%
(index\_t const \&
\textit{index,}%
Functor
\textit{f}%
)
}

\begin{itemize}
\item
Applies the provided function to the value referenced for the given index and returns the result of the operation.
\end{itemize}

\noindent
\texttt{%
void
}
\newline
\textbf{set\_element}%
\texttt{%
(index\_t const \&
\textit{index,}%
value\_t const \&
\textit{value}%
)
}

\begin{itemize}
\item
Set the element index in the container to value value.
\end{itemize}

\noindent
\texttt{%
template<class Functor >
}
\texttt{%
void
}
\newline
\textbf{apply\_set}%
\texttt{%
(index\_t const \&
\textit{index,}%
Functor
\textit{f}%
)
}

\begin{itemize}
\item
Applies the provided function to the value referenced for the given index and mutates the element with the resulting value.
\end{itemize}

\noindent
\texttt{%
reference
}
\newline
\textbf{operator[]}%
\texttt{%
(index\_t
\textit{index}%
) const
}

\begin{itemize}
\item
The bracket operator is the basic access method.
\end{itemize}

\noindent
\texttt{%
reference
}
\newline
\textbf{make\_reference}%
\texttt{%
(index\_t
\textit{index}%
) const
}

\subsubsection{ View Attributes}

\noindent
\texttt{%
size\_type
}
\newline
\textbf{size}%
\texttt{%
(void) const
}

\begin{itemize}
\item
Returns the number of elements referenced for the view.
\end{itemize}

\noindent
\texttt{%
bool
}
\newline
\textbf{empty}%
\texttt{%
(void) const
}

\begin{itemize}
\item
Returns if the view does not reference any element.
\end{itemize}

\subsection{Usage Example} \label{sec-map-vw-use}

The following example shows how to use the \texttt{map view}:
%%\newline\vspace{0.4cm}\newline\rule{12cm}{0.5mm}
\input{view/map.tex}
%%\noindent\rule{12cm}{0.5mm}\vspace{0.5cm}

\subsection{Implementation} \label{sec-map-vw-impl}

\textit{WRITE}

\subsection{Performance} \label{sec-map-vw-perf}

\begin{itemize}
\item
Fig. \ref{fig:map-vw-assign-exper}
shows the performance of assigning values through a \stapl\ map view
whose elements are atomic values, \stl\ views, or \stapl\ views.
\item
Fig. \ref{fig:map-vw-access-exper}
shows the performance of accessing values through a \stapl\ map view
whose elements are atomic values, \stl\ views, or \stapl\ views.
\end{itemize}

\begin{figure}[p]
%%\includegraphics[scale=0.50]{figs/map_vw_assign}
\caption{Assign Values through Map View Execution Time}
\label{fig:map-vw-assign-exper}
\end{figure}

\begin{figure}[p]
%%\includegraphics[scale=0.50]{figs/map_vw_access}
\caption{Access Values through Map View Execution Time}
\label{fig:map-vw-access-exper}
\end{figure}

\emph{WRITE - reference complexity of underlying container}

% % % % % % % % % % % % % % % % % % % % % % % % % % % % % % % % % % % % % % 

\section{Multiarray View} \label{sec-multi-vw}
\index{multiarray!view}
\index{view!multiarray}

\subsection{Definition}

\textit{EXPAND - A view that provides that interface of a multi-dimensional array abstract data type.}

\subsection{Relationship to \stapl\ multiarray container}

\textit{WRITE}

\subsection{Relationship to other \stapl\ views}

\textit{WRITE}

\subsection{Implementation}

\textit{WRITE - details specific to this view}

\subsection{Interface} \label{sec-multi-vw-inter}

\subsubsection{Constructors}

\noindent
\textbf{multiarray\_view}%
\texttt{%
(view\_container\_type \&
\textit{vcont,}%
domain\_type const \&
\textit{dom,}%
map\_function
\textit{mfunc=map\_function()}%
)
}

\begin{itemize}
\item
Constructor that does not take ownership over the passed container.
\end{itemize}

\noindent
\textbf{multiarray\_view}%
\texttt{%
(view\_container\_type *
\textit{vcont}%
)
}

\begin{itemize}
\item
Constructor used to pass ownership of the container to the view.
\end{itemize}

\noindent
\textbf{multiarray\_view}%
\texttt{%
(view\_container\_type \&
\textit{vcont}%
)
}

\begin{itemize}
\item
Constructor that does not take ownership over the passed container.
\end{itemize}

\noindent
\textbf{multiarray\_view}%
\texttt{%
(view\_container\_type *
\textit{vcont,}%
domain\_type const \&
\textit{dom,}%
map\_function
\textit{mfunc=map\_function()}%
)
}

\subsubsection{ View Attributes}

\begin{itemize}
\item
Constructor used to pass ownership of the container to the view.
\end{itemize}

\noindent
\texttt{%
domain\_type::size\_type
}
\newline
\textbf{dimensions}%
\texttt{%
(void) const
}

\begin{itemize}
\item
The length of each of the dimensions of the multiarray.
\end{itemize}

\noindent
\texttt{%
void
}
\newline
\textbf{resize}%
\texttt{%
(typename domain\_type::size\_type
\textit{size}%
)
}

\begin{itemize}
\item
Update the underlying container to hold the specified number of elements.
\end{itemize}
All previous information is lost.

\noindent
\texttt{%
size\_type
}
\newline
\textbf{size}%
\texttt{%
(void) const
}

\begin{itemize}
\item
Returns the number of elements referenced for the view.
\end{itemize}

\noindent
\texttt{%
bool
}
\newline
\textbf{empty}%
\texttt{%
(void) const
}

\begin{itemize}
\item
Returns if the view does not reference any element.
\end{itemize}

\subsubsection{ Element Manipulation}

\noindent
\texttt{%
value\_type
}
\newline
\textbf{get\_element}%
\texttt{%
(index\_t const \&
\textit{index}%
) const
}

\begin{itemize}
\item
Get the element index from the container.
\end{itemize}

\noindent
\texttt{%
template<class Functor >
}
\texttt{%
Functor::result\_type
}
\newline
\textbf{apply\_get}%
\texttt{%
(index\_t const \&
\textit{index,}%
Functor
\textit{f}%
)
}

\begin{itemize}
\item
Applies the provided function to the value referenced for the given index and returns the result of the operation.
\end{itemize}

\noindent
\texttt{%
void
}
\newline
\textbf{set\_element}%
\texttt{%
(index\_t const \&
\textit{index,}%
value\_t const \&
\textit{value}%
)
}

\begin{itemize}
\item
Set the element index in the container to value value.
\end{itemize}

\noindent
\texttt{%
template<class Functor >
}
\texttt{%
void
\newline
\textbf{apply\_set}%
(index\_t const \&
\textit{index,}%
Functor
\textit{f}%
)
}

\begin{itemize}
\item
Applies the provided function to the value referenced for the given index and mutates the element with the resulting value.
\end{itemize}

\noindent
\texttt{%
reference
}
\newline
\textbf{operator[]}%
\texttt{%
(index\_t
\textit{index}%
) const
}

\begin{itemize}
\item
The bracket operator is the basic access method.
\end{itemize}

\noindent
\texttt{%
reference
}
\newline
\textbf{make\_reference}%
\texttt{%
(index\_t
\textit{index}%
) const
}

\subsection{Usage Example} \label{sec-multi-vw-use}

The following example shows how to use the \texttt{multiarray view}:
%%\newline\vspace{0.4cm}\newline\rule{12cm}{0.5mm}
\input{view/multi.tex}
%%\noindent\rule{12cm}{0.5mm}\vspace{0.5cm}

\subsection{Implementation} \label{sec-multi-vw-impl}

\textit{WRITE}

\subsection{Performance} \label{sec-multi-vw-perf}

\begin{itemize}
\item
Fig. \ref{fig:multi-vw-assign-exper}
shows the performance of assigning values through a \stapl\ multiarray view
whose elements are atomic values, \stl\ views, or \stapl\ views.
\item
Fig. \ref{fig:multi-vw-access-exper}
shows the performance of accessing values through a \stapl\ multiarray view
whose elements are atomic values, \stl\ views, or \stapl\ views.
\end{itemize}

\begin{figure}[p]
%%\includegraphics[scale=0.50]{figs/multiarray_vw_assign}
\caption{Assign Values through Multiarray View Execution Time}
\label{fig:multi-vw-assign-exper}
\end{figure}

\begin{figure}[p]
%%\includegraphics[scale=0.50]{figs/multiarray_vw_access}
\caption{Access Values through Multiarray View Execution Time}
\label{fig:multi-vw-access-exper}
\end{figure}

\emph{WRITE - reference complexity of underlying container}

% % % % % % % % % % % % % % % % % % % % % % % % % % % % % % % % % % % % % % 

\section{Matrix View} \label{sec-matrix-vw}
\index{matrix!view}
\index{view!matrix}

\subsection{Definition}

\textit{ EXPAND- A view that provides that interface of a two-dimensional array abstract data type.}

\subsection{Relationship to \stapl\ matrix container}

\textit{WRITE}

\subsection{Relationship to other \stapl\ views}

\textit{WRITE}

\subsection{Implementation}

\textit{WRITE - details specific to this view}

\subsection{Interface} \label{sec-mat-vw-inter}

\subsubsection{Constructors}

\noindent
\textbf{matrix\_view}%
\texttt{%
(view\_container\_type \&
\textit{vcont,}%
domain\_type const \&
\textit{dom,}%
map\_function
\textit{mfunc=map\_function()}%
)
}

\begin{itemize}
\item
Constructor that does not take ownership over the passed container.
\end{itemize}

\noindent
\textbf{matrix\_view}%
\texttt{%
(view\_container\_type *
\textit{vcont}%
)
}

\begin{itemize}
\item
Constructor used to pass ownership of the container to the view.
\end{itemize}

\noindent
\textbf{matrix\_view}%
\texttt{%
(view\_container\_type \&
\textit{vcont}%
)
}

\begin{itemize}
\item
Constructor that does not take ownership over the passed container.
\end{itemize}

\noindent
\textbf{matrix\_view}%
\texttt{%
(view\_container\_type *
\textit{vcont,}%
domain\_type const \&
\textit{dom,}%
map\_function
\textit{mfunc=map\_function()}%
)
}

\begin{itemize}
\item
Constructor used to pass ownership of the container to the view.
\end{itemize}

\subsubsection{ View Attributes}

\noindent
\texttt{%
domain\_type::size\_type
}
\newline
\textbf{dimensions}%
\texttt{%
(void) const
}

\begin{itemize}
\item
The length of each of the dimensions of the matrix.
\end{itemize}

\noindent
\texttt{%
void
}
\newline
\textbf{resize}%
\texttt{%
(typename domain\_type::size\_type
\textit{size}%
)
}

\begin{itemize}
\item
Update the underlying container to hold the specified number of elements.
\end{itemize}
All previous information is lost.

\noindent
\texttt{%
size\_type
}
\newline
\textbf{size}%
\texttt{%
(void) const
}

\begin{itemize}
\item
Returns the number of elements referenced for the view.
\end{itemize}

\noindent
\texttt{%
bool
}
\newline
\textbf{empty}%
\texttt{%
(void) const
}

\begin{itemize}
\item
Returns if the view does not reference any element.
\end{itemize}

\subsubsection{Element Manipulation}

\noindent
\texttt{%
value\_type
}
\newline
\textbf{get\_element}%
\texttt{%
(index\_t const \&
\textit{index}%
) const
}

\begin{itemize}
\item
Get the element index from the container.
\end{itemize}

\noindent
\texttt{%
template<class Functor >
}
\texttt{%
Functor::result\_type
}
\newline
\textbf{apply\_get}%
\texttt{%
(index\_t const \&
\textit{index,}%
Functor
\textit{f}%
)
}

\begin{itemize}
\item
Applies the provided function to the value referenced for the given index and returns the result of the operation.
\end{itemize}

\noindent
\texttt{%
void
}
\newline
\textbf{set\_element}%
\texttt{%
(index\_t const \&
\textit{index,}%
value\_t const \&
\textit{value}%
)
}

\begin{itemize}
\item
Set the element index in the container to value value.
\end{itemize}

\noindent
\texttt{%
template<class Functor >
}
\texttt{%
void
\newline
\textbf{apply\_set}%
(index\_t const \&
\textit{index,}%
Functor
\textit{f}%
)
}

\begin{itemize}
\item
Applies the provided function to the value referenced for the given index and mutates the element with the resulting value.
\end{itemize}

\noindent
\texttt{%
reference
}
\newline
\textbf{operator[]}%
\texttt{%
(index\_t
\textit{index}%
) const
}

\begin{itemize}
\item
The bracket operator is the basic access method.
\end{itemize}

\noindent
\texttt{%
reference
}
\newline
\textbf{make\_reference}%
\texttt{%
(index\_t
\textit{index}%
) const
}

\subsection{Usage Example} \label{sec-mat-vw-use}

The following example shows how to use the \texttt{matrix view}:
%%\newline\vspace{0.4cm}\newline\rule{12cm}{0.5mm}
\input{view/matrix.tex}
%%\noindent\rule{12cm}{0.5mm}\vspace{0.5cm}

\subsection{Implementation} \label{sec-mat-vw-impl}

\textit{WRITE}

\subsection{Performance} \label{sec-mat-vw-perf}

\begin{itemize}
\item
Fig. \ref{fig:mat-vw-assign-exper}
shows the performance of assigning values through a \stapl\ matrix view
whose elements are atomic values, \stl\ views, or \stapl\ views.
\item
Fig. \ref{fig:mat-vw-access-exper}
shows the performance of accessing values through a \stapl\ matrix view
whose elements are atomic values, \stl\ views, or \stapl\ views.
\end{itemize}

\begin{figure}[p]
%%\includegraphics[scale=0.50]{figs/matrix_vw_assign}
\caption{Assign Values through Matrix View Execution Time}
\label{fig:mat-vw-assign-exper}
\end{figure}

\begin{figure}[p]
%%\includegraphics[scale=0.50]{figs/matrix_vw_access}
\caption{Access Values through Matrix View Execution Time}
\label{fig:mat-vw-access-exper}
\end{figure}

\emph{WRITE - reference complexity of underlying container}

% % % % % % % % % % % % % % % % % % % % % % % % % % % % % % % % % % % % % % 

\section{Read-Only Array View} \label{sec-aryro-vw}
\index{array!read-only view}
\index{view!array!read-only}

\subsection{Definition}

\textit{EXPAND - A view that provides that interface of a one-dimensional array abstract data type, which is read-only.}
\vspace{0.4cm}

\subsection{Relationship to \stapl\ array container}

\textit{WRITE}

\subsection{Relationship to other \stapl\ views}

\textit{WRITE}

\subsection{Implementation}

\textit{WRITE - details specific to this view}

\subsection{Interface} \label{sec-aryro-vw-inter}

\subsubsection{Constructors}

\noindent
\textbf{array\_ro\_view}%
\texttt{%
(view\_container\_type *
\textit{(vcont)}%
)
}

\begin{itemize}
\item
Constructs a view that can reference all the elements of the container provided. The view takes ownership of the container.
\end{itemize}

\noindent
\textbf{array\_ro\_view}%
\texttt{%
(view\_container\_type *
\textit{vcont,}%
domain\_type const \&
\textit{dom}%
)
}

\begin{itemize}
\item
Constructs a view with a restricted domain of the elements of the container provided. The view takes ownership of the container.
\end{itemize}

\noindent
\textbf{array\_ro\_view}%
\texttt{%
(view\_container\_type *
\textit{vcont,}%
domain\_type const \&
\textit{dom,}%
    map\_func\_type
\textit{mfunc}%
)
}

\begin{itemize}
\item
Constructs a view with a potentially restricted domain of the elements of the container provided and a non-identity mapping function. The view takes ownership of the container.
\end{itemize}

\noindent
\textbf{array\_ro\_view}%
\texttt{%
(view\_container\_type const \&
\textit{vcont}%
)
}

\begin{itemize}
\item
Constructs a view that can reference all the elements of the container provided. The view does not take ownership of the container.
\end{itemize}

\noindent
\textbf{array\_ro\_view}%
\texttt{%
(view\_container\_type const \&
\textit{vcont,}%
domain\_type const \&
\textit{dom}%
)
}

\begin{itemize}
\item
Constructs a view with a restricted domain of the elements of the container provided. The view does not take ownership of the container.
\end{itemize}

\noindent
\textbf{array\_ro\_view}%
\texttt{%
(view\_container\_type const \&
\textit{vcont,}%
domain\_type const \&
\textit{dom,}%
    map\_func\_type
\textit{mfunc}%
)
}

\begin{itemize}
\item
Constructs a view with a potentially restricted domain of the elements of the container provided and a non-identity mapping function. The view does not take ownership of the container.
\end{itemize}

\noindent
\textbf{array\_ro\_view}%
\texttt{%
(view\_container\_type const \&
\textit{vcont,}%
domain\_type const \&
\textit{dom,}%
map\_func\_type
\textit{mfunc,}%
array\_ro\_view const \&
)
}

\begin{itemize}
\item
Constructs a view with a potentially restricted domain of the elements of the container provided and a non-identity mapping function. The constructor accepts another ArrayRO instance from which it could copy additional state. The view does not take ownership of the container.
\end{itemize}

\noindent
\texttt{%
template<typename Derived1 >
}
\newline
\textbf{array\_ro\_view}%
\texttt{%
(array\_ro\_view< C, Dom, MapFunc, Derived1 > const \&
\textit{other}%
)
}

\begin{itemize}
\item
Copy constructor when the view provided is not the most derived view.
\end{itemize}

\noindent
\textbf{array\_ro\_view}%
\texttt{%
(array\_ro\_view const \&
\textit{other}%
)
}

\subsubsection{ View Attributes}

\noindent
\texttt{%
size\_type
}
\newline
\textbf{size}%
\texttt{%
(void) const
}

\begin{itemize}
\item
Returns the number of elements referenced for the view.
\end{itemize}

\noindent
\texttt{%
bool
}
\newline
\textbf{empty}%
\texttt{%
(void) const
}

\begin{itemize}
\item
Returns if the view does not reference any element.
\end{itemize}

\subsubsection{ Element Manipulation}

\noindent
\texttt{%
value\_type
}
\newline
\textbf{get\_element}%
\texttt{%
(index\_t const \&
\textit{index}%
) const
}

\begin{itemize}
\item
Get the element index from the container.
\end{itemize}

\noindent
\texttt{%
Functor::result\_type
}
\newline
\textbf{apply\_get}%
\texttt{%
(index\_t const \&
\textit{index,}%
Functor
f
)
}

\begin{itemize}
\item
Applies the provided function to the value referenced for the given index and returns the result of the operation.
\end{itemize}

\noindent
\texttt{%
reference
}
\newline
\textbf{operator[]}%
\texttt{%
(index\_t
\textit{index}%
) const
}

\begin{itemize}
\item
The bracket operator is the basic access method.
\end{itemize}

\noindent
\texttt{%
reference
}
\newline
\textbf{make\_reference}%
\texttt{%
(index\_t index) const
}

\subsection{Usage Example} \label{sec-aryro-vw-use}

The following example shows how to use the \texttt{array read-only view}:
%%\newline\vspace{0.4cm}\newline\rule{12cm}{0.5mm}
\input{view/arrayro.tex}
%%\noindent\rule{12cm}{0.5mm}\vspace{0.5cm}

\subsection{Implementation} \label{sec-aryro-vw-impl}

\textit{WRITE}

\subsection{Performance} \label{sec-aryro-vw-perf}

\begin{itemize}
%%\item
%%Fig. \ref{fig:roary-vw-assign-exper}
%%shows the performance of assigning values through a \stapl\ read-only array view
%%whose elements are atomic values, \stl\ views, or \stapl\ views.
\item
Fig. \ref{fig:roary-vw-access-exper}
shows the performance of accessing values through a \stapl\ read-only array view
whose elements are atomic values, \stl\ views, or \stapl\ views.
\end{itemize}

%%\begin{figure}[p]
%%\includegraphics[scale=0.50]{figs/read-only array_vw_assign}
%%\caption{Assign Values through Read-only Array View Execution Time}
%%\label{fig:roary-vw-assign-exper}
%%\end{figure}

\begin{figure}[p]
%%\includegraphics[scale=0.50]{figs/read-only array_vw_access}
\caption{Access Values through Read-only Array View Execution Time}
\label{fig:roary-vw-access-exper}
\end{figure}

\emph{WRITE - reference complexity of underlying container}

% % % % % % % % % % % % % % % % % % % % % % % % % % % % % % % % % % % % % % 

\section{Set View} \label{sec-set-vw}
\index{set!view}
\index{view!set}

\subsection{Definition}

\textit{EXPAND - A view that provides that interface of a set abstract data type.}

\subsection{Relationship to \stapl\ set container}

\textit{WRITE}

\subsection{Relationship to other \stapl\ views}

\textit{WRITE}

\subsection{Implementation}

\textit{WRITE - details specific to this view}

\subsection{Interface} \label{sec-set-vw-inter}

\subsubsection{Constructors}

\noindent
\textbf{set\_view}%
\texttt{%
(view\_container\_type *
\textit{vcont,}%
domain\_type const \&
\textit{dom,}%
map\_func\_type
\textit{mfunc=MapFunc()}%
)
}

\begin{itemize}
\item
Constructor used to pass ownership of the container to the view.
\end{itemize}

\noindent
\textbf{set\_view}%
\texttt{%
(view\_container\_type *
\textit{vcont,}%
domain\_type const \&
\textit{dom, }%
map\_func\_type
\textit{mfunc,}%
set\_view const \&
\textit{other}%
)
}

\begin{itemize}
\item
Constructor used to pass ownership of the container to the view.
\end{itemize}

\noindent
\textbf{set\_view}%
\texttt{%
(view\_container\_type const \&
\textit{vcont,}%
domain\_type const \&
\textit{dom,}%
map\_func\_type
\textit{mfunc=MapFunc(),}%
set\_view const \&
\textit{other=set\_view()}%
)
}

\begin{itemize}
\item
Constructor that does not takes ownership over the passed container.
\end{itemize}

\noindent
\textbf{set\_view}%
\texttt{%
(view\_container\_type *
\textit{vcont}%
)
}

\begin{itemize}
\item
Constructs a view that can reference all the elements of the passed container. The view takes ownership of the container.
\end{itemize}

\noindent
\textbf{set\_view}%
\texttt{%
(view\_container\_type \&
\textit{vcont}%
)
}

\begin{itemize}
\item
Constructs a view that can reference all the elements of the passed container.
\end{itemize}

\noindent
\texttt{%
template<typename Derived1 >
}
\newline
\textbf{set\_view}%
\texttt{%
(set\_view< C, Dom, MapFunc, Derived1 > const \&
\textit{other}%
)
}

\begin{itemize}
\item
\textit{WRITE - fix Doxygen and copy material here}
\end{itemize}

\subsubsection{ View Attributes}

\noindent
\texttt{%
size\_type
}
\newline
\textbf{size}%
\texttt{%
(void) const
}

\begin{itemize}
\item
Returns the number of elements referenced for the view.
\end{itemize}

\noindent
\texttt{%
bool
}
\newline
\textbf{empty}%
\texttt{%
(void) const
}

\begin{itemize}
\item
Returns if the view does not reference any element.
\end{itemize}


\subsubsection{ Element Manipulation}

\noindent
\texttt{%
void
}
\newline
\textbf{insert}%
\texttt{%
(key\_type const \&
\textit{key)}%
}

\begin{itemize}
\item
Inserts the given key if the key does not exist in the container.
\end{itemize}

\noindent
\texttt{%
void
}
\newline
\textbf{erase}%
\texttt{%
(key\_type const \&
\textit{key)}%
}

\begin{itemize}
\item
Removes the value associated with the specified key.
\end{itemize}

\noindent
\texttt{%
iterator
}
\newline
\textbf{find}%
\texttt{%
(key\_type const \&
\textit{key)}%
}

\begin{itemize}
\item
Returns an iterator pointing to the element associated with the given key.
\end{itemize}

\noindent
\texttt{%
void
}
\newline
\textbf{clear}%
\texttt{%
(void)
}

\begin{itemize}
\item
Removes all the elements stored in the container.
\end{itemize}

\noindent
\texttt{%
value\_type
}
\newline
\textbf{get\_element}%
\texttt{%
(index\_t const \&
\textit{index}%
) const
}

\begin{itemize}
\item
Get the element index from the container.
\end{itemize}

\noindent
\texttt{%
template<class Functor >
}
\texttt{%
Functor::result\_type
}
\newline
\textbf{apply\_get}%
\texttt{%
(index\_t const \&
\textit{index,}%
Functor
\textit{f}%
)
}

\begin{itemize}
\item
Applies the provided function to the value referenced for the given index and returns the result of the operation.
\end{itemize}

\noindent
\texttt{%
void
}
\newline
\textbf{set\_element}%
\texttt{%
(index\_t const \&
\textit{index,}%
value\_t const \&
\textit{value}%
)
}

\begin{itemize}
\item
Set the element index in the container to value value.
\end{itemize}

\noindent
\texttt{%
template<class Functor >
}
\texttt{%
void
}
\newline
\textbf{apply\_set}%
\texttt{%
(index\_t const \&
\textit{index,}%
Functor
\textit{f}%
)
}

\begin{itemize}
\item
Applies the provided function to the value referenced for the given index and mutates the element with the resulting value.
\end{itemize}

\noindent
\texttt{%
reference
}
\newline
\textbf{operator[]}%
\texttt{%
(index\_t
\textit{index}%
) const
}

\begin{itemize}
\item
The bracket operator is the basic access method.
\end{itemize}

\noindent
\texttt{%
reference
}
\newline
\textbf{make\_reference}%
\texttt{%
(index\_t
\textit{index}%
) const
}

\subsection{Usage Example} \label{sec-set-vw-use}

The following example shows how to use the \texttt{set view}:
%%\newline\vspace{0.4cm}\newline\rule{12cm}{0.5mm}
\input{view/set.tex}
%%\noindent\rule{12cm}{0.5mm}\vspace{0.5cm}

\subsection{Implementation} \label{sec-set-vw-impl}

\textit{WRITE}

\subsection{Performance} \label{sec-set-vw-perf}

\begin{itemize}
\item
Fig. \ref{fig:set-vw-assign-exper}
shows the performance of assigning values through a \stapl\ set view
whose elements are atomic values, \stl\ views, or \stapl\ views.
\item
Fig. \ref{fig:set-vw-access-exper}
shows the performance of accessing values through a \stapl\ set view
whose elements are atomic values, \stl\ views, or \stapl\ views.
\end{itemize}

\begin{figure}[p]
%%\includegraphics[scale=0.50]{figs/set_vw_assign}
\caption{Assign Values through Set View Execution Time}
\label{fig:set-vw-assign-exper}
\end{figure}

\begin{figure}[p]
%%\includegraphics[scale=0.50]{figs/set_vw_access}
\caption{Access Values through Set View Execution Time}
\label{fig:set-vw-access-exper}
\end{figure}

\emph{WRITE - reference complexity of underlying container}

% % % % % % % % % % % % % % % % % % % % % % % % % % % % % % % % % % % % % % 

\section{Graph View} \label{sec-graf-vw}
\index{graph!view}
\index{view!graph}

\subsection{Definition}

\textit{EXPAND - A view that provides that interface of a graph abstract data type.}

\subsection{Relationship to \stapl\ static and dynamic graph containers}

\textit{WRITE}

\subsection{Relationship to other \stapl\ views}

\textit{WRITE}

\subsection{Implementation}

\textit{WRITE - details specific to this view}

\subsection{Interface} \label{sec-graf-vw-inter}

\subsubsection{Constructors}

\noindent
\textbf{graph\_view}%
\texttt{%
(view\_container\_type *
\textit{vcont,}%
domain\_type const \&
\textit{dom,}%
map\_func\_type
\textit{mfunc=MapFunc()}%
)
}

\begin{itemize}
\item
Constructor used to pass ownership of the container to the view.
\end{itemize}

\noindent
\texttt{%
template<typename OV >
}
\newline
\textbf{graph\_view}%
\texttt{%
(view\_container\_type *
\textit{vcont,}%
domain\_type const \&
\textit{dom,}%
map\_func\_type
\textit{mfunc,}%
OV const \&
)
}

\begin{itemize}
\item
Constructor used to pass ownership of the container to the view.
\end{itemize}

\noindent
\textbf{graph\_view}%
\texttt{%
(view\_container\_type const \&
\textit{vcont,}%
domain\_type const \&
\textit{dom,}%
map\_func\_type
\textit{mfunc=MapFunc(),}%
graph\_view const \&
\textit{other=graph\_view()}%
)
}

\begin{itemize}
\item
Constructor that does not takes ownership over the passed container.
\end{itemize}

\noindent
\texttt{%
template<typename OV >
}
\newline
\textbf{graph\_view}%
\texttt{%
(view\_container\_type const \&
\textit{vcont,}%
domain\_type const \&
\textit{dom,}%
map\_func\_type
\textit{mfunc,}%
OV const \&
)
}

\begin{itemize}
\item
Constructor that does not takes ownership over the passed container.
\end{itemize}

\noindent
\textbf{graph\_view}%
\texttt{%
(view\_container\_type *
\textit{vcont}%
)
}

\begin{itemize}
\item
Constructs a view that can reference all the elements of the passed container. The view takes ownership of the container.
\end{itemize}

\noindent
\textbf{graph\_view}%
\texttt{%
(view\_container\_type \&
\textit{vcont}%
)
}

\begin{itemize}
\item
Constructs a view that can reference all the elements of the passed container.
\end{itemize}

\noindent
\textbf{graph\_view}%
\texttt{%
(view\_container\_type const \&
\textit{vcont}%
)
}

\begin{itemize}
\item
Constructs a view that can reference all the elements of the passed container.
\end{itemize}

\noindent
\texttt{%
template<typename Derived1 >
}
\newline
\textbf{graph\_view}%
\texttt{%
(graph\_view< PG, Dom, MapFunc, Derived1 > const \&
\textit{other}%
)
}

\begin{itemize}
\item
Copy constructor when the passed view is not the most derived view.
\end{itemize}

\noindent
\texttt{%
template<typename T1, typename T2 >
}
\newline
\textbf{graph\_view}%
\texttt{%
(graph\_view< PG, iterator\_domain< T1, T2 >, MapFunc, Derived > const \&
\textit{other}%
)
}

\begin{itemize}
\item
Copy constructor when the passed view has an iterator domain. This constructor converts a view using the default graph domain (iterator\_domain) to view using a gid-based domain (domset1D) as iterator domain can't handle a selective set of elements.
\end{itemize}

\subsubsection{ Element Manipulation}

\noindent
\texttt{%
vertex\_descriptor
}
\newline
\textbf{add\_vertex}%
\texttt{%
(void)
}

\begin{itemize}
\item
Adds a vertex to the graph with a default-constructed property.
\end{itemize}

\noindent
\texttt{%
vertex\_descriptor
}
\newline
\textbf{add\_vertex}%
\texttt{%
(vertex\_property const \&
\textit{vp}%
)
}

\begin{itemize}
\item
Adds a vertex to the graph with the given property.
\end{itemize}

\noindent
\texttt{%
vertex\_descriptor
}
\newline
\textbf{add\_vertex\_uniform}%
\texttt{%
(vertex\_property const \&
\textit{vp}%
)
}

\begin{itemize}
\item
Adds a vertex to the graph with the given property to a location based on the vertex descriptor assigned by the graph. This method is asynchronous. This method differs from the typical add\_vertex as it inserts the vertex into a potentially remote location, rather than the calling location.
\end{itemize}

\noindent
\texttt{%
void
}
\texttt{%
void
}
\newline
\textbf{add\_vertex}%
\texttt{%
(vertex\_descriptor const \&
\textit{vd,}%
vertex\_property const \&
\textit{vp}%
)
}

\begin{itemize}
\item
Adds a vertex to the graph with the given property and descriptor.
\end{itemize}

\noindent
\texttt{%
template<typename Functor >
}
\texttt{%
void
}
\newline
\textbf{add\_vertex}%
\texttt{%
(vertex\_descriptor const \&
\textit{vd,}%
vertex\_property const \&
\textit{vp,}%
Functor const \&
\textit{f}%
)
}

\begin{itemize}
\item
Adds a vertex to the graph with the given property and descriptor, if the vertex does not exist, or applies the given functor to the existing vertex. The vertex is added to the home location, unlike the other add\_vertex calls that add the vertex at the current location.
\end{itemize}

\noindent
\texttt{%
void
}
\newline
\textbf{delete\_vertex}%
\texttt{%
(vertex\_descriptor const \&
\textit{vd}%
)
}

\begin{itemize}
\item
\textit{WRITE - fix Doxygen and copy material here}
\end{itemize}

\noindent
\texttt{%
void
}
\newline
\textbf{add\_edge\_async}%
\texttt{%
(vertex\_descriptor const \&
\textit{src,}%
vertex\_descriptor const \&
\textit{tgt}%
)
}

\begin{itemize}
\item
Adds an edge between the two given vertices with given property. The edge is added asynchronously and method returns immediately. Edge is not guaranteed to be added until after a global synchronization.
\end{itemize}

\noindent
\texttt{%
void
}
\newline
\textbf{add\_edge\_async}%
\texttt{%
(vertex\_descriptor const \&
\textit{src,}%
vertex\_descriptor
const \&
\textit{tgt,}%
edge\_property const \&
\textit{p}%
)
}

\begin{itemize}
\item
Adds an edge between the two given vertices with given property. The edge is added asynchronously and method returns immediately. Edge is not guaranteed to be added until after a global synchronization.
\end{itemize}

\noindent
\texttt{%
void
}
\newline
\textbf{add\_edge\_async}%
\texttt{%
(edge\_descriptor const \&
\textit{ed}%
)
}

\begin{itemize}
\item
Adds an edge with given descriptor. The edge is added asynchronously and method returns immediately. Edge is not guaranteed to be added until after a global synchronization.
\end{itemize}

\noindent
\texttt{%
void
}
\newline
\textbf{add\_edge\_async}%
\texttt{%
(edge\_descriptor const \&
\textit{ed,}%
edge\_property const \&
\textit{p}%
)
}

\begin{itemize}
\item
Adds an edge with given descriptor and property. The edge is added asynchronously and method returns immediately. Edge is not guaranteed to be added until after a global synchronization.
\end{itemize}

\noindent
\texttt{%
edge\_descriptor
}
\newline
\textbf{add\_edge}%
\texttt{%
(vertex\_descriptor const \&
\textit{src,}%
vertex\_descriptor const \&
\textit{tgt}%
)
}

\begin{itemize}
\item
Adds an edge between the two given vertices.
\end{itemize}

\noindent
\texttt{%
edge\_descriptor
}
\newline
\textbf{add\_edge}%
\texttt{%
(vertex\_descriptor const \&
\textit{src,}%
vertex\_descriptor const \&
\textit{tgt,}%
edge\_property const \&
\textit{p}%
)
}

\begin{itemize}
\item
Adds an edge between the two given vertices with given property.
\end{itemize}

\noindent
\texttt{%
edge\_descriptor
}
\newline
\textbf{add\_edge}%
\texttt{%
(edge\_descriptor const \&
\textit{ed}%
)
}

\begin{itemize}
\item
Adds an edge with given descriptor.
\end{itemize}

\noindent
\texttt{%
edge\_descriptor
}
\newline
\textbf{add\_edge}%
\texttt{%
(edge\_descriptor const \&
\textit{ed,}%
edge\_property const \&
\textit{p}%
)
}

\begin{itemize}
\item
Adds an edge with given descriptor and property.
\end{itemize}

\noindent
\texttt{%
void
}
\newline
\textbf{%
 delete\_edge
}
\texttt{%
(vertex\_descriptor const \&
\textit{src,}%
vertex\_descriptor const \&
\textit{tgt}%
)
}

\begin{itemize}
\item
Deletes the edge between the given source and target vertices. The edge is deleted asynchronously. The edge is not guaranteed to have been deleted until after a global synchronization.
\end{itemize}

\noindent
\texttt{%
void
}
\newline
\textbf{%
 delete\_edge
}
\texttt{%
(edge\_descriptor const \&
ed
\textit{)}%
}

\begin{itemize}
\item
Deletes the edge with given descriptor. Asynchronous.
\end{itemize}

\noindent
\texttt{%
void
}
\newline
\textbf{clear}%
\texttt{%
(void)
}

\begin{itemize}
\item
Clears the graph. This resets internal counters for vertex-descriptor and edge-id assignments, and clears graph storage.
\end{itemize}

\noindent
\texttt{%
vertex\_iterator
}
\newline
\textbf{find\_vertex}%
\texttt{%
(vertex\_descriptor const \&
\textit{vd}%
) const
}

\begin{itemize}
\item
Returns a global vertex iterator to the given descriptor.
\end{itemize}

\noindent
\texttt{%
void
}
\newline
\textbf{sort\_edges}%
\texttt{%
(void)
}

\begin{itemize}
\item
Sorts edges of each vertex by home-location of target-vertex.
\end{itemize}

\noindent
\texttt{%
template<typename F >
}
\texttt{%
void
}
\newline
\textbf{apply\_set}%
\texttt{%
(vertex\_descriptor const \&
\textit{gid,}%
F const \&
\textit{f}%
)
}

\begin{itemize}
\item
Applies a function to the vertex with the given descriptor.
\end{itemize}

\noindent
\texttt{%
template<class Functor >
}
\texttt{%
void
}
\newline
\textbf{vp\_apply\_async}%
\texttt{%
(vertex\_descriptor const \&
\textit{vd,}%
Functor const \&
\textit{f}%
)
}

\begin{itemize}
\item
Applies a function to the property of vertex with given descriptor. This method is asynchronous.
\end{itemize}

\noindent
\texttt{%
template<typename Functor >
}
\texttt{%
Functor::result\_type
}
\newline
\textbf{vp\_apply}%
\texttt{%
(vertex\_descriptor const \&
\textit{vd,}%
Functor const \&
\textit{f}%
) const
}

\begin{itemize}
\item
Applies a function to the property of vertex with given descriptor.
\end{itemize}

\noindent
\texttt{%
template<class Functor >
}
\texttt{%
void ep\_apply\_async
}
\newline
\textbf{ep\_apply\_async}%
\texttt{%
(edge\_descriptor const \&
\textit{ed,}%
Functor const \&
\textit{f}%
)
}

\begin{itemize}
\item
Applies a function to the property of edge with given descriptor. This method is asynchronous.
\end{itemize}

\noindent
\texttt{%
template<class Functor >
}
\texttt{%
Functor::result\_type
}
\newline
\textbf{ep\_apply}%
\texttt{%
(edge\_descriptor const \&
\textit{ed,}%
Functor const \&
\textit{f })}%

\begin{itemize}
\item
Applies a function to the property of edge with given descriptor.
\end{itemize}

\noindent
\texttt{%
vertices\_view\_type
}
\newline
\textbf{vertices}%
\texttt{%
(void) const
}

\begin{itemize}
\item
Returns an Array over the vertices of the graph.
\end{itemize}

\noindent
\texttt{%
value\_type
}
\newline
\textbf{get\_element}%
\texttt{%
(index\_t const \&
\textit{index}%
) const
}

\begin{itemize}
\item
Get the element index from the container.
\end{itemize}

\noindent
\texttt{%
template<class Functor >
}
\texttt{%
Functor::result\_type
}
\newline
\textbf{apply\_get}%
\texttt{%
(index\_t const \&
\textit{index,}%
Functor
\textit{f}%
)
}

\begin{itemize}
\item
Applies the provided function to the value referenced for the given index and returns the result of the operation.
\end{itemize}

\noindent
\texttt{%
void
}
\newline
\textbf{set\_element}%
\texttt{%
(index\_t const \&
\textit{index,}%
value\_t const \&
\textit{value}%
)
}

\begin{itemize}
\item
Set the element index in the container to value value.
\end{itemize}

\noindent
\texttt{%
template<class Functor >
}
\texttt{%
void
}
\newline
\textbf{apply\_set}%
\texttt{%
(index\_t const \&
\textit{index,}%
Functor
\textit{f}%
)
}

\begin{itemize}
\item
Applies the provided function to the value referenced for the given index and mutates the element with the resulting value.
\end{itemize}

\noindent
\texttt{%
reference
}
\newline
\textbf{operator[]}%
\texttt{%
(index\_t
\textit{index}%
) const
}

\begin{itemize}
\item
The bracket operator is the basic access method.
\end{itemize}

\noindent
\texttt{%
reference
}
\newline
\textbf{make\_reference}%
\texttt{%
(index\_t
\textit{index }) const}%

\subsubsection{ View Attributes}

\noindent
\texttt{%
size\_t
}
\newline
\textbf{num\_vertices}%
\texttt{%
(void) const
}

\begin{itemize}
\item
Return the number of vertices in the graph.
\end{itemize}

\noindent
\texttt{%
size\_t
}
\newline
\textbf{num\_edges}%
\texttt{%
(void) const
}

\begin{itemize}
\item
Returns the number of edges in the graph. This method is a non-collective version of num\_edges. This must be used when not all locations are calling num\_edges.
\end{itemize}

\noindent
\texttt{%
size\_t
}
\newline
\textbf{num\_edges\_collective}%
\texttt{%
(void) const
}

\begin{itemize}
\item
Returns the number of edges in the graph.
\end{itemize}

\noindent
\texttt{%
size\_t
}
\newline
\textbf{num\_local\_edges}%
\texttt{%
(void) const
}

\begin{itemize}
\item
Returns the number of local outgoing edges in the graph. This is a non-blocking method.
\end{itemize}

\noindent
\texttt{%
bool
}
\newline
\textbf{is\_directed}%
\texttt{%
(void) const
}

\noindent
\texttt{%
size\_type
}
\newline
\textbf{size}%
\texttt{%
(void) const
}

\begin{itemize}
\item
Returns the number of elements referenced for the view.
\end{itemize}

\noindent
\texttt{%
bool
}
\newline
\textbf{empty}%
\texttt{%
(void) const
}

\begin{itemize}
\item
Returns if the view does not reference any element.
\end{itemize}

\begin{itemize}
\item
\textit{WRITE - fix Doxygen and copy material here}
\end{itemize}

\subsection{Usage Example} \label{sec-graf-vw-use}

The following example shows how to use the \texttt{graph view}:
%%\newline\vspace{0.4cm}\newline\rule{12cm}{0.5mm}
\input{view/graph.tex}
%%\noindent\rule{12cm}{0.5mm}\vspace{0.5cm}

\subsection{Implementation} \label{sec-graf-vw-impl}

\textit{WRITE}

\subsection{Performance} \label{sec-graf-vw-perf}

\begin{itemize}
\item
Fig. \ref{fig:graf-vw-assign-exper}
shows the performance of assigning values through a \stapl\ graph view
whose elements are atomic values, \stl\ views, or \stapl\ views.
\item
Fig. \ref{fig:graf-vw-access-exper}
shows the performance of accessing values through a \stapl\ graph view
whose elements are atomic values, \stl\ views, or \stapl\ views.
\end{itemize}

\begin{figure}[p]
%%\includegraphics[scale=0.50]{figs/graph_vw_assign}
\caption{Assign Values through Graph View Execution Time}
\label{fig:graf-vw-assign-exper}
\end{figure}

\begin{figure}[p]
%%\includegraphics[scale=0.50]{figs/graph_vw_access}
\caption{Access Values through Graph View Execution Time}
\label{fig:graf-vw-access-exper}
\end{figure}

\emph{WRITE - reference complexity of underlying container}

% % % % % % % % % % % % % % % % % % % % % % % % % % % % % % % % % % % % % % 

\section {Overlap View} \label{sec-over-vw}
\index{overlap view}
\index{view!overlap}

\subsection{Definition}

\textit{EXPAND - A view that provides overlapped access to elements.}

\subsection{Relationship to other \stapl\ views}

\textit{WRITE}

\subsection{Implementation}

\textit{WRITE - details specific to this view}

\subsection{Interface} \label{sec-over-vw-inter}

\subsubsection{Constructors}

\textit{WRITE}

\subsubsection{ Element Manipulation}

\noindent
\texttt{%
view\_type
}
\newline
\textbf{operator()}%
\texttt{%
(View \&
\textit{v,}%
size\_t
\textit{c=1,}%
size\_t
\textit{l=0, }%
size\_t
\textit{r=0}%
)
}

\begin{itemize}
\item
\textit{WRITE - fix Doxygen and copy material here}
\end{itemize}

\noindent
\texttt{%
view\_type
}
\newline
\textbf{operator()}%
\texttt{%
(View const \&
\textit{v,}%
size\_t
\textit{c=1,}%
size\_t
\textit{l=0,}%
size\_t
\textit{r=0}%
)
}

\begin{itemize}
\item
\textit{WRITE - fix Doxygen and copy material here}
\end{itemize}

\subsection{Usage Example} \label{sec-over-vw-use}

The following example shows how to use the \texttt{overlap view}:
%%\newline\vspace{0.4cm}\newline\rule{12cm}{0.5mm}
\input{view/overlap.tex}
%%\noindent\rule{12cm}{0.5mm}\vspace{0.5cm}

\subsection{Implementation} \label{sec-over-vw-impl}

\textit{WRITE}

\subsection{Performance} \label{sec-over-vw-perf}

\begin{itemize}
\item
Fig. \ref{fig:over-vw-assign-exper}
shows the performance of assigning values through a \stapl\ overlap view
whose elements are atomic values, \stl\ views, or \stapl\ views.
\item
Fig. \ref{fig:over-vw-access-exper}
shows the performance of accessing values through a \stapl\ overlap view
whose elements are atomic values, \stl\ views, or \stapl\ views.
\end{itemize}

\begin{figure}[p]
%%\includegraphics[scale=0.50]{figs/overlap_vw_assign}
\caption{Assign Values through Overlap View Execution Time}
\label{fig:over-vw-assign-exper}
\end{figure}

\begin{figure}[p]
%%\includegraphics[scale=0.50]{figs/overlap_vw_access}
\caption{Access Values through Overlap View Execution Time}
\label{fig:over-vw-access-exper}
\end{figure}

\emph{WRITE - complexity}

% % % % % % % % % % % % % % % % % % % % % % % % % % % % % % % % % % % % % % 

\section{Repeat View} \label{sec-rep-vw}
\index{repeat view}
\index{view!repeat}

\subsection{Definition}

\textit{EXPAND - A view that can produce infinite copies of the same value.}

\subsection{Relationship to other \stapl\ views}

\textit{WRITEe}

\subsection{Implementation}

\textit{WRITE - details specific to this view}

\subsection{Interface} \label{sec-rep-vw-inter}

\subsubsection{Constructors}

\noindent
\textbf{repeated\_view}%
\texttt{%
(view\_container\_type *
\textit{vcont,}%
domain\_type const \&
\textit{dom,}%
map\_func\_type
\textit{mfunc=map\_func\_type()}%
)
}

\begin{itemize}
\item
Constructor used to pass ownership of the container to the view.
\end{itemize}

\noindent
\textbf{repeated\_view}%
\texttt{%
(view\_container\_type const \&
\textit{vcont,}%
domain\_type const \&
\textit{dom,}%
map\_func\_type
\textit{mfunc=map\_func\_type(),}%
repeated\_view const \&
\textit{other=repeated\_view()}%
)
}

\begin{itemize}
\item
Constructor that does not takes ownership over the passed container.
\end{itemize}

\noindent
\textbf{repeated\_view}%
\texttt{%
(value\_type const \&
\textit{data}%
)
}

\begin{itemize}
\item
Constructs a view that represents an infinite array of the same element (data).
\end{itemize}

\noindent
\textbf{repeated\_view}%
\texttt{%
(repeated\_view const \&
\textit{other}%
)
}

\subsubsection{ View Attributes}

\noindent
\texttt{%
size\_type
}
\newline
\textbf{size}%
\texttt{%
(void) const
}

\begin{itemize}
\item
Returns the number of elements referenced for the view.
\end{itemize}

\noindent
\texttt{%
bool
}
\newline
\textbf{empty}%
\texttt{%
(void) const
}

\begin{itemize}
\item
Returns if the view does not reference any element.
\end{itemize}

\noindent
\texttt{%
value\_type
}
\newline
\textbf{get\_element}%
\texttt{%
(index\_t const \&
\textit{index}%
) const
}

\begin{itemize}
\item
Get the element index from the container.
\end{itemize}

\noindent
\texttt{%
Functor::result\_type
}
\newline
\textbf{apply\_get}%
\texttt{%
(index\_t const \&
\textit{index,}%
Functor
\textit{f}%
)
}

\begin{itemize}
\item
Applies the provided function to the value referenced for the given index and returns the result of the operation.
\end{itemize}

\noindent
\texttt{%
reference
}
\newline
\textbf{operator[]}%
\texttt{%
(index\_t
\textit{%
index
}
) const
}

\begin{itemize}
\item
The bracket operator is the basic access method.
\end{itemize}

\noindent
\texttt{%
reference
}
\newline
\textbf{make\_reference}%
\texttt{%
(index\_t
\textit{index}%
) const
}

\subsection{Usage Example} \label{sec-rep-vw-use}

The following example shows how to use the \texttt{repeat view}:
%%\newline\vspace{0.4cm}\newline\rule{12cm}{0.5mm}
\input{view/repeat.tex}
%%\noindent\rule{12cm}{0.5mm}\vspace{0.5cm}

\subsection{Implementation} \label{sec-rep-vw-impl}

\textit{WRITE}

\subsection{Performance} \label{sec-rep-vw-perf}

\begin{itemize}
\item
Fig. \ref{fig:rep-vw-assign-exper}
shows the performance of assigning values through a \stapl\ repeat view
whose elements are atomic values, \stl\ views, or \stapl\ views.
\item
Fig. \ref{fig:rep-vw-access-exper}
shows the performance of accessing values through a \stapl\ repeat view
whose elements are atomic values, \stl\ views, or \stapl\ views.
\end{itemize}

\begin{figure}[p]
%%\includegraphics[scale=0.50]{figs/repeat_vw_assign}
\caption{Assign Values through Repeat View Execution Time}
\label{fig:rep-vw-assign-exper}
\end{figure}

\begin{figure}[p]
%%\includegraphics[scale=0.50]{figs/repeat_vw_access}
\caption{Access Values through Repeat View Execution Time}
\label{fig:rep-vw-access-exper}
\end{figure}

\emph{WRITE - complexity}

% % % % % % % % % % % % % % % % % % % % % % % % % % % % % % % % % % % % % % 

\section{Counting View} \label{sec-cnt-vw}
\index{counting view}
\index{view!counting}

\subsection{Definition}

\textit{EXPAND - A view that provides a sequence of integers.}

\subsection{Relationship to other \stapl\ views}

\textit{WRITE e}

\subsection{Implementation}

\textit{WRITE - details specific to this view}

\subsection{Interface} \label{sec-cnt-vw-inter}

\subsubsection{Classes}

\noindent
\texttt{%
struct stapl::view\_impl::counting\_container< T >
}

\begin{itemize}
\item
Container that represents an increasing sequence of elements.
\end{itemize}

\noindent
\texttt{%
struct stapl::result\_of::counting\_view< T >
}

\begin{itemize}
\item
Defines the type of a counting view parameterized with T.
\end{itemize}

\subsubsection{Functions}

\noindent
\texttt{%
template<typename T >
}
\texttt{%
result\_of::counting\_view< T >::type
}
\newline
\textbf{stapl::counting\_view}%
\texttt{%
(size\_t
\textit{n,}%
T
\textit{init=0}%
)
}

\begin{itemize}
\item
Helper function that creates a read-only view representing a set of increasing elements.
\end{itemize}

\subsection{Usage Example} \label{sec-cnt-vw-use}

The following example shows how to use the \texttt{counting view}:
%%\newline\vspace{0.4cm}\newline\rule{12cm}{0.5mm}
\input{view/counting.tex}
%%\noindent\rule{12cm}{0.5mm}\vspace{0.5cm}

\subsection{Implementation} \label{sec-cnt-vw-impl}

\textit{WRITE}

\subsection{Performance} \label{sec-cnt-vw-perf}

\begin{itemize}
\item
Fig. \ref{fig:cnt-vw-assign-exper}
shows the performance of assigning values through a \stapl\ counting view
whose elements are atomic values, \stl\ views, or \stapl\ views.
\item
Fig. \ref{fig:cnt-vw-access-exper}
shows the performance of accessing values through a \stapl\ counting view
whose elements are atomic values, \stl\ views, or \stapl\ views.
\end{itemize}

\begin{figure}[p]
%%\includegraphics[scale=0.50]{figs/counting_vw_assign}
\caption{Assign Values through Counting View Execution Time}
\label{fig:cnt-vw-assign-exper}
\end{figure}

\begin{figure}[p]
%%\includegraphics[scale=0.50]{figs/counting_vw_access}
\caption{Access Values through Counting View Execution Time}
\label{fig:cnt-vw-access-exper}
\end{figure}

\emph{WRITE - complexity}

% % % % % % % % % % % % % % % % % % % % % % % % % % % % % % % % % % % % % % 

\section{Functor View} \label{sec-func-vw}
\index{functor view}
\index{view!functor}

\subsection{Definition}

\textit{EXPAND - A view that creates an array view on top of a functor container.}

\subsection{Relationship to other \stapl\ views}

\textit{WRITEe}

\subsection{Implementation}

\textit{WRITE - details specific to this view}

\subsection{Interface} \label{sec-func-vw-inter}

\subsubsection{Constructors}

\noindent
\texttt{%
template<typename Func>
}
\texttt{%
typename functor\_view\_type<Func>::type
}
\newline
\textbf{functor\_view}%
\texttt{%
(size\_t const \&
\textit{n,}%
Func const \&
\textit{func}%
)
}

\begin{itemize}
\item
Create an array view on top of a functor container.
\end{itemize}

\subsection{Usage Example} \label{sec-fun-vw-use}

The following example shows how to use the \texttt{functor view}:
%%\newline\vspace{0.4cm}\newline\rule{12cm}{0.5mm}
\input{view/functor.tex}
%%\noindent\rule{12cm}{0.5mm}\vspace{0.5cm}

\subsection{Implementation} \label{sec-fun-vw-impl}

\textit{WRITE}

\subsection{Performance} \label{sec-fun-vw-perf}

\begin{itemize}
\item
Fig. \ref{fig:fun-vw-assign-exper}
shows the performance of assigning values through a \stapl\ Functor view
whose elements are atomic values, \stl\ views, or \stapl\ views.
\item
Fig. \ref{fig:fun-vw-access-exper}
shows the performance of accessing values through a \stapl\ Functor view
whose elements are atomic values, \stl\ views, or \stapl\ views.
\end{itemize}

\begin{figure}[p]
%%\includegraphics[scale=0.50]{figs/Functor_vw_assign}
\caption{Assign Values through Functor View Execution Time}
\label{fig:fun-vw-assign-exper}
\end{figure}

\begin{figure}[p]
%%\includegraphics[scale=0.50]{figs/Functor_vw_access}
\caption{Access Values through Functor View Execution Time}
\label{fig:fun-vw-access-exper}
\end{figure}

\emph{WRITE - complexity}

% % % % % % % % % % % % % % % % % % % % % % % % % % % % % % % % % % % % % % 

\section{Strided View} \label{sec-str-vw}
\index{strided view}
\index{view!strided}

\subsection{Definition}

\textit{WRITE}

\subsection{Relationship to other \stapl\ views}

\textit{WRITE}

\subsection{Implementation}

\textit{WRITE - details specific to this view}

\subsection{Interface} \label{sec-str-vw-inter}

\subsubsection{Classes}

\noindent
\texttt{%
struct stapl::strided\_view\_type< BV >
}

\begin{itemize}
\item
A metafunction that computes the type of a strided view based on a given view.
\end{itemize}

\subsubsection{Functions}

\noindent
\texttt{%
template<typename BV >
}
\texttt{%
strided\_view\_type< BV >::type
}
\newline
\textbf{stapl::make\_strided\_view}%
\texttt{%
(BV
\textit{view,}%
typename view\_traits< BV >::index\_type const \&
\textit{step,}%
typename view\_traits< BV >::index\_type const \&
\textit{start}%
)
}

\begin{itemize}
\item
Takes an input view and creates a strided view over the original view's container.
\end{itemize}

\subsection{Usage Example} \label{sec-str-vw-use}

The following example shows how to use the \texttt{strided view}:
%%\newline\vspace{0.4cm}\newline\rule{12cm}{0.5mm}
\input{view/strided.tex}
%%\noindent\rule{12cm}{0.5mm}\vspace{0.5cm}

\subsection{Implementation} \label{sec-str-vw-impl}

\textit{WRITE}

\subsection{Performance} \label{sec-str-vw-perf}

\begin{itemize}
\item
Fig. \ref{fig:str-vw-assign-exper}
shows the performance of assigning values through a \stapl\ Strided view
whose elements are atomic values, \stl\ views, or \stapl\ views.
\item
Fig. \ref{fig:str-vw-access-exper}
shows the performance of accessing values through a \stapl\ Strided view
whose elements are atomic values, \stl\ views, or \stapl\ views.
\end{itemize}

\begin{figure}[p]
%%\includegraphics[scale=0.50]{figs/Strided_vw_assign}
\caption{Assign Values through Strided View Execution Time}
\label{fig:str-vw-assign-exper}
\end{figure}

\begin{figure}[p]
%%\includegraphics[scale=0.50]{figs/Strided_vw_access}
\caption{Access Values through Strided View Execution Time}
\label{fig:str-vw-access-exper}
\end{figure}

\emph{WRITE - complexity}

% % % % % % % % % % % % % % % % % % % % % % % % % % % % % % % % % % % % % % 

\section{Reverse View} \label{sec-rev-vw}
\index{reverse view}
\index{view!reverse}

\subsection{Definition}

\textit{EXPAND - A view that reverses the order of the elements in a sequence.}

\subsection{Relationship to other \stapl\ views}

\textit{WRITE}

\subsection{Implementation}

\textit{WRITE - details specific to this view}

\subsection{Interface} \label{sec-rev-vw-inter}

\subsubsection{Constructors}

\noindent
\textbf{reverse\_view}%
\texttt{%
(View const \&
\textit{view}%
)
}

\subsubsection{Classes}
\begin{itemize}
\item
Constructs a reverse view over the given view.
\end{itemize}

\noindent
\textbf{reverse\_view}%
\texttt{%
(view\_container\_type *
\textit{vcont,}%
domain\_type const \&
\textit{dom,}%
map\_func\_type
\textit{mfunc=map\_func\_type()}%
)
}

\begin{itemize}
\item
Constructor used to pass ownership of the container to the view.
\end{itemize}

\noindent
\textbf{reverse\_view}%
\texttt{%
(view\_container\_type const \&
\textit{vcont,}%
domain\_type const \&
\textit{dom,}%
map\_func\_type
\textit{mfunc,}%
reverse\_view const \&
\textit{other}%
)
}

\begin{itemize}
\item
Constructor that does not takes ownership over the passed container.
\end{itemize}

\noindent
\textbf{reverse\_view}%
\texttt{%
(view\_container\_type const \&
\textit{vcont,}%
domain\_type const \&
\textit{dom,}%
map\_func\_type
\textit{mfunc=map\_func\_type()}%
)
}

\begin{itemize}
\item
Constructor that does not takes ownership over the passed container.
\end{itemize}

\noindent
\textbf{reverse\_view}%
\texttt{%
(reverse\_view const \&
\textit{other}%
)
}

\subsubsection{ View Attributes}

\noindent
\texttt{%
size\_type
}
\newline
\textbf{size}%
\texttt{%
(void) const
}

\begin{itemize}
\item
Returns the number of elements referenced for the view.
\end{itemize}

\noindent
\texttt{%
bool
}
\newline
\textbf{empty}%
\texttt{%
(void) const
}

\begin{itemize}
\item
Returns if the view does not reference any element.
\end{itemize}

\subsection{Usage Example} \label{sec-rev-vw-use}

The following example shows how to use the \texttt{reverse view}:
%%\newline\vspace{0.4cm}\newline\rule{12cm}{0.5mm}
\input{view/reverse.tex}
%%\noindent\rule{12cm}{0.5mm}\vspace{0.5cm}

\subsection{Implementation} \label{sec-rev-vw-impl}

\textit{WRITE}

\subsection{Performance} \label{sec-rev-vw-perf}

\begin{itemize}
\item
Fig. \ref{fig:rev-vw-assign-exper}
shows the performance of assigning values through a \stapl\ Reverse view
whose elements are atomic values, \stl\ views, or \stapl\ views.
\item
Fig. \ref{fig:rev-vw-access-exper}
shows the performance of accessing values through a \stapl\ Reverse view
whose elements are atomic values, \stl\ views, or \stapl\ views.
\end{itemize}

\begin{figure}[p]
%%\includegraphics[scale=0.50]{figs/Reverse_vw_assign}
\caption{Assign Values through Reverse View Execution Time}
\label{fig:rev-vw-assign-exper}
\end{figure}

\begin{figure}[p]
%%\includegraphics[scale=0.50]{figs/Reverse_vw_access}
\caption{Access Values through Reverse View Execution Time}
\label{fig:rev-vw-access-exper}
\end{figure}

\emph{WRITE - complexity}

% % % % % % % % % % % % % % % % % % % % % % % % % % % % % % % % % % % % % % 

\section{Filter View} \label{sec-filt-vw}
\index{filter view}
\index{view!filter}

\subsection{Definition}

\textit{EXPAND - A view that filters the elements in a sequence.}

\subsection{Relationship to other \stapl\ views}

\textit{WRITE}

\subsection{Implementation}

\textit{WRITE - details specific to this view}

\subsection{Interface} \label{sec-filt-vw-inter}

\subsubsection{Constructors}

\noindent
\textbf{filter\_view}%
\texttt{%
(View const \&
\textit{view,}%
Pred const \&
\textit{pred}%
)
}

\begin{itemize}
\item
Constructs a filter view over the given view.
\end{itemize}

\noindent
\textbf{filter\_view}%
\texttt{%
(view\_container\_type *
\textit{vcont,}%
domain\_type const \&
\textit{dom,}%
map\_func\_type
\textit{mfunc=map\_func\_type(),}%
Pred const \&
\textit{pred=Pred()}%
)
}

\begin{itemize}
\item
Constructor used to pass ownership of the container to the view.
\end{itemize}

\noindent
\textbf{filter\_view}%
\texttt{%
(view\_container\_type const \&
\textit{vcont,}%
domain\_type const \&
\textit{dom,}%
map\_func\_type
\textit{mfunc=map\_func\_type(),}%
Pred const \&
\textit{pred=Pred()}%
)
}

\begin{itemize}
\item
Constructor that does not takes ownership over the passed container.
\end{itemize}

\noindent
\textbf{filter\_view}%
\texttt{%
(view\_container\_type const \&
\textit{vcont,}%
domain\_type const \&
\textit{dom,}%
map\_func\_type
\textit{mfunc,}%
filter\_view const \&
\textit{other}%
)
}

\noindent
\textbf{filter\_view}%
\texttt{%
(filter\_view const \&
\textit{other}%
)
}

\begin{itemize}
\item
Constructor that does not takes ownership over the passed container.
\end{itemize}

\subsubsection{ Element Manipulation}

\noindent
\texttt{%
iterator
}
\newline
\textbf{begin}%
\texttt{%
(void)
}

\begin{itemize}
\item
Return an iterator over the element whose GID is the first valid index,
\end{itemize}
based on applying the predicate to the element value.

\noindent
\texttt{%
const\_iterator
}
\newline
\textbf{begin}%
\texttt{%
(void) const
}

\begin{itemize}
\item
Return an iterator over the element whose GID is the first valid index,
\end{itemize}
based on applying the predicate to the element value.

\noindent
\texttt{%
iterator
}
\newline
\textbf{end}%
\texttt{%
(void)
}

\begin{itemize}
\item
Return an iterator over the element whose GID is the last valid index,
\end{itemize}
based on applying the predicate to the element value.

\noindent
\texttt{%
const\_iterator
}
\newline
\textbf{end}%
\texttt{%
(void) const
}

\begin{itemize}
\item
Return an iterator over the element whose GID is the last valid index,
\end{itemize}
based on applying the predicate to the element value.

\noindent
\texttt{%
index\_type
}
\newline
\textbf{next}%
\texttt{%
\textit{(index\_type}%
index) const
}

\begin{itemize}
\item
Return an iterator over the element whose GID is the next valid index,
\end{itemize}
based on applying the predicate to the element value.

\noindent
\texttt{%
value\_type
}
\newline
\textbf{get\_element}%
\texttt{%
(index\_type
\textit{index}%
)
}

\begin{itemize}
\item
Returns the value at the specified index.
\end{itemize}

\noindent
\texttt{%
Pred const \&
}
\newline
\textbf{predicate}%
\texttt{%
(void) const
}

\begin{itemize}
\item
Returns the predicate used to filter the values.
\end{itemize}

\noindent
\texttt{%
size\_t
}
\newline
\textbf{size}%
\texttt{%
(void) const
}

\begin{itemize}
\item
Returns the number of elements referenced for the view.
\end{itemize}

\noindent
\texttt{%
bool
}
\newline
\textbf{empty}%
\texttt{%
(void) const
}

\begin{itemize}
\item
Returns true if the view does not reference any element.
\end{itemize}

\subsection{Usage Example} \label{sec-filt-vw-use}

The following example shows how to use the \texttt{filter view}:
%%\newline\vspace{0.4cm}\newline\rule{12cm}{0.5mm}
\input{view/filter.tex}
%%\noindent\rule{12cm}{0.5mm}\vspace{0.5cm}

\subsection{Implementation} \label{sec-filt-vw-impl}

\textit{WRITE}

\subsection{Performance} \label{sec-filt-vw-perf}

\begin{itemize}
\item
Fig. \ref{fig:filt-vw-assign-exper}
shows the performance of assigning values through a \stapl\ Filter view
whose elements are atomic values, \stl\ views, or \stapl\ views.
\item
Fig. \ref{fig:filt-vw-access-exper}
shows the performance of accessing values through a \stapl\ Filter view
whose elements are atomic values, \stl\ views, or \stapl\ views.
\end{itemize}

\begin{figure}[p]
%%\includegraphics[scale=0.50]{figs/Filter_vw_assign}
\caption{Assign Values through Filter View Execution Time}
\label{fig:filt-vw-assign-exper}
\end{figure}

\begin{figure}[p]
%%\includegraphics[scale=0.50]{figs/Filter_vw_access}
\caption{Access Values through Filter View Execution Time}
\label{fig:filt-vw-access-exper}
\end{figure}

\emph{WRITE - complexity}

% % % % % % % % % % % % % % % % % % % % % % % % % % % % % % % % % % % % % % 

\section{Transform View} \label{sec-trans-vw}
\index{transform view}
\index{view!transform}

\subsection{Definition}

\textit{EXPAND - A view that applies the given functor to the elements when they are returned by get data methods.}

\subsection{Relationship to other \stapl\ views}

\textit{WRITE}

\subsection{Implementation}

\textit{WRITE - details specific to this view}

\subsection{Interface} \label{sec-trans-vw-inter}

\subsubsection{Constructors}

\noindent
\textbf{transform\_view}%
\texttt{%
(container\_t \&
\textit{vcont,}%
domain\_t const \&
\textit{dom,}%
map\_func\_t
\textit{mfunc,}%
transform\_view const \&
\textit{other}%
)
}

\begin{itemize}
\item
\textit{WRITE - fix Doxygen  and copy material here}
\end{itemize}

\noindent
\textbf{transform\_view}%
\texttt{%
(container\_t \&
\textit{vcont,}%
domain\_t const \&
\textit{dom,}%
map\_func\_t
\textit{mfunc=map\_func\_t()}%
)
}

\begin{itemize}
\item
\textit{WRITE - fix Doxygen and copy material here}
\end{itemize}

\noindent
\textbf{transform\_view}%
\texttt{%
(View const \&
\textit{view,}%
Functor const \&
\textit{func}%
)
}

\begin{itemize}
\item
\textit{WRITE - fix Doxygen  and copy material here}
\end{itemize}

\subsubsection{ Element Manipulation}

\noindent
\texttt{%
reference
}
\newline
\textbf{operator[]}%
\texttt{%
(index\_type
\textit{g}%
) const
}

\begin{itemize}
\item
\textit{WRITE - fix Doxygen and copy material here}
\end{itemize}

\noindent
\texttt{%
Functor const \&
}
\newline
\textbf{func}%
\texttt{%
(void) const
}

\subsection{Usage Example} \label{sec-trans-vw-use}

The following example shows how to use the \texttt{transform view}:
%%\newline\vspace{0.4cm}\newline\rule{12cm}{0.5mm}
\input{view/transform.tex}
%%\noindent\rule{12cm}{0.5mm}\vspace{0.5cm}

\subsection{Implementation} \label{sec-trans-vw-impl}

\textit{WRITE}

\subsection{Performance} \label{sec-trans-vw-perf}

\begin{itemize}
\item
Fig. \ref{fig:trans-vw-assign-exper}
shows the performance of assigning values through a \stapl\ Transform view
whose elements are atomic values, \stl\ views, or \stapl\ views.
\item
Fig. \ref{fig:trans-vw-access-exper}
shows the performance of accessing values through a \stapl\ Transform view
whose elements are atomic values, \stl\ views, or \stapl\ views.
\end{itemize}

\begin{figure}[p]
%%\includegraphics[scale=0.50]{figs/Transform_vw_assign}
\caption{Assign Values through Transform View Execution Time}
\label{fig:trans-vw-assign-exper}
\end{figure}

\begin{figure}[p]
%%\includegraphics[scale=0.50]{figs/Transform_vw_access}
\caption{Access Values through Transform View Execution Time}
\label{fig:trans-vw-access-exper}
\end{figure}

\emph{WRITE - complexity}

% % % % % % % % % % % % % % % % % % % % % % % % % % % % % % % % % % % % % % 

\section{Banded View} \label{sec-band-vw}
\index{banded view}
\index{view!banded}

\subsection{Definition}

\textit{WRITE}

\subsection{Relationship to other \stapl\ views}

\textit{WRITE}

\subsection{Implementation}

\textit{WRITE - details specific to this view}

\subsection{Interface} \label{sec-band-vw-inter}

\subsubsection{Constructors}

\textit{WRITE - fix Doxygen link and copy material here}

\subsubsection{ Element Manipulation}

\noindent
\texttt{%
view\_type
}
\newline
\textbf{operator()}%
\texttt{%
(View const \&
\textit{v}%
)
}

\begin{itemize}
\item
\textit{WRITE - fix Doxygen link and copy material here}
\end{itemize}

\subsection{Usage Example} \label{sec-band-vw-use}

The following example shows how to use the \texttt{banded\_view }:
%%\newline\vspace{0.4cm}\newline\rule{12cm}{0.5mm}
\input{view/band.tex}
%%\noindent\rule{12cm}{0.5mm}\vspace{0.5cm}

\subsection{Implementation} \label{sec-band-vw-impl}

\textit{WRITE}

\subsection{Performance} \label{sec-band-vw-perf}

\begin{itemize}
\item
Fig. \ref{fig:band-vw-assign-exper}
shows the performance of assigning values through a \stapl\ Banded view
whose elements are atomic values, \stl\ views, or \stapl\ views.
\item
Fig. \ref{fig:band-vw-access-exper}
shows the performance of accessing values through a \stapl\ Banded view
whose elements are atomic values, \stl\ views, or \stapl\ views.
\end{itemize}

\begin{figure}[p]
%%\includegraphics[scale=0.50]{figs/Banded_vw_assign}
\caption{Assign Values through Banded View Execution Time}
\label{fig:band-vw-assign-exper}
\end{figure}

\begin{figure}[p]
%%\includegraphics[scale=0.50]{figs/Banded_vw_access}
\caption{Access Values through Banded View Execution Time}
\label{fig:band-vw-access-exper}
\end{figure}

\emph{WRITE - complexity}

% % % % % % % % % % % % % % % % % % % % % % % % % % % % % % % % % % % % % % 

\section{Extended View} \label{sec-extend-vw}
\index{extended view}
\index{view!extended}

\subsection{Definition}

\textit{WRITE}

\subsection{Relationship to other \stapl\ views}

\textit{WRITE}

\subsection{Implementation}

\textit{WRITE - details specific to this view}

\subsection{Interface} \label{sec-ext-vw-inter}

\subsubsection{Constructors}

\textit{WRITE - fix Doxygen link and copy material here}

\subsubsection{ Element Manipulation}

\noindent
\texttt{%%
view\_type
}
\newline
\textbf{operator()}%
\texttt{%
(View const \&
\textit{v}%
)
}

\begin{itemize}
\item
\textit{WRITE - fix Doxygen link and copy material here}
\end{itemize}

\subsection{Usage Example} \label{sec-ext-vw-use}

The following example shows how to use the \texttt{extended\_view }:
%%\newline\vspace{0.4cm}\newline\rule{12cm}{0.5mm}
\input{view/extend.tex}
%%\noindent\rule{12cm}{0.5mm}\vspace{0.5cm}

\subsection{Implementation} \label{sec-ext-vw-impl}

\textit{WRITE}

\subsection{Performance} \label{sec-ext-vw-perf}

\begin{itemize}
\item
Fig. \ref{fig:ext-vw-assign-exper}
shows the performance of assigning values through a \stapl\ Extended view
whose elements are atomic values, \stl\ views, or \stapl\ views.
\item
Fig. \ref{fig:ext-vw-access-exper}
shows the performance of accessing values through a \stapl\ Extended view
whose elements are atomic values, \stl\ views, or \stapl\ views.
\end{itemize}

\begin{figure}[p]
%%\includegraphics[scale=0.50]{figs/Extended_vw_assign}
\caption{Assign Values through Extended View Execution Time}
\label{fig:ext-vw-assign-exper}
\end{figure}

\begin{figure}[p]
%%\includegraphics[scale=0.50]{figs/Extended_vw_access}
\caption{Access Values through Extended View Execution Time}
\label{fig:ext-vw-access-exper}
\end{figure}

\emph{WRITE - complexity}

% % % % % % % % % % % % % % % % % % % % % % % % % % % % % % % % % % % % % % 

\section{Slices View} \label{sec-slices-vw}
\index{slices view}
\index{view!slices}

\subsection{Definition}

\textit{WRITE}

\subsection{Relationship to other \stapl\ views}

\textit{WRITE}

\subsection{Implementation}

\textit{WRITE - details specific to this view}

\subsection{Interface} \label{sec-slc-vw-inter}

\subsubsection{Constructors}

\textit{WRITE-  fix Doxygen link and copy material here}

\subsubsection{ Element Manipulation}

\noindent
\texttt{%
view\_type
}
\newline
\textbf{operator()}%
\texttt{%
(View const \&
\textit{v}%
)
}

\begin{itemize}
\item
\textit{ WRITE - fix Doxygen and copy material here}
\end{itemize}

\subsection{Usage Example} \label{sec-slc-vw-use}

The following example shows how to use the \texttt{slices\_view}:
%%\newline\vspace{0.4cm}\newline\rule{12cm}{0.5mm}
\input{view/slices.tex}
%%\noindent\rule{12cm}{0.5mm}\vspace{0.5cm}

\subsection{Implementation} \label{sec-slc-vw-impl}

\textit{WRITE}

\subsection{Performance} \label{sec-slc-vw-perf}

\begin{itemize}
\item
Fig. \ref{fig:slc-vw-assign-exper}
shows the performance of assigning values through a \stapl\ Slices view
whose elements are atomic values, \stl\ views, or \stapl\ views.
\item
Fig. \ref{fig:slc-vw-access-exper}
shows the performance of accessing values through a \stapl\ Slices view
whose elements are atomic values, \stl\ views, or \stapl\ views.
\end{itemize}

\begin{figure}[p]
%%\includegraphics[scale=0.50]{figs/Slices_vw_assign}
\caption{Assign Values through Slices View Execution Time}
\label{fig:slc-vw-assign-exper}
\end{figure}

\begin{figure}[p]
%%\includegraphics[scale=0.50]{figs/Slices_vw_access}
\caption{Access Values through Slices View Execution Time}
\label{fig:slc-vw-access-exper}
\end{figure}

\emph{WRITE - complexity}

% % % % % % % % % % % % % % % % % % % % % % % % % % % % % % % % % % % % % % 

\section{Stencil View} \label{sec-stencil-vw}
\index{stencil view}
\index{view!stencil}

\subsection{Definition}

\textit{WRITE}

\subsection{Relationship to other \stapl\ views}

\textit{WRITE}

\subsection{Implementation}

\textit{WRITE - details specific to this view}

\subsection{Interface} \label{sec-sten-vw-inter}

\subsubsection{Constructors}

\textit{WRITE-  fix Doxygen link and copy material here}

\subsubsection{ Element Manipulation}

\noindent
\texttt{%
view\_type
}
\newline
\textbf{operator()}%
\texttt{%
(View const \&
\textit{v}%
)
}

\begin{itemize}
\item
\textit{WRITE -  fix Doxygen and copy material here }
\end{itemize}

\subsection{Usage Example} \label{sec-sten-vw-use}

The following example shows how to use the \texttt{stencil\_view}:
%%\newline\vspace{0.4cm}\newline\rule{12cm}{0.5mm}
\input{view/stencil.tex}
%%\noindent\rule{12cm}{0.5mm}\vspace{0.5cm}

\subsection{Implementation} \label{sec-sten-vw-impl}

\textit{WRITE}

\subsection{Performance} \label{sec-sten-vw-perf}

\begin{itemize}
\item
Fig. \ref{fig:sten-vw-assign-exper}
shows the performance of assigning values through a \stapl\ Stencil view
whose elements are atomic values, \stl\ views, or \stapl\ views.
\item
Fig. \ref{fig:sten-vw-access-exper}
shows the performance of accessing values through a \stapl\ Stencil view
whose elements are atomic values, \stl\ views, or \stapl\ views.
\end{itemize}

\begin{figure}[p]
%%\includegraphics[scale=0.50]{figs/Stencil_vw_assign}
\caption{Assign Values through Stencil View Execution Time}
\label{fig:sten-vw-assign-exper}
\end{figure}

\begin{figure}[p]
%%\includegraphics[scale=0.50]{figs/Stencil_vw_access}
\caption{Access Values through Stencil View Execution Time}
\label{fig:sten-vw-access-exper}
\end{figure}

\emph{WRITE - complexity}

% % % % % % % % % % % % % % % % % % % % % % % % % % % % % % % % % % % % % % 

\section{Native View} \label{sec-nat-vw}
\index{native view}
\index{view!native}

\subsection{Definition}

\textit{EXPAND - A view that creates a virtual container of views based on
the native distribution of the container.}

\subsection{Relationship to other \stapl\ views}

\textit{WRITE}

\subsection{Implementation}

\textit{WRITE - details specific to this view}

\subsection{Interface} \label{sec-nat-vw-inter}

\subsubsection{Constructors}

\textit{WRITE-  fix Doxygen link and copy material here}

\subsubsection{ Element Manipulation}

\noindent
\texttt{%
view\_type
}
\newline
\textbf{operator()}%
\texttt{%
(View const \&
\textit{v}%
)
}

\begin{itemize}
\item
\textit{WRITE -  fix Doxygen and copy material here}
\end{itemize}

\subsection{Usage Example} \label{sec-nat-vw-use}

The following example shows how to use the \texttt{ }:
%%\newline\vspace{0.4cm}\newline\rule{12cm}{0.5mm}
\input{view/native.tex}
%%\noindent\rule{12cm}{0.5mm}\vspace{0.5cm}

\subsection{Implementation} \label{sec-nat-vw-impl}

\textit{WRITE}

\subsection{Performance} \label{sec-nat-vw-perf}

\begin{itemize}
\item
Fig. \ref{fig:nat-vw-assign-exper}
shows the performance of assigning values through a \stapl\ Native view
whose elements are atomic values, \stl\ views, or \stapl\ views.
\item
Fig. \ref{fig:nat-vw-access-exper}
shows the performance of accessing values through a \stapl\ Native view
whose elements are atomic values, \stl\ views, or \stapl\ views.
\end{itemize}

\begin{figure}[p]
%%\includegraphics[scale=0.50]{figs/Native_vw_assign}
\caption{Assign Values through Native View Execution Time}
\label{fig:nat-vw-assign-exper}
\end{figure}

\begin{figure}[p]
%%\includegraphics[scale=0.50]{figs/Native_vw_access}
\caption{Access Values through Native View Execution Time}
\label{fig:nat-vw-access-exper}
\end{figure}

\emph{WRITE - complexity}

% % % % % % % % % % % % % % % % % % % % % % % % % % % % % % % % % % % % % % 

\section{Segmented View} \label{sec-seg-vw}
\index{segmenteded view}
\index{view!segmented}

\subsection{Definition}

\textit{EXPAND - A view that creates a virtual container of views based upon
a set of segment lengths.}

\subsection{Relationship to other \stapl\ views}

\textit{WRITE}

\subsection{Implementation}

\textit{WRITE - details specific to this view}

\subsection{Interface} \label{sec-seg-vw-inter}

\subsubsection{Constructors}

\noindent
\textbf{segmented\_view}%
\texttt{%
(view\_container\_type *
\textit{c,}%
domain\_type const \&
\textit{dom,}%
const map\_func\_type
\textit{mf=map\_func\_type()}%
)
}

\begin{itemize}
\item
Constructor used to pass ownership of the container to the view.
\end{itemize}

\noindent
\textbf{segmented\_view}%
\texttt{%
(view\_container\_type const \&
\textit{c,}%
domain\_type const \&
\textit{dom,}%
map\_func\_type const \&
\textit{mf=map\_func\_type(),}%
partitioned\_view const \&
\textit{other=partitioned\_view()}%
)
}

\begin{itemize}
\item
Constructor that does not takes ownership over the passed container.
\end{itemize}

\noindent
\textbf{segmented\_view}%
\texttt{%
(view\_container\_type const \&
\textit{c}%
)
}

\begin{itemize}
\item
Constructs a view that can reference all the elements of the passed container.
\end{itemize}

\noindent
\textbf{segmented\_view}%
\texttt{%
(C const \&
\textit{c,}%
partition\_type const \&
\textit{part,}%
map\_fun\_gen\_t
const \&
\textit{mfg=map\_fun\_gen\_t()}%
)
}

\begin{itemize}
\item
Constructs a partitioned\_view over the container c using the partition part.
\end{itemize}

\subsubsection{ View Attributes}

\noindent
\texttt{%
size\_type
}
\newline
\textbf{size}%
\texttt{%
(void) const
}

\begin{itemize}
\item
Returns the number of elements referenced for the view.
\end{itemize}

\noindent
\texttt{%
bool
}
\newline
\textbf{empty}%
\texttt{%
(void) const
}

\begin{itemize}
\item
Returns if the view does not reference any element.
\end{itemize}

\noindent
\texttt{%
value\_type
}
\newline
\textbf{get\_element}%
\texttt{%
(index\_t const \&
\textit{index}%
) const
}

\begin{itemize}
\item
Get the element index from the container.
\end{itemize}

\noindent
\texttt{%
template<class Functor >
}
\texttt{%
Functor::result\_type
}
\newline
\textbf{apply\_get}%
\texttt{%
(index\_t const \&
\textit{index,}%
Functor
\textit{f}%
)
}

\begin{itemize}
\item
Applies the provided function to the value referenced for the given index and returns the result of the operation.
\end{itemize}

\noindent
\texttt{%
reference
}
\newline
\textbf{operator[]}%
\texttt{%
(index\_t
\textit{index}%
) const
}

\begin{itemize}
\item
The bracket operator is the basic access method.
\end{itemize}

\noindent
\texttt{%
reference
}
\newline
\textbf{make\_reference}%
\texttt{%
(index\_t
\textit{index}%
) const
}

\subsection{Usage Example} \label{sec-seg-vw-use}

The following example shows how to use the \texttt{segmented view }:
%%\newline\vspace{0.4cm}\newline\rule{12cm}{0.5mm}
\input{view/segment.tex}
%%\noindent\rule{12cm}{0.5mm}\vspace{0.5cm}

\subsection{Implementation} \label{sec-seg-vw-impl}

\textit{WRITE}

\subsection{Performance} \label{sec-seg-vw-perf}

\begin{itemize}
\item
Fig. \ref{fig:seg-vw-assign-exper}
shows the performance of assigning values through a \stapl\ Segmented view
whose elements are atomic values, \stl\ views, or \stapl\ views.
\item
Fig. \ref{fig:seg-vw-access-exper}
shows the performance of accessing values through a \stapl\ Segmented view
whose elements are atomic values, \stl\ views, or \stapl\ views.
\end{itemize}

\begin{figure}[p]
%%\includegraphics[scale=0.50]{figs/Segmented_vw_assign}
\caption{Assign Values through Segmented View Execution Time}
\label{fig:seg-vw-assign-exper}
\end{figure}

\begin{figure}[p]
%%\includegraphics[scale=0.50]{figs/Segmented_vw_access}
\caption{Access Values through Segmented View Execution Time}
\label{fig:seg-vw-access-exper}
\end{figure}

\emph{WRITE - complexity}

% % % % % % % % % % % % % % % % % % % % % % % % % % % % % % % % % % % % % % 

\section{Zip View} \label{sec-zip-vw}
\index{zip view}
\index{view!zip}

\subsection{Definition}

\textit{WRITE}

\subsection{Relationship to other \stapl\ views}

\textit{WRITE}

\subsection{Implementation}

\textit{WRITE - details specific to this view}

\subsection{Interface} \label{sec-zip-vw-inter}

\subsubsection{Constructors}

\textit{WRITE-  fix Doxygen link and copy material here}

\subsubsection{View Attributes}

\textit{WRITE-  fix Doxygen link and copy material here}

\subsection{Usage Example} \label{sec-zip-vw-use}

The following example shows how to use the \texttt{zip view }:
%%\newline\vspace{0.4cm}\newline\rule{12cm}{0.5mm}
\input{view/zip.tex}
%%\noindent\rule{12cm}{0.5mm}\vspace{0.5cm}

\subsection{Implementation} \label{sec-zip-vw-impl}

\textit{WRITE}

\subsection{Performance} \label{sec-zip-vw-perf}

\begin{itemize}
\item
Fig. \ref{fig:zip-vw-assign-exper}
shows the performance of assigning values through a \stapl\ Zip view
whose elements are atomic values, \stl\ views, or \stapl\ views.
\item
Fig. \ref{fig:zip-vw-access-exper}
shows the performance of accessing values through a \stapl\ Zip view
whose elements are atomic values, \stl\ views, or \stapl\ views.
\end{itemize}

\begin{figure}[p]
%%\includegraphics[scale=0.50]{figs/Zip_vw_assign}
\caption{Assign Values through Zip View Execution Time}
\label{fig:zip-vw-assign-exper}
\end{figure}

\begin{figure}[p]
%%\includegraphics[scale=0.50]{figs/Zip_vw_access}
\caption{Access Values through Zip View Execution Time}
\label{fig:zip-vw-access-exper}
\end{figure}

\emph{WRITE - complexity}

% % % % % % % % % % % % % % % % % % % % % % % % % % % % % % % % % % % % % % 

\section{Cross View} \label{sec-cross-vw}
\index{cross view}
\index{view!cross}

\subsection{Definition}

\subsection{Relationship to other \stapl\ views}

\textit{WRITE}

\subsection{Implementation}

\textit{WRITE - details specific to this view}

\emph{WRITE}

\subsection{Interface} \label{sec-cross-vw-inter}

\subsubsection{Constructors}

\textit{WRITE-  fix Doxygen link and copy material here}

\subsubsection{View Attributes}

\textit{WRITE-  fix Doxygen link and copy material here}

\subsection{Usage Example} \label{sec-cross-vw-use}

The following example shows how to use the \texttt{cross view }:
%%\newline\vspace{0.4cm}\newline\rule{12cm}{0.5mm}
\input{view/cross.tex}
%%\noindent\rule{12cm}{0.5mm}\vspace{0.5cm}

\subsection{Implementation} \label{sec-cross-vw-impl}

\textit{WRITE}

\subsection{Performance} \label{sec-cross-vw-perf}

\begin{itemize}
\item
Fig. \ref{fig:cross-vw-assign-exper}
shows the performance of assigning values through a \stapl\ Cross view
whose elements are atomic values, \stl\ views, or \stapl\ views.
\item
Fig. \ref{fig:cross-vw-access-exper}
shows the performance of accessing values through a \stapl\ Cross view
whose elements are atomic values, \stl\ views, or \stapl\ views.
\end{itemize}

\begin{figure}[p]
%%\includegraphics[scale=0.50]{figs/Cross_vw_assign}
\caption{Assign Values through Cross View Execution Time}
\label{fig:cross-vw-assign-exper}
\end{figure}

\begin{figure}[p]
%%\includegraphics[scale=0.50]{figs/Cross_vw_access}
\caption{Access Values through Cross View Execution Time}
\label{fig:cross-vw-access-exper}
\end{figure}

\emph{WRITE - complexity}

% % % % % % % % % % % % % % % % % % % % % % % % % % % % % % % % % % % % % % 

\section{System View} \label{sec-dist-sys-vw}
\index{system view}
\index{view!system}

\subsection{Definition}

\textit{EXPAND - A view that provides a set of location ids on to which partition ids may be mapped.  Used to specify a data distribution.}

\subsection{Relationship to other \stapl\ views}

\textit{WRITE}

\subsection{Implementation}

\textit{WRITE - details specific to this view}

\subsection{Interface} \label{sec-dist-sys-vw-inter}

\subsubsection{Classes}
\noindent

\texttt{%
struct stapl::dist\_view\_impl::system\_container
}

\begin{itemize}
\item
Container that represents the set of locations in the system.
\end{itemize}

\noindent
\texttt{%
struct stapl::result\_of::system\_view
}

Defines the type of a system view.

\subsubsection{Functions}

\noindent
\texttt{%
result\_of::system\_view::type * stapl::system\_view
}
\texttt{%
    (location\_type
\textit{nlocs=get\_num\_locations()}%
)
}

\begin{itemize}
\item
Creates a read-only view representing the set of locations in the system. 
\end{itemize}

\subsection{Implementation} \label{sec-dist-sys-vw-impl}

\textit{WRITE}

\subsection{Usage Example} 

See \ref{sec-dist-part-vw-use} for an example of using this view.

% % % % % % % % % % % % % % % % % % % % % % % % % % % % % % % % % % % % % % 

\section{Mapping View} \label{sec-dist-map-vw}
\index{mapping view}
\index{view!mapping}

\subsection{Definition}

\textit{EXPAND - A view that maps partitions to location ids.  Used to specify a data distribution.}

\subsection{Relationship to other \stapl\ views}

\textit{WRITE}

\subsection{Implementation}

\textit{WRITE - details specific to this view}

\subsection{Interface} \label{sec-dist-map-vw-inter}

\subsubsection{Classes}

\noindent
\texttt{%
struct stapl::result\_of::mapping\_view< SysView,
PartitionIdDomain, MappingFunction >
}

\begin{itemize}
\item
Defines the type of a view that maps partition ids to locations.
\end{itemize}

\subsubsection{Functions}

\noindent
\texttt{%
template<typename SysView, typename PartitionIds, typename MappingFunction >
}
\texttt{%
result\_of::mapping\_view < SysView, PartitionIds, MappingFunction >::type
}
\newline
\textbf{stapl::mapping\_view}%
\texttt{%
(SysView const \&system, PartitionIds const \&domain,
    MappingFunction const \&
mapping\_func
)
}

\begin{itemize}
\item
Creates a read-only view whose domain is the identifiers of the partitions of an element partitioning and whose mapping function maps from partition id to location id.
\end{itemize}

\subsection{Usage Example} 

See \ref{sec-dist-part-vw-use} for an example of using this view.

\subsection{Implementation} \label{sec-dist-map-vw-impl}

\textit{WRITE}

% % % % % % % % % % % % % % % % % % % % % % % % % % % % % % % % % % % % % % 

\section{Partitioning View} \label{sec-dist-part-vw}
\index{partitioning view}
\index{view!partitioning}

\subsection{Definition}

\textit{EXPAND - A view that maps element ids to partition ids.
Used to specify a data distribution.}

\subsection{Relationship to other \stapl\ views}

\textit{WRITE}

\subsection{Implementation}

\textit{WRITE - details specific to this view}

\subsection{Interface} \label{sec-dist-part-vw-inter}

\subsubsection{Classes}

\noindent
\texttt{%
struct stapl::result\_of::partitioning\_view< MappingView,
GIDDomain, PartitioningFunction >
}

\begin{itemize}
\item
Defines the type of a view that partitions element GIDs to partition ids.
\end{itemize}

\subsubsection{Functions}

\noindent
\texttt{%
template<typename MappingView, typename GIDDomain, typename PartitioningFunction >
}
\texttt{%
result\_of::partitioning\_view < MappingView, GIDDomain, PartitioningFunction > ::type
}
\newline
\textbf{stapl::partitioning\_view}%
\texttt{%
(MappingView const \&
\textit{mapping\_view,}%
GIDDomain const \&
\textit{domain,}%
PartitioningFunction const \&
\textit{partitioning\_func}%
)
}

\begin{itemize}
\item
Creates a read-only view whose domain is the identifiers of a container's elements and whose mapping function maps from element GID to partition id.
\end{itemize}

\subsection{Usage Example} \label{sec-dist-part-vw-use}

The following example shows how to use the \texttt{partitioning view}:
%%\newline\vspace{0.4cm}\newline\rule{12cm}{0.5mm}
\input{view/partitioning.tex}
%%\noindent\rule{12cm}{0.5mm}\vspace{0.5cm}

\subsection{Implementation} \label{sec-dist-part-vw-impl}

\textit{WRITE}

% % % % % % % % % % % % % % % % % % % % % % % % % % % % % % % % % % % % % % 

\section{Distribution Specifications View} \label{sec-dist-spec-vw}
\index{distribution specifications view}
\index{view!distribution specifications}

\subsection{Definition} 

\textit{EXPAND - A view that provides a collection of functions that create view-based specifications of common data distributions.  Used to specify a data distribution.}

\subsection{Relationship to other \stapl\ views}

\textit{WRITE}

\subsection{Implementation}

\textit{WRITE - details specific to this view}

\subsection{Interface} \label{sec-dist-spec-vw-inter}

\subsubsection{Functions}

\noindent
\textbf{distribution\_spec stapl::block}%
\texttt{%
(unsigned long int
\textit{n,}%
unsigned long int
\textit{block\_size,}%
location\_type
\textit{num\_locs=get\_num\_locations()}%
)
}

\begin{itemize}
\item
Construct the specification of a blocked distribution.
\end{itemize}

\noindent
\textbf{distribution\_spec stapl::cyclic}%
\texttt{%
(unsigned long int
\textit{n,}%
location\_type
\textit{num\_locs=get\_num\_locations()}%
)
}

\begin{itemize}
\item
Construct the specification of a cyclic distribution.
\end{itemize}

\noindent
\textbf{distribution\_spec stapl::block\_cyclic}%
\texttt{%
(unsigned long int
\textit{n,}%
nsigned long int
\textit{block\_size,}%
location\_type
\textit{num\_locs=get\_num\_locations()}%
)
}

\begin{itemize}
\item
Construct the specification of a block-cyclic distribution.
\end{itemize}

\noindent
\textbf{distribution\_spec stapl::balance}%
\texttt{%
(unsigned long int
\textit{n,}%
location\_type
\textit{num\_locs=get\_num\_locations()}%
)
}

\begin{itemize}
\item
Construct the specification of a balanced distribution.
\end{itemize}

\noindent
\texttt{%
template<typename Traversal , typename Size >
}
\texttt{%
dist\_spec\_impl::md\_distribution\_spec < tuple\_size< Size >::value,
Traversal >::type
}
\newline
\textbf{stapl::volumetric}%
\texttt{%
(Size const \&
\textit{n,}%
location\_type
\textit{num\_locs=get\_num\_locations()}%
)
}

\begin{itemize}
\item
Construct the specification of a volumetric distribution of a multidimensional container.
\end{itemize}

\subsection{Usage Example} \label{sec-dist-spec-vw-use}

The following example shows how to use the \texttt{distribution specificationv iew}:
%%\newline\vspace{0.4cm}\newline\rule{12cm}{0.5mm}
The following example shows how to use the \texttt{distribution specificationv iew}:
%%\newline\vspace{0.4cm}\newline\rule{12cm}{0.5mm}
\input{view/distspec.tex}
%%\noindent\rule{12cm}{0.5mm}\vspace{0.5cm}

\subsection{Implementation} \label{sec-dist-spec-vw-impl}

\textit{WRITE}



% % % % % % % % % % % % % % % % % % % % % % % % % % % % % % % % % % % % %

\chapter{Parallel Algorithms}

STAPL parallel algorithms are organized into the following categories:

\begin{itemize}
\item
Non-modifying Sequence Operations : (Section \ref{sec-nonmod-alg} )
\newline
search and query view elements.

\begin{itemize}
\item
Search Operations API (Section \ref{sec-search-alg} )
\item
Summary Operations API (Section \ref{sec-sumry-alg} )
\item
Extrema Operations API (Section \ref{sec-extrem-alg} )
\item
Counting Operations API (Section \ref{sec-count-alg} )
\end{itemize}

\item
Mutating Sequence Operations : (Section \ref{sec-mutseq-alg} )
\newline
modify the elements in a view.

\begin{itemize}
\item
Mutating Sequence Operations API (Section \ref{sec-mutate-alg} )
\item
Removing Operations API (Section \ref{sec-remove-alg} )
\item
Reordering Operations API (Section \ref{sec-reord-alg} )
\end{itemize}

\item
Sorting and Related Operations : (Section \ref{sec-sorting-alg} )
\newline
sort elements in a view or perform operations on sorted sequences.

\begin{itemize}
\item
Sorting Operations API (Section \ref{sec-sort-alg} )
\item
Binary Search Operations API (Section \ref{sec-binsrch-alg} )
\item
Sorting Related Operations API (Section \ref{sec-sortrel-alg} )
\end{itemize}

\item
Generalized Numeric Algorithms : (Section \ref{sec-numer-alg} )
\newline
algorithms for numeric operations on view elements.

\end{itemize}

% % % % % % % % % % % % % % % % % % % % % % % % % % % % % % % % % % % % % % %

\section{Introduction}

A Parallel algorithm is the parallel counterpart of the STL algorithm. There are three types of Parallel algorithms in STAPL:

Parallel algorithms with semantics identical to their sequential counterparts.
Parallel algorithms with enhanced semantics (e.g. a parallel find could return any (or all) element found, while STL find only returns the first).
Parallel algorithms with no sequential equivalent in STL.

STL algorithms take iterators marking the start and end of an input sequence as parameters. Using STL constructs, such as the vector, this can be illustrated as follows:

\begin{verbatim}
std::vector<int> v(1000000);
// initialize v
std::find( v.begin(), v.end(), 0 );
\end{verbatim}

However, regular C++ arrays also support iterators, because iterators are in fact just generalized pointers:

\begin{verbatim}
int v[1000000];
... initialize v ...
find( &v[0], &v[1000000], 0 );
\end{verbatim}

STAPL Parallel algorithms take one or more pView instances as parameters instead. For example, STL provides an algorithm to find an element in a list, find. STAPL provides find which works with pViews. The construction of the pView over a container is an additional step, but the same pView instance can be used across multiple Parallel algorithm calls and allows additional flexibility such as providing access to a portion of the container instead of the entire data set.

\begin{verbatim}
stapl::vector<int> v(1000000);
stapl::vector_view<stapl::vector<int>> vw(v);
// initialize v
stapl::find( vw, 0 );
\end{verbatim}

In describing the parameters of these sets of Parallel algorithms, some conventions are used. All of the Parallel algorithms operate on sequences of input and/or output data (there are a few STL algorithms that only operate on a few elements, such as min or max, which are not parallel operations). STL generally describes this sequence using set notation as $[first, last)$, where first is an iterator to the start of a sequence and last is an iterator to the end of a sequence, and everything from the first element up to, but not including, the last element is considered part of the sequence. STAPL's pViews completely encapsulate this information. Hence, when describing a given Parallel algorithm, a sequence is represented as a pView.

Many Parallel algorithms behavior is described in terms of operator $?$, where $?$ is one of the C++ operators such as $<, >, ==$, etc. C++ allows the programmer to override the actions taken when one of the operators is called on a given class or type, and it may be helpful for the learning STAPL programmer to study this mechanism in C++. Another method that STL uses to change the default behavior of operators is to define Function Objects. These are functions or classes that implement operator() that will be used instead of the given operator. Most Parallel algorithms (and algorithms in STL), accept function objects, and in fact such flexibility lets both STL and STAPL algorithms to be adjusted to exactly what is needed, reducing the amount of code that a user needs to rewrite to obtain the desired effect.

All Parallel algorithms are expressed using dependence patterns, which when combined with the functor describing the operation on a single element and the set of pViews to process are using to instantiate the parallel task graph, PARAGRAPH. When the PARAGRAPH instance is executed using the executor and scheduler facilities of the STAPL runtime the desired parallel computation is performed. The scheduling policy can be specified for each PARAGRAPH instance if desired, otherwise the default FIFO policy is used. Multiple PARAGRAPHS may be processed concurrently by a PARAGRAPH executor that the set of locations executing the STAPL applications use to perform work.

% % % % % % % % % % % % % % % % % % % % % % % % % % % % % % % % % % % % % % %

\pagebreak

\section{Non-modifying Sequence Operations} \label{sec-nonmod-alg}

The non-modifying sequence operations do not directly modify the sequences of data they operate on. Each algorithm has two versions, one using operator $==$ for comparisons, and the other using a user-defined function object.

\subsection{Search Operations API} \label{sec-search-alg}

\begin{verbatim}
template<typename View0, typename View1, typename Pred >
View0::reference stapl::find_first_of (View0 const &view0,
                 View1 const &view1, Pred const &predicate)
\end{verbatim}

Finds the first element in the input which matches any of the elements in the given view, according to the given predicate.

\begin{verbatim}
template<typename View0, typename View1 >
View0::reference stapl::find_first_of (View0 const &view0, View1 const &view1)
\end{verbatim}

Finds the first element in the input which matches any of the elements in the given view.

\begin{verbatim}
template<typename View, typename Predicate >
View::reference stapl::find_if (View const &view, Predicate const &pred)
\end{verbatim}

Finds the first element in the input for which the predicate returns true, or NULL if none exist.

\begin{verbatim}
template<typename View, typename Predicate >
View::reference stapl::find_if_not (View const &view, Predicate const &pred)
\end{verbatim}

Finds the first element in the input for which the predicate returns false, or NULL if none exist.

\begin{verbatim}
template<typename View, typename T >
View::reference stapl::find (View const &view, T const &value)
\end{verbatim}

Finds the first occurrence of the given value in the input, or NULL if it is not found.

\begin{verbatim}
template<typename View, typename Pred >
View::reference stapl::partition_point (View const &pview, Pred predicate)
\end{verbatim}

Finds the position of the first element for which the functor returns false, indicating the partition point.

\begin{verbatim}
template<typename View1, typename View2, typename Predicate >
std::pair< typename View1::reference, typename View2::reference >
stapl::mismatch (View1 const &view1, View2 const &view2, Predicate pred)
\end{verbatim}

Given two input views, returns the positions of the first elements which do not match.

\begin{verbatim}
template<typename View1, typename View2 >
std::pair< typename View1::reference, typename View2::reference >
stapl::mismatch (View1 const &view1, View2 const &view2)
\end{verbatim}

Given two input views, returns the positions of the first elements which do not match.

\begin{verbatim}
template<typename View1, typename View2, typename Pred >
View1::reference stapl::find_end (const View1 &sequence, const View2 &pattern,
    Pred const &predicate)
\end{verbatim}

Finds the last occurrence of the given pattern in the input sequence.

\begin{verbatim}
template<typename View1, typename View2 >
View1::reference stapl::find_end (const View1 &sequence, const View2 &pattern)
\end{verbatim}

Finds the last occurrence of the given pattern in the input sequence.

\begin{verbatim}
template<typename View, typename BinPredicate >
View::reference stapl::adjacent_find (View view, BinPredicate bin_predicate)
\end{verbatim}

Return the position of the first adjacent pair of equal elements.

\begin{verbatim}
template<typename View >
View::reference stapl::adjacent_find (View view)
\end{verbatim}

Return the position of the first adjacent pair of equal elements.

\begin{verbatim}
template<class View1, class View2, class Predicate >
View1::reference stapl::search (View1 v1, View2 v2, Predicate pred)
\end{verbatim}

Return the position of the first occurrence of the given sequence within the input, or NULL if it is not found.

\begin{verbatim}
template<class View1, class View2 >
View1::reference stapl::search (View1 v1, View2 v2)
\end{verbatim}

Return the position of the first occurrence of the given sequence within the input, or NULL if it is not found.

\begin{verbatim}
template<class View1, class Predicate >
View1::reference stapl::search_n (View1 v1, size_t count,
    typename View1::value_type value, Predicate pred)
\end{verbatim}

Return the position of the first occurrence of a sequence of the given value which is of the given length, or NULL if none exists.

\begin{verbatim}
template<class View1 >
View1::reference stapl::search_n (View1 v1, size_t count,
    typename View1::value_type value)
\end{verbatim}

Return the position of the first occurrence of a sequence of the given value which is of the given length, or NULL if none exists.

% % % % % % % % % % % % % % % % % % % % % % % % % % % % % % % % % % % % % % %

\subsection{Summary Operations API} \label{sec-sumry-alg}

\begin{verbatim}
template<typename View, typename Predicate >
bool stapl::is_partitioned (View const &pview, Predicate predicate)
\end{verbatim}

Decides if the input view is partitioned according to the given functor, in that all elements which return true precede all those that do not.

\begin{verbatim}
template<typename View1, typename View2, typename Pred >
bool stapl::is_permutation (View1 const &view1, View2 const &view2, Pred pred)
\end{verbatim}

Computes whether all the elements in the first view are contained in the second view, even in a different order.

\begin{verbatim}
template<typename View1, typename View2 >
bool stapl::is_permutation (View1 const &view1, View2 const &view2)
\end{verbatim}

Computes whether all the elements in the first view are contained in the second view, even in a different order.

\begin{verbatim}
template<typename View0, typename View1, typename Predicate >
bool stapl::equal (View0 const &view0, View1 const &view1, Predicate pred)
\end{verbatim}

Compares the two input views and returns true if all of their elements compare pairwise equal.

\begin{verbatim}
template<typename View0, typename View1 >
bool stapl::equal (View0 const &view0, View1 const &view1)
\end{verbatim}

Compares the two input views and returns true if all of their elements compare pairwise equal.

% % % % % % % % % % % % % % % % % % % % % % % % % % % % % % % % % % % % % % %

\subsection{Extrema Operations API} \label{sec-extrem-alg}

\begin{verbatim}
template<typename View, typename Pred >
std::pair< typename View::value_type, typename View::value_type >
stapl::minmax_value (View const &view, Pred const &pred)
\end{verbatim}

Returns a pair containing the minimum and maximum values returned by the given predicate when called on all values in the input view.

\begin{verbatim}
template<typename View, typename Compare >
View::value_type stapl::min_value (View const &view, Compare comp)
\end{verbatim}

Finds the smallest value in the input view.

\begin{verbatim}
template<typename View >
View::value_type stapl::min_value (View const &view)
\end{verbatim}

Finds the smallest value in the input view.

\begin{verbatim}
template<typename View, typename Compare >
View::value_type stapl::max_value (View const &view, Compare comp)
\end{verbatim}

Finds the largest value in the input view.

\begin{verbatim}
template<typename View >
View::value_type stapl::max_value (View const &view)
\end{verbatim}

Finds the largest value in the input view.

\begin{verbatim}
template<typename View, typename Compare >
View::reference stapl::min_element (View const &view, Compare comp)
\end{verbatim}

Finds the smallest element in the input view (or the first smallest if there are multiple), which compares less than any other element using the given functor.

\begin{verbatim}
template<typename View >
View::reference stapl::min_element (View const &view)
\end{verbatim}

Finds the smallest element in the input view (or the first smallest if there are multiple).

\begin{verbatim}
template<typename View, typename Compare >
View::reference stapl::max_element (View const &view, Compare comp)
\end{verbatim}

Finds the largest element in the input view (or the first largest if there are multiple), which does not compare less than any other element using the given functor.

\begin{verbatim}
template<typename View >
View::reference stapl::max_element (View const &view)
\end{verbatim}

Finds the largest element in the input view (or the first largest if there are multiple).

% % % % % % % % % % % % % % % % % % % % % % % % % % % % % % % % % % % % % % %

\subsection{Counting Operations API} \label{sec-count-alg}

\begin{verbatim}
template<typename View0, typename Pred >
bool stapl::all_of (View0 const &view, Pred predicate)
\end{verbatim}

Returns true if the given predicate returns true for all of the elements in the input view.

\begin{verbatim}
template<typename View0, typename Pred >
bool stapl::none_of (View0 const &view, Pred predicate)
\end{verbatim}

Returns true if the given predicate returns false for all of the elements in the input view, or the view is empty.

\begin{verbatim}
template<typename View0, typename Pred >
bool stapl::any_of (View0 const &view, Pred predicate)
\end{verbatim}

Returns true if the given predicate returns true for any of the elements in the input view.

\begin{verbatim}
template<typename View, typename Predicate >
View::iterator::difference_type stapl::count_if (View const &view,
    Predicate pred)
\end{verbatim}

Computes the number of elements in the input view for which the given functor returns true.

\begin{verbatim}
template<typename View, typename T >
View::iterator::difference_type stapl::count (View const &view, T const &value)
\end{verbatim}

Computes the number of elements in the input view which compare equal to the given value.

\begin{verbatim}
template<typename View, typename T >
result_of::count< View, T >::type stapl::prototype::count (View const &view,
    T const &value)
\end{verbatim}

Computes the number of elements in the input view which compare equal to the given value.

% % % % % % % % % % % % % % % % % % % % % % % % % % % % % % % % % % % % % % %

\pagebreak

\section{Mutating Sequence Operations} \label{sec-mutseq-alg}

Mutating algorithms modify the sequences of data that they operate on in some way. The replace(), remove(), and unique() each have two versions, one using operator $==$ for comparisons, and the other using a function object.

\subsection{Mutating Sequence Operations API} \label{sec-mutate-alg}

\begin{verbatim}
template<typename View0, typename View1 >
void stapl::copy (View0 const &vw0, View1 const &vw1)
\end{verbatim}

Copy the elements of the input view to the output view.

\begin{verbatim}
template<typename View0, typename View1, typename Size >
void stapl::copy_n (View0 const &vw0, View1 const &vw1, Size n)
\end{verbatim}

Copy the first n elements from the input view to the output view.

\begin{verbatim}
template<typename View, typename Generator >
void stapl::generate (View const &view, Generator gen)
\end{verbatim}

Assign each value of the input view to the result of a successive call to the provided functor.

\begin{verbatim}
template<typename View, typename Generator >
void stapl::generate_n (View const &view, size_t first_elem, size_t n,
    Generator gen)
\end{verbatim}


Assign the n values of the input view starting at the given element to the result of a successive call to the provided functor.

\begin{verbatim}
template<typename View, typename Predicate >
void stapl::replace_if (View &vw, Predicate pred,
    typename View::value_type const &new_value)
\end{verbatim}

Replace the values from the input view for which the given predicate returns true with the new value.

\begin{verbatim}
template<typename View >
void stapl::replace (View &vw, typename View::value_type const &old_value,
    typename View::value_type const &new_value)
\end{verbatim}

Replace the given value in the input with the new value.

\begin{verbatim}
template<typename View0, typename View1, typename Predicate >
View1::iterator stapl::replace_copy_if (View0 const &vw0, View1 const &vw1,
    Predicate pred, typename View0::value_type new_value)
\end{verbatim}

Copy the values from the input view to the output, except for those elements for which the given predicate returns true, which are replaced with the given value.

\begin{verbatim}
template<typename View, typename Size >
void stapl::fill_n (View &vw, typename View::value_type value, Size n)
\end{verbatim}

Assigns the given value to the first n elements of the input view.

\begin{verbatim}
template<typename View >
void stapl::fill (View const &vw, typename View::value_type value)
\end{verbatim}

Assigns the given value to the elements of the input view.

\begin{verbatim}
template<typename View >
void stapl::swap_ranges (View &vw0, View &vw1)
\end{verbatim}

Swaps the elements of the two input views.

\begin{verbatim}
template<typename View0, typename View1 >
View1::iterator stapl::replace_copy (View0 &vw0, View1 &vw1,
    typename View0::value_type old_value, typename View0::value_type new_value)
\end{verbatim}

Copy the elements from the input to the output, replacing the given old\_value with the new\_value.

\begin{verbatim}
template<typename View0, typename Function >
Function stapl::for_each (const View0 &vw0, Function func)
\end{verbatim}

Applies the given functor to all of the elements in the input.

\begin{verbatim}
template<typename View0, typename View1, typename Function >
void stapl::transform (const View0 &vw0, const View1 &vw1, Function func)
\end{verbatim}

Applies the given function to the input, and stores the result in the output.

\begin{verbatim}
template<typename View0, typename View1, typename View2, typename Function >
void stapl::transform (View0 &vw0, View1 &vw1, View2 &vw2, Function func)
\end{verbatim}

Applies the given function to the inputs, and stores the result in the output.

\begin{verbatim}
template<typename View >
void stapl::prototype::fill (View &vw, typename View::value_type value)
\end{verbatim}

Assigns the given value to the elements of the input view.

\begin{verbatim}
template<typename View0, typename Function >
Function stapl::prototype::for_each (View0 const &vw0, Function func)
\end{verbatim}

Applies the given functor to all of the elements in the input.

\begin{verbatim}
template<typename View0 >
void stapl::iota (View0 const &view, typename View0::value_type const &value)
\end{verbatim}

Initializes the elements of the view such that the first element is assigned value, the next element value+1, etc.

% % % % % % % % % % % % % % % % % % % % % % % % % % % % % % % % % % % % % % %

\subsection{Removing Operations API} \label{sec-remove-alg}

\begin{verbatim}
template<typename View, typename Pred >
View stapl::keep_if (View const &pview, Pred predicate)
\end{verbatim}

Remove the values from the input view for which the given predicate returns false.

\begin{verbatim}
template<typename View, typename Pred >
View stapl::remove_if (View const &pview, Pred predicate)
\end{verbatim}

Remove the values from the input view for which the given predicate returns true.

\begin{verbatim}
template<typename View0, typename View1, typename Pred >
View1 stapl::copy_if (View0 const &vw0, View1 &vw1, Pred predicate)
\end{verbatim}

Copy the values from the input view to the output those elements for which the given predicate returns true.

\begin{verbatim}
template<typename View0, typename View1, typename Pred >
View1 stapl::remove_copy_if (View0 const &vw0, View1 &vw1, Pred predicate)
\end{verbatim}

Copy the values from the input view to the output, except for those elements for which the given predicate returns true.

\begin{verbatim}
template<typename View >
View stapl::remove (View &vw0, typename View::value_type valuetoremove)
\end{verbatim}

Remove the given value from the input.

\begin{verbatim}
template<typename View0, typename View1 >
View1 stapl::remove_copy (View0 const &vw0, View1 &vw1,
    typename View0::value_type valuetoremove)
\end{verbatim}

Copy the values from the input view to the output, except for those elements which are equal to the given value.

\begin{verbatim}
template<typename View, typename DestView, typename BinPredicate >
std::pair< DestView, DestView > stapl::unique_copy (View &src_view,
    DestView &dest_view, BinPredicate bin_predicate)
\end{verbatim}

Copies all of the elements from the source to the destination view, except those that are consecutive duplicates (equal to the preceding element), which are moved to the end of the destination view.

\begin{verbatim}
template<typename View0, typename View1 >
std::pair< View1, View1 > stapl::unique_copy (View0 src_view, View1 dest_view)
\end{verbatim}

Copies all of the elements from the source to the destination view, except those that are consecutive duplicates (equal to the preceding element), which are moved to the end of the destination view.

\begin{verbatim}
template<typename View, typename BinPredicate >
std::pair< View, View > stapl::unique (View view, BinPredicate bin_predicate)
\end{verbatim}

Remove all duplicate elements from the given view.

\begin{verbatim}
template<typename View >
std::pair< View, View > stapl::unique (View view)
\end{verbatim}

Remove all duplicate elements from the given view.

% % % % % % % % % % % % % % % % % % % % % % % % % % % % % % % % % % % % % % %

\subsection{Reordering Operations API} \label{sec-reord-alg}

\begin{verbatim}
template<typename View0, typename View1 >
void stapl::reverse_copy (View0 const &vw0, View1 const &vw1)
\end{verbatim}

Copy elements of the input view to the output view in reverse order.

\begin{verbatim}
template<typename View0 >
void stapl::reverse (View0 const &vw0)
\end{verbatim}

Reverse the order of the elements in the input view.

\begin{verbatim}
template<typename View0, typename View1 >
void stapl::rotate_copy (View0 const &vw0, View1 const &vw1, int k)
\end{verbatim}

Copy the elements in the input view to the output view rotated to the left by k positions.

\begin{verbatim}
template<typename View0 >
void stapl::rotate (View0 const &vw1, int k)
\end{verbatim}

Rotate the elements in the view to the left by k positions.

\begin{verbatim}
template<typename View, typename Pred >
size_t stapl::stable_partition (View const &pview, Pred predicate)
\end{verbatim}

Partition the input such that all elements for which the predicate returns true are ordered before those for which it returned false, while also maintaining the relative ordering of the elements.

\begin{verbatim}
template<typename View0, typename View1, typename View2, typename Pred >
std::pair< View1, View2 > stapl::partition_copy (View0 const &pview0,
    View1 const &pview1, View2 const &pview2, Pred predicate)
\end{verbatim}

Copies all elements from the input for which the functor returns true into the first output view, and all others into the second output.

\begin{verbatim}
template<typename View, typename Pred >
size_t stapl::partition (View const &pview, Pred predicate)
\end{verbatim}

Partition the input such that all elements for which the predicate returns true are ordered before those for which it returned false.

\begin{verbatim}
template<typename View, typename RandomNumberGenerator >
void stapl::random_shuffle (View const &view, RandomNumberGenerator const &rng)
\end{verbatim}

Computes a random shuffle of elements in the input view.

\begin{verbatim}
template<typename View >
void stapl::random_shuffle (View const &view)
\end{verbatim}

Computes a random shuffle of elements in the input view.

\begin{verbatim}
template<typename View, typename UniformRandomNumberGenerator >
void stapl::shuffle (const View &view, const UniformRandomNumberGenerator &rng)
\end{verbatim}

Computes a random shuffle of elements in the input view, using the given uniform random number generator.

% % % % % % % % % % % % % % % % % % % % % % % % % % % % % % % % % % % % % % %

\pagebreak

\section{Sorting and Related Operations} \label{sec-sorting-alg}

The sorting algorithms perform operations related to sorting or depending on sorted order. All algorithms define ordering of elements based on operator< or an optional StrictWeakOrdering function object.

\subsection{Sorting Operations API} \label{sec-sort-alg}

\begin{verbatim}
template<typename View, typename Compare >
void stapl::sample_sort (View &view, Compare comp, size_t sampling_method,
    size_t over_partitioning_ratio=1, size_t over_sampling_ratio=128)
\end{verbatim}

Sorts the elements of the input view according to the comparator provided using a sample-based approach.

\begin{verbatim}
template<typename View >
void stapl::sample_sort (View &view, size_t sampling_method,
    size_t over_partitioning_ratio=1, size_t over_sampling_ratio=128)
\end{verbatim}

Sorts the elements of the input view according to the comparator provided using a sample-based approach.

\begin{verbatim}
template<typename View, typename Compare >
void stapl::sample_sort (View &view, Compare comp)
\end{verbatim}

Sorts the elements of the input view according to the comparator provided using a sample-based approach.

\begin{verbatim}
template<typename View >
void stapl::sample_sort (View &view)
\end{verbatim}

Sorts the elements of the input view according to the comparator provided using a sample-based approach.

\begin{verbatim}
template<typename View, typename Comparator >
void stapl::sort (View &view, Comparator comp)
\end{verbatim}

Sorts the elements of the input view according to the comparator provided.

\begin{verbatim}
template<typename View >
void stapl::sort (View &view)
\end{verbatim}

Sorts the elements of the input view according to the comparator provided.

\begin{verbatim}
template<typename InputView, typename SplittersView, typename Compare,
    typename Functor >
segmented_view< InputView, splitter_partition<
    typename InputView::domain_type > > stapl::n_partition
    (InputView input_v, SplittersView splitters,
    Compare comp, Functor partition_functor)
\end{verbatim}

Reorders the elements in the input view in such a way that all elements for which the comparator returns true for a splitter s.

\begin{verbatim}
template<typename InputView, typename SplittersView, typename Compare >
segmented_view< InputView, splitter_partition< typename
    InputView::domain_type > >
stapl::n_partition (InputView input_v, SplittersView splitters, Compare comp)
\end{verbatim}

Reorders the elements in the input view in such a way that all elements for which the comparator returns true for a splitter s.

\begin{verbatim}
template<typename InputView, typename SplittersView >
segmented_view< InputView, splitter_partition< typename
     InputView::domain_type > >
stapl::n_partition (InputView input_v, SplittersView splitters)
\end{verbatim}

Reorders the elements in the input view in such a way that all elements for which the default '<' comparison function returns true for a splitter s - within the input splitters set - precede the elements for which the compare function returns false. The relative ordering of the elements is not preserved.

\begin{verbatim}
template<typename InputView, typename Compare, typename Functor >
segmented_view< InputView, splitter_partition< typename
    InputView::domain_type > >
stapl::n_partition (InputView input_v, std::vector< typename
    InputView::value_type > splitters, Compare comp, Functor partition_functor)
\end{verbatim}

Reorders the elements in the input view in such a way that all elements for which the comparator returns true for a splitter s.

\begin{verbatim}
template<typename InputView, typename Compare >
void stapl::partial_sort (InputView input_v, typename InputView::iterator nth,
    Compare comp)
\end{verbatim}

Performs a partial sort of the data in the input view using the comparator provided such that all elements before the nth position are sorted using the comparator.

\begin{verbatim}
template<typename InputView >
void stapl::partial_sort (InputView input_v, typename InputView::iterator nth)
\end{verbatim}

Performs a partial sort of the data in the input view using the comparator provided such that all elements before the nth position are sorted using the comparator.

\begin{verbatim}
template<typename InputView, typename OutputView, typename Compare >
OutputView stapl::partial_sort_copy (InputView input_v, OutputView output_v,
    Compare comp)
\end{verbatim}

Performs a partial sort of the input view data into the output view using the comparator provided such that all elements before the nth position are sorted using the comparator.

\begin{verbatim}
template<typename InputView, typename OutputView >
OutputView stapl::partial_sort_copy (InputView input_v, OutputView output_v)
\end{verbatim}

Performs a partial sort of the input view data into the output view using the comparator provided such that all elements before the nth position are sorted using the comparator.

\begin{verbatim}
template<typename InputView >
void stapl::radix_sort (InputView input_v)
\end{verbatim}

Sorts the element in the input view according to the radix-sort algorithm.

% % % % % % % % % % % % % % % % % % % % % % % % % % % % % % % % % % % % % % %

\subsection{Permuting Operations}

Reorder the elements in a view.

\begin{verbatim}
template<typename View, typename Predicate >
bool stapl::next_permutation (View &vw, Predicate pred)
\end{verbatim}

Computes the next lexicographic ordering of the input view (where the highest is sorted in decreasing order), or if input is already in highest order, places it in the lowest permutation (increasing order).

\begin{verbatim}
template<typename View >
bool stapl::next_permutation (View &vw)
\end{verbatim}

Computes the next lexicographic ordering of the input view (where the highest is sorted in decreasing order), or if input is already in highest order, places it in the lowest permutation (increasing order).

\begin{verbatim}
template<typename View, typename Predicate >
bool stapl::prev_permutation (View &vw, Predicate pred)
\end{verbatim}

Computes the previous lexicographic ordering of the input view (where the lowest is sorted in increasing order), or if input is already in lowest order, places it in the highest permutation (decreasing order).

\begin{verbatim}
template<typename View >
bool stapl::prev_permutation (View &vw)
\end{verbatim}

Computes the previous lexicographic ordering of the input view (where the lowest is sorted in increasing order), or if input is already in lowest order, places it in the highest permutation (decreasing order).

% % % % % % % % % % % % % % % % % % % % % % % % % % % % % % % % % % % % % % %

\subsection{Binary Search Operations API} \label{sec-binsrch-alg}

\begin{verbatim}
template<typename View, typename StrictWeakOrdering >
bool stapl::binary_search (View const &view, typename View::value_type value,
    StrictWeakOrdering comp)
\end{verbatim}

Searches the input view for the given value using a binary search, and returns true if that value exists in the input.

\begin{verbatim}
template<typename View >
bool stapl::binary_search (View const &view, typename View::value_type value)
\end{verbatim}

Searches the input view for the given value using a binary search, and returns true if that value exists in the input.

\begin{verbatim}
template<typename View, typename T, typename StrictWeakOrdering >
View::reference stapl::lower_bound (View const &view, T const &value,
    StrictWeakOrdering comp)
\end{verbatim}

Finds the first element in the input view which compares greater than or equal to the given value.

\begin{verbatim}
template<typename View, typename T >
View::reference stapl::lower_bound (View const &view, T const &value)
\end{verbatim}

Finds the first element in the input view which compares greater than or equal to the given value.

\begin{verbatim}
template<typename View, typename T, typename StrictWeakOrdering >
View::reference stapl::upper_bound (View const &view, T const &value,
    StrictWeakOrdering comp)
\end{verbatim}

Finds the first element in the input view which compares greater than the given value.

\begin{verbatim}
template<typename View, typename T >
View::reference stapl::upper_bound (View const &view, T const &value)
\end{verbatim}

Finds the first element in the input view which compares greater than the given value.

\begin{verbatim}
template<typename View, typename StrictWeakOrdering >
View stapl::equal_range (View const &view, typename View::value_type
    const &value, StrictWeakOrdering const &comp)
\end{verbatim}

Computes the range of elements which are equal to the given value.

\begin{verbatim}
template<typename View >
View stapl::equal_range (View const &view, typename View::value_type
    const &value)
\end{verbatim}

Computes the range of elements which are equal to the given value.

% % % % % % % % % % % % % % % % % % % % % % % % % % % % % % % % % % % % % % %

\subsection{Sorting Related Operations API} \label{sec-sortrel-alg}

\begin{verbatim}
template<typename View, typename View2, typename Pred >
bool stapl::lexicographical_compare (View const &pview1, View2 const &pview2,
    Pred const &pred)
\end{verbatim}

Determines if the first view is lexicographically less than the second view, using the given functor.

\begin{verbatim}
template<typename View, typename View2 >
bool stapl::lexicographical_compare (View const &pview1, View2 const &pview2)
\end{verbatim}

Determines if the first view is lexicographically less than the second view.

\begin{verbatim}
template<typename View1, typename View2, typename MergeView >
void stapl::merge (View1 const &view1, View2 const &view2, MergeView &merged)
\end{verbatim}

Merges the two sorted input views into the output view in sorted order.

\begin{verbatim}
template<typename View, typename Comp >
bool stapl::is_sorted (View const &view, Comp comp)
\end{verbatim}

Computes whether the input view is sorted.

\begin{verbatim}
template<typename View >
bool stapl::is_sorted (View const &view)
\end{verbatim}

Computes whether the input view is sorted.

\begin{verbatim}
template<typename View, typename Comp >
View stapl::is_sorted_until (View const &v, Comp const &c)
\end{verbatim}

Finds the range of elements in the input which are sorted.

\begin{verbatim}
template<typename View >
View stapl::is_sorted_until (View const &v)
\end{verbatim}

Finds the range of elements in the input which are sorted.

% % % % % % % % % % % % % % % % % % % % % % % % % % % % % % % % % % % % % % %

\pagebreak

\section{Generalized Numeric Algorithms API} \label{sec-numer-alg}

\begin{verbatim}
template<typename View, typename T, typename Oper >
T stapl::accumulate (View const &view, T init, Oper oper)
\end{verbatim}

Compute the sum of the elements and the initial value.

\begin{verbatim}
template<typename View, typename T >
T stapl::accumulate (View const &view, T init)
\end{verbatim}

Compute the sum of the elements and the initial value.

\begin{verbatim}
template<typename View1, typename View2, typename Oper >
void stapl::adjacent_difference (View1 const &view1, View2 const &view2,
    Oper oper)
\end{verbatim}

Assign each element of the output the difference between the corresponding input element and the input element that precedes it.

\begin{verbatim}
template<typename View1, typename View2 >
void stapl::adjacent_difference (View1 const &view1, View2 const &view2)
\end{verbatim}

Assign each element of the output the difference between the corresponding input element and the input element that precedes it.

\begin{verbatim}
template<typename View1, typename View2, typename Init, typename Sum,
    typename Product > Init
stapl::inner_product (View1 const &view1, View2 const &view2, Init init,
    Sum op1, Product op2)
\end{verbatim}

Compute the inner product of the elements of two input views. Inner product is defined as the sum of pair-wise products of the elements of two input views.

\begin{verbatim}
template<typename View1, typename View2, typename Init >
Init stapl::inner_product (View1 const &view1, View2 const &view2, Init init)
\end{verbatim}

Compute the inner product of the elements of two input views. Inner product is defined as the sum of pair-wise products of the elements of two input views.

\begin{verbatim}
template<typename T, typename ViewA, typename Views >
std::vector< T > stapl::inner_product (const ViewA &va, const Views &views,
    std::vector< T > init)
\end{verbatim}

Compute the inner product of the elements of one view and the elements of each of a set of views.

\begin{verbatim}
template<typename Init, typename View1, typename View2, typename View3,
     typename Sum, typename Product >
Init stapl::weighted_inner_product (View1 const &view1, View2 const &view2,
    View3 const &wt, Init init, Sum op1, Product op2)
\end{verbatim}

Compute a weighted inner product of the elements of two views, where each pair-wise product is multiplied by a weight factor before the sum of products is performed.

\begin{verbatim}
template<typename Init, typename View1, typename View2, typename View3 >
Init stapl::weighted_inner_product (View1 const &view1, View2 const &view2,
    View3 const &wt, Init init)
\end{verbatim}

Compute a weighted inner product of the elements of two views, where each pair-wise product is multiplied by a weight factor before the sum of products is performed.

\begin{verbatim}
template<typename View1, typename View2, typename Sum, typename Product >
View1::value_type stapl::weighted_norm (View1 const &view1, View2 const &wt,
    Sum op1, Product op2)
\end{verbatim}

Compute a weighted normal of the elements of a view.

\begin{verbatim}
template<typename View1, typename View2 >
View1::value_type stapl::weighted_norm (View1 const &view1, View2 const &wt)
\end{verbatim}

Compute a weighted normal of the elements of a view.

\begin{verbatim}
template<typename View0, typename View1, typename BinaryFunction >
void stapl::partial_sum (View0 const &view0, View1 const &view1,
    BinaryFunction binary_op, const bool shift)
\end{verbatim}

Computes the prefix sum of the elements of the input view and stores the result in the output view.

\begin{verbatim}
template<typename View0, typename View1 >
void stapl::partial_sum (View0 const &view0, View1 const &view1,
    const bool shift)
\end{verbatim}

Computes the prefix sum of the elements of the input view and stores the result in the output view.



% % % % % % % % % % % % % % % % % % % % % % % % % % % % % % % % % % % % %

\chapter{Parallel Input/Output}

\section{User Messages}

STAPL is a SPMD system.  So, every statement in a program
run using STAPL will be executed by every process.  This means typical
C++ input and output statements will perform in duplicate, unless something 
is done to avoid this.

The STAPL basic parallel algorithms ( \texttt{
map\_func, map\_reduce, scan, reduce, serial, serial\_io, do\_once ) }
distribute execution among processing locations, rather than duplicate it.
The last two constructs are used to perform I/O without duplication.

The \texttt{ do\_once }
is used to generate messages to the user.  It executes a work function one
time on location zero.  Define an appropriate work function as follows:

\begin{verbatim}
#include <iostream> 
#include <stapl/utility/do_once.hpp>
#include <stapl/runtime.hpp> 

template<typename Value>
struct msg_val {
private:
  const char *m_txt;
  Value m_val;
public:
  msg_val(const char *text, Value val)
    : m_txt(text), m_val(val)
  { }

  typedef void result_type;
  result_type operator() () {
    cout << m_txt << " " << m_val << endl;
  }
  void define_type(stapl::typer& t) {
    t.member(m_txt);
    t.member(m_val);
  }
};
\end{verbatim}

Execute the work function in your application as follows:

\begin{verbatim}
  stapl::do_once( msg_val<int>( "The answer is ", 42 ) );
\end{verbatim}

If you are familiar with the C++11 \texttt{lambda} feature, you can use it
with \texttt{do\_once} to generate user messages, without having to use
a separate work function.

\begin{verbatim}
  stapl::do_once([&](void) {
    cout << "The answer is " << 42 << endl;
  });
\end{verbatim}

% % % % % % % % % % % % % % % % % % % % % % % % % % % % % % % % % % % % % % % 

\section{Serial Data Values}

The 
\texttt{
serial\_io }
basic algorithm is used to apply a work function to the elements of
the input views in serial order.  The work function will only be executed
on location 0.

The STAPL \texttt{stream} class is used to encapsulate the C++ 
\texttt{iostream} file classes in a form that can be used with STAPL.

In order to read or write STAPL containers using \texttt{serial\_io},
we need a work function like the following.

\begin{verbatim}
class get_val_wf
{
private:
  stapl::stream<ifstream> m_zin;
public:
  get_val_wf(stapl::stream<ifstream> const& zin)
    : m_zin(zin)
  { }
  typedef void result_type;
  template <typename Ref>
  result_type operator()(Ref val) {
    m_zin >> val;
  }
  void define_type(stapl::typer& t) {
    t.member(m_zin);
  }
};
\end{verbatim}

The work function is used as follows.  First, we open the file, and then
we apply 

\begin{verbatim}
  stapl::vector<int> a_ct(size);
  stapl::vector_view<vector<int> > a_vw(a_ct);
  const char *file_name = "path";
  stapl::stream<ifstream> zin;
  zin.open(file_name);
  stapl::serial_io(get_val_wf(zin), a_vw);
\end{verbatim}

Performing output from a STAPL parallel container is done in a completely
analogous manner, substituting \texttt{ofstream} for \texttt{ifstream}, and
the insert operator $<<$ for the extract operator $>>$.

% % % % % % % % % % % % % % % % % % % % % % % % % % % % % % % % % % % % % % % 

\section{Parallel Data Values}

The STAPL graph class provides some special methods that make it possible
to perform parallel I/O on graphs stored in files.  One method can perform
\textit{sharding} on an input file.  This process splits a file into a set
of files, with a metadata file that describes the set as a whole.  These
files can then each be read independently by separate processing locations.
The sharder is called as follows;

\begin{verbatim}
  const char *file_name = "path";
  int block_size = 100;
  stapl::graph_sharder(file_name,block_size);
  stapl::rmi_fence();
\end{verbatim}

Sharding is possible for graphs because the edge list representation
that is stored on a file can be easily be made into self-describing parts.

A file containing a graph edge list that has been sharded can be read
in parallel with code that looks like the following

\begin{verbatim}
typedef stapl::graph<stapl::UNDIRECTED, stapl::NONMULTIEDGES, int, int>
        graf_int_tp;
typedef stapl::graph_view<graf_int_tp> graf_int_vw_tp;

graf_int_vw_tp gr_vw = stapl::sharded_graph_reader<graf_int_tp>
                       (filename, stapl::read_adj_list_line() );
\end{verbatim}

Future releases of STAPL may included sharding for representations of
other data structures which can be made self-identifying, such as sparse arrays.


% % % % % % % % % % % % % % % % % % % % % % % % % % % % % % % % % % % % %

\end{document}
