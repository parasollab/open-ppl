\chapter{Parallel Containers}

STAPL supports the following parallel containers:
\vspace{0.4cm}
\newline
Array (section \ref{sec-ary-cont}),
Static Array (section \ref{sec-stary-cont}),
\newline
Vector (section \ref{sec-vec-cont}),
List (section \ref{sec-list-cont}),
\newline
Static Graph (section \ref{sec-stgraf-cont}),
Dynamic Graph (section \ref{sec-dygraf-cont}),
\newline
Matrix (section \ref{sec-mat-cont}),
Multiarray (section \ref{sec-multi-cont}),
\newline
Set (section \ref{sec-set-cont}),
Map (section \ref{sec-map-cont}),
Unordered Map (section \ref{sec-unmap-cont})
\vspace{0.4cm}

A parallel container is the parallel equivalent of the STL container and is backward compatible with STL containers through its ability to provide iterators. Each parallel container provides (semi–) random access to its elements, a prerequisite for efficient parallel processing. Random access to the subsets of a parallel container's data is provided by an internal distribution maintained by the parallel container. The distribution is updated when elements are added or removed from the parallel container, and when migration of elements between locations is requested. The distribution has two primary components. 

The container manager maintains the subsets of elements stored on a location. Each subset is referred to as a bContainer. A parallel container instance may have more than one bContainer per location depending on the desired data distribution and independent migration of elements. The second component of the distribution is the directory, which enables any location to determine the location on which an element of the parallel container is stored.

\vspace{0.4cm}
\textbf{
STAPL provides many other methods for manipulating these 
containers, for the purposes of backwards compatibility.  
Today, there is no reason to use any methods other than the methods 
described in this chapter.
The purpose of a container is to be constructed and initialized with data.
All other operations should be performed with a view applied to the container.
}

% % % % % % % % % % % % % % % % % % % % % % % % % % % % % % % % % % % % % % % 

\pagebreak

\section{ Array Container API } \label{sec-ary-cont}

\emph{Parallel sequence container with fixed size. }

\begin{verbatim}
array (void)
\end{verbatim}

Create an array with size 0. Initially places the array in an unusable state. Should be used in conjunction with array::resize.
 
\begin{verbatim}
array (size_type n)
\end{verbatim}

Create an array with a given size and default construct all elements. 
 
\begin{verbatim}
array (size_type n, value_type const &default_value)
\end{verbatim}

Create an array with a given size and construct all elements with a default value. 
 
\begin{verbatim}
array (size_type n, mapper_type const &mapper)
\end{verbatim}

Create an array with a given size and instance of mapper. 
 
\begin{verbatim}
array (partition_type const &ps)
\end{verbatim}

Create an array with a given instance of partition. 
 
\begin{verbatim}
array (partition_type const &partitioner, mapper_type const &mapper)
\end{verbatim}

Create an array with a given partitioner and a mapper. 
 
\begin{verbatim}
template<typename DistSpecView >
array (DistSpecView const &dist_view, typename boost::enable_if< 
    is_distribution_view< DistSpecView > >::type *=0)
\end{verbatim}

Create an array with a distribution that is specified by the dist\_view provided. 
 
\begin{verbatim}
template<typename DistSpecView >
array (DistSpecView const &dist_view, value_type const &default_value, 
    typename boost::enable_if< is_distribution_view< DistSpecView > >::type *=0)
\end{verbatim}

Create an array with a distribution that is specified by the dist\_view provided, and initialize all elements to default\_value. 
 
\begin{verbatim}
template<typename DP >
array (size_type n, value_type const &default_value, DP const &dis_policy)
\end{verbatim}

Create an array with a given size and default value where the value type of the container is itself a parallel container. 
 
\begin{verbatim}
template<typename X , typename Y >
array (boost::tuples::cons< X, Y > dims)
\end{verbatim}

Create an array of arrays with given n-dimensional size. 
 
\begin{verbatim}
template<typename X , typename Y , typename DP >
array (boost::tuples::cons< X, Y > dims, DP const &dis_policy)
\end{verbatim}

Create an array of arrays with given n-dimensional size. 
 
\begin{verbatim}
template<typename SizesView >
array (SizesView const &sizes_view, typename boost::enable_if< 
    boost::mpl::and_< boost::is_same< size_type, typename 
    SizesView::size_type >, boost::mpl::not_< is_distribution_view< 
    SizesView > > > >::type *=0)
\end{verbatim}

Constructor for composed containers. For an m-level composed container, sizes\_view is an m-1 level composed view representing the sizes of the nested containers. 

% % % % % % % % % % % % % % % % % % % % % % % % % % % % % % % % % % % % % % % 

\section{Static Array Container API } \label{sec-stary-cont}

\emph{Parallel sequence container with fixed size. 
This container provides fast access, but has limits: it cannot be migrated and, the distribution is fixed to a balanced partition with one base container per location. }

\begin{verbatim}
static_array (size_type n)
\end{verbatim}

Create an array with a given size and default constructs all elements. 
 
\begin{verbatim}
static_array (size_type n, value_type const &default_value)
\end{verbatim}

Create an array with a given size and constructs all elements with a default value. 
 
\begin{verbatim}
template<typename DP >
static_array (size_type n, value_type const &default_value, 
    DP const &dis_policy)
\end{verbatim}

Create an array with a given size and default value where the value type of the container is itself a parallel container. 
 
\begin{verbatim}
template<typename X , typename Y >
static_array (boost::tuples::cons< X, Y > dims)
\end{verbatim}

Create an array of arrays with given n-dimensional size. 
 
\begin{verbatim}
template<typename X , typename Y , typename DP >
static_array (boost::tuples::cons< X, Y > dims, DP const &dis_policy)
\end{verbatim}

Create an array of arrays with given n-dimensional size. 

% % % % % % % % % % % % % % % % % % % % % % % % % % % % % % % % % % % % % % % 

\section{Vector Container API } \label{sec-vec-cont}

\emph{Parallel sequence container with element insertion and deletion methods. }

\begin{verbatim}
vector (void)
\end{verbatim}
 
\begin{verbatim}
vector (size_t n)
\end{verbatim}

Construct a vector with of size n. 
 
\begin{verbatim}
vector (size_t n, value_type const &default_value)
\end{verbatim}

Construct a vector with of size n filled with value default\_value. 
 
\begin{verbatim}
vector (partition_type const &ps)
\end{verbatim}

Construct a vector given a partitioner. 
 
\begin{verbatim}
vector (size_t n, mapper_type const &mapper)
\end{verbatim}

Construct a vector given a mapper. 
 
\begin{verbatim}
vector (partition_type const &partitioner, mapper_type const &mapper)
\end{verbatim}

Construct a vector given a partitioner and a mapper. 
 
\begin{verbatim}
template<typename DistSpecView >
vector (DistSpecView const &dist_view, typename boost::enable_if< 
    is_distribution_view< DistSpecView > >::type *=0)
\end{verbatim}

Create a vector with a distribution that is specified by the dist\_view provided. 
 
\begin{verbatim}
template<typename DistSpecView >
vector (DistSpecView const &dist_view, value_type const &default_value, 
    typename boost::enable_if< is_distribution_view< DistSpecView > >::type *=0)
\end{verbatim}

Create a vector with a distribution that is specified by the dist\_view provided, and initialize all elements to default\_value. 
 
\begin{verbatim}
template<typename DP >
vector (size_t n, value_type const &default_value, DP const &dis_policy)
\end{verbatim}

Create a vector with a given size and default value where the value type of the container is itself a parallel container. 
 
\begin{verbatim}
template<typename X , typename Y >
vector (boost::tuples::cons< X, Y > dims)
\end{verbatim}

Create a vector of vectors with given n-dimensional size. 
 
\begin{verbatim}
template<typename X , typename Y , typename DP >
vector (boost::tuples::cons< X, Y > dims, DP const &dis_policy)
\end{verbatim}

Create a vector of vectors with given n-dimensional size. 
 
\begin{verbatim}
template<typename SizesView >
vector (SizesView const &sizes_view, typename boost::enable_if< 
    boost::mpl::and_< boost::is_same< size_type, typename 
    SizesView::size_type >, boost::mpl::not_< is_distribution_view< 
    SizesView > > > >::type *=0)
\end{verbatim}

Constructor for composed containers. For an m level composed container, sizes\_view is an m-1 level composed view representing the sizes of the nested containers. 

% % % % % % % % % % % % % % % % % % % % % % % % % % % % % % % % % % % % % % % 

\section{ List Container API } \label{sec-list-cont}

\emph{ Parallel list container. }

\begin{verbatim}
list (void)
\end{verbatim}
 
\begin{verbatim}
list (size_t n)
\end{verbatim}

Constructs a list with n default constructed elements.
 
\begin{verbatim}
list (size_t n, value_type const &default_value)
\end{verbatim}

Constructs a list with a given size and default value. 
 
\begin{verbatim}
list (size_t n, mapper_type const &mapper)
\end{verbatim}

Construct a list given a mapper and a size parameter. 
 
\begin{verbatim}
list (partition_type const &ps)
\end{verbatim}

Constructs a list with a given partitioner. 
 
\begin{verbatim}
list (partition_type const &partitioner, mapper_type const &mapper)
\end{verbatim}

Constructs a list given a partitioner and a mapper. 
 
\begin{verbatim}
template<typename DistSpecView >
list (DistSpecView const &dist_view, typename boost::enable_if< 
    is_distribution_view< DistSpecView > >::type *=0)
\end{verbatim}
 
\begin{verbatim}
template<typename DistSpecView >
list (DistSpecView const &dist_view, value_type const &default_value, 
    typename boost::enable_if< is_distribution_view< DistSpecView > >::type *=0)
\end{verbatim}
 
\begin{verbatim}
list (list const &other)
\end{verbatim}

Copy constructs a list from another list container.
 
\begin{verbatim}
template<typename DP >
list (size_t n, value_type const &default_value, DP const &dis_policy)
\end{verbatim}

Constructs a list with a given size and default value where the value type of the container is itself a parallel container. 
 
\begin{verbatim}
template<typename X , typename Y >
list (boost::tuples::cons< X, Y > dims)
\end{verbatim}

Constructs a list of lists with given n-dimensional size. 
 
\begin{verbatim}
template<typename X , typename Y , typename DP >
list (boost::tuples::cons< X, Y > dims, DP const &dis_policy)
\end{verbatim}

Constructs a list of lists with given n-dimensional size. 

% % % % % % % % % % % % % % % % % % % % % % % % % % % % % % % % % % % % % % % 

\section{ Static Graph Container API } \label{sec-stgraf-cont}

\emph{ Parallel static graph container. Inherits all functionality from either undirected\_graph or directed\_graph.  Static graphs do not allow addition or deletion of vertices. The number of vertices must be known at construction. Edges may be added/deleted. Uses directedness selector to inherit from correct directed/undirected base.  }
 
\begin{verbatim}
graph (void)
\end{verbatim}

Creates an empty graph.
 
\begin{verbatim}
graph (size_t const &n)
\end{verbatim}

Creates a graph with a given size.
 
\begin{verbatim}
graph (size_t const &n, vertex_property const &default_value)
\end{verbatim}

Creates a graph with a given size and constructs all elements with a default value for vertex property.
 
\begin{verbatim}
graph (partition_type const &ps, vertex_property const 
    &default_value=vertex_property())
\end{verbatim}

Creates a graph with a given partition and default value for vertex property.
 
\begin{verbatim}
graph (partition_type const &ps, mapper_type const &m)
\end{verbatim}

Creates a graph with a given partition and mapper.
 
\begin{verbatim}
graph (partition_type const &ps, mapper_type const &m, vertex_property 
    const &default_value)
\end{verbatim}

Creates a graph with a given partition and mapper, with default value for vertex property.
 
\begin{verbatim}
graph (size_t const &n, mapper_type const &m, vertex_property const 
    &default_value=vertex_property())
\end{verbatim}

Creates a graph with a given size and mapper, with default value for vertex property. 
 
\begin{verbatim}
template<typename DistSpecsView >
graph (DistSpecsView const &dist_view, typename boost::enable_if< 
    is_distribution_view< DistSpecsView > >::type *=0)
\end{verbatim}
 
\begin{verbatim}
template<typename DistSpecsView >
graph (DistSpecsView const &dist_view, vertex_property const &default_value, 
    typename boost::enable_if< is_distribution_view< DistSpecsView > >::type *=0)
\end{verbatim}

 
\begin{verbatim}
template<typename DP >
graph (size_t n, vertex_property const &default_value, DP const &dis_policy)
\end{verbatim}

Creates a graph with a given size and default value where the vertex\_property may itself be a parallel container. Required for pC composition. 
 
\begin{verbatim}
template<typename DP >
graph (size_t n, DP const &dis_policy)
\end{verbatim}

Creates composed parallel containers with a given distribution policy. Required for pC composition. 
 
\begin{verbatim}
template<typename X , typename Y >
graph (boost::tuples::cons< X, Y > dims)
\end{verbatim}

Creates composed parallel containers with a given size-specifications. Required for pC composition. 
 
\begin{verbatim}
template<typename X , typename Y , typename DP >
graph (boost::tuples::cons< X, Y > dims, const DP &dis_policy)
\end{verbatim}

Creates composed parallel containers with a given size-specifications. Required for pC composition. 
 
\begin{verbatim}
template<typename SizesView >
graph (SizesView const &sizes_view, typename boost::enable_if< 
    boost::mpl::and_< boost::is_same< size_type, typename SizesView::size_type
    >, boost::mpl::not_< is_distribution_view< SizesView > > > >::type *=0)
\end{verbatim}

Constructor for composed containers. For an m-level composed container, sizes\_view is an m-1 level composed view representing the sizes of the nested containers. 

% % % % % % % % % % % % % % % % % % % % % % % % % % % % % % % % % % % % % % % 

\section{Dynamic Graph Container API } \label{sec-dygraf-cont}

\emph{ Dynamic graph that supports addition and deletion of vertices and edges.Inherits from stapl::graph and adds functionality to add/delete vertices. }

\begin{verbatim}
dynamic_graph (void)
\end{verbatim}
 
\begin{verbatim}
dynamic_graph (size_t const &n)
\end{verbatim}

Creates a graph with a given size. 
 
\begin{verbatim}
dynamic_graph (size_t const &n, VertexP const &default_value)
\end{verbatim}

Creates a graph with a given size and constructs all elements with a default value for vertex property. 
 
\begin{verbatim}
dynamic_graph (partition_type const &ps, VertexP const &default_value=VertexP())
\end{verbatim}

Creates a graph with a given partition and default value for vertex property. 
 
\begin{verbatim}
dynamic_graph (partition_type const &ps, mapper_type const &mapper, 
    VertexP const &default_value=VertexP())
\end{verbatim}

 
\begin{verbatim}
template<typename DistSpecsView >
dynamic_graph (DistSpecsView const &dist_view, typename boost::enable_if< 
    is_distribution_view< DistSpecsView > >::type *=0)
\end{verbatim}
 
\begin{verbatim}
template<typename DistSpecsView >
dynamic_graph (DistSpecsView const &dist_view, VertexP const &default_value, 
    typename boost::enable_if< is_distribution_view< DistSpecsView > 
    >::type *=0)
\end{verbatim}
 
\begin{verbatim}
template<typename DP >
dynamic_graph (size_t n, vertex_property const &default_value, 
    DP const &dis_policy)
\end{verbatim}

Creates a graph with a given size and default value where the vertex\_property may itself be a parallel container. Required for pC composition. 
 
\begin{verbatim}
template<typename X , typename Y >
dynamic_graph (boost::tuples::cons< X, Y > dims)
\end{verbatim}

Creates composed parallel containers with a given size-specifications. Required for pC composition. 
 
\begin{verbatim}
template<typename X , typename Y , typename DP >
dynamic_graph (boost::tuples::cons< X, Y > dims, DP const &dis_policy)
\end{verbatim}

Creates composed parallel containers with a given size-specifications. Required for pC composition. 
 
\begin{verbatim}
template<typename SizesView >
dynamic_graph (SizesView const &sizes_view, typename boost::enable_if< 
    boost::mpl::and_< boost::is_same< size_type, typename SizesView::size_type
    >, boost::mpl::not_< is_distribution_view< SizesView > > > >::type *=0)
\end{verbatim}

Constructor for composed containers. For an m-level composed container, sizes\_view is an m-1 level composed view representing the sizes of the nested containers. 

% % % % % % % % % % % % % % % % % % % % % % % % % % % % % % % % % % % % % % % 

\section { Matrix Container API } \label{sec-mat-cont}

\emph { Parallel dense matrix container.  }

\begin{verbatim}
matrix (size_type const &n, partition_type const &ps)
\end{verbatim}

Create an matrix with a given size and partition, default constructing all elements. 
 
\begin{verbatim}
matrix (size_type const &sizes)
\end{verbatim}

Create a matrix with a given size that default constructs all elements. 
 
\begin{verbatim}
matrix (size_type const &sizes, mapper_type const &mapper)
\end{verbatim}

Create a matrix with a given size and mapper. 
 
\begin{verbatim}
matrix (partition_type const &partitioner, mapper_type const &mapper)
\end{verbatim}

Create a matrix with a given mapper and partition. 
 
\begin{verbatim}
matrix (size_type const &sizes, value_type const &default_value)
\end{verbatim}

Create a matrix with a given size and default value default constructing all elements. 
 
\begin{verbatim}
matrix (partition_type const &ps)
\end{verbatim}

Create a matrix with a given partition. 
 
\begin{verbatim}
mapper_type (partition_type(part_dom_t(tuple_ops::transform(size_type(),[]
    (size_t const &){return 0;})), multiarray_impl::make_multiarray_size< N >
    ()(get_num_locations()))))
\end{verbatim}
 
\begin{verbatim}
multiarray_impl::make_multiarray_size ()(get_num_locations()))
\end{verbatim}

% % % % % % % % % % % % % % % % % % % % % % % % % % % % % % % % % % % % % % % 

\section{ Multiarray Container API  } \label{sec-multi-cont}

\emph{ Parallel multi-dimensional array container.  }

\begin{verbatim}
multiarray (void)
\end{verbatim}

Default constructor creates an empty multiarray with zero size in all dimensions.
 
\begin{verbatim}
multiarray_impl::make_multiarray_size ()(get_num_locations()))
\end{verbatim}
 
\begin{verbatim}
mapper_type (partition_type(part_dom_t(tuple_ops::transform(size_type(),
    [](size_t const &){return 0;})), multiarray_impl::make_multiarray_size
    < N >()(get_num_locations()))))
\end{verbatim}
 
\begin{verbatim}
multiarray (size_type const &sizes)
\end{verbatim}

Create an multiarray with a given size and default construct all elements. 
 
\begin{verbatim}
multiarray (size_type const &sizes, value_type const &default_value)
\end{verbatim}

Create a multiarray with given sizes in each dimension and a default value. 
 
\begin{verbatim}
multiarray (size_type const &sizes, mapper_type const &mapper)
\end{verbatim}

Create a multiarray with a given sizes in each dimension and a mapper. 
 
\begin{verbatim}
multiarray (partition_type const &ps)
\end{verbatim}

Create an multiarray with a given partition. 
 
\begin{verbatim}
multiarray (partition_type const &partitioner, mapper_type const &mapper)
\end{verbatim}

Create an multiarray with a given partitioner and mapper, default constructing all elements. 
 
\begin{verbatim}
template<typename DistSpecsView >
multiarray (DistSpecsView const &dist_view, typename boost::enable_if< 
    is_distribution_view< DistSpecsView > >::type *=0)
\end{verbatim}
 
\begin{verbatim}
template<typename SizesView >
multiarray (SizesView const &sizes_view, typename boost::enable_if< 
    boost::mpl::and_< boost::is_same< size_type, typename SizesView::size_type
    >, boost::mpl::not_< is_distribution_view< SizesView > > > >::type *=0)
\end{verbatim}

Constructor for composed containers. For an m level composed container, sizes\_view is an m-1 level composed view representing the sizes of the nested containers. 

% % % % % % % % % % % % % % % % % % % % % % % % % % % % % % % % % % % % % % % 

\section{ Set Container API } \label{sec-set-cont}

\emph{Parallel set container}

\begin{verbatim}
set (void)
\end{verbatim}

Construct a parallel set where the ownership of keys is determined by a simple block-cyclic distribution of the key space. 
 
\begin{verbatim}
set (typename Partitioner::value_type const &domain)
\end{verbatim}

Create a set with a given domain.
 
\begin{verbatim}
set (Partitioner const &partition)
\end{verbatim}

Construct a parallel set where the ownership of keys is determined by a given partition. 
 
\begin{verbatim}
set (Partitioner const &partitioner, Mapper const &mapper)
\end{verbatim}

Create a set with a given partitioner and mapper an instance of mapper. More

% % % % % % % % % % % % % % % % % % % % % % % % % % % % % % % % % % % % % % % 

\section{ Map Container API } \label{sec-map-cont}

\emph {Parallel ordered map container.}

\begin{verbatim}
map ()
\end{verbatim}

Construct a parallel map where the ownership of keys is determined by a simple block-cyclic distribution of the key space. 
 
\begin{verbatim}
map (typename partition_type::value_type const &domain)
\end{verbatim}

Construct a parallel map where the ownership of keys is determined by a balanced partition of a given domain. 
 
\begin{verbatim}
map (partition_type const &partition)
\end{verbatim}

Construct a parallel map where the ownership of keys is determined by a given partition. 
 
\begin{verbatim}
map (partition_type const &partitioner, mapper_type const &mapper)
\end{verbatim}

Construct a parallel map where the ownership of keys is determined by a given partition and mapper. 

% % % % % % % % % % % % % % % % % % % % % % % % % % % % % % % % % % % % % % % 

\section{ Unordered Map Container API } \label{sec-unmap-cont}

\emph {Parallel unordered map container.}

\begin{verbatim}
unordered_map (hasher const &hash=hasher(), key_equal const &comp=key_equal())
\end{verbatim}

Constructs an unordered map container. 
 
\begin{verbatim}
unordered_map (PS const &part, hasher const &hash=hasher(), key_equal 
    const &comp=key_equal())
\end{verbatim}

Constructs an unordered map container given a partition strategy. 
 
\begin{verbatim}
unordered_map (PS const &partitioner, M const &mapper, hasher const 
    &hash=hasher(), key_equal const &comp=key_equal())
\end{verbatim}

Constructs an unordered map container given a partition strategy and a mapper for the distribution. 
 
\begin{verbatim}
unordered_map (unordered_map const &other)
\end{verbatim}

