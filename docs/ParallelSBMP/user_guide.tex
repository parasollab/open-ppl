\documentclass[12pt]{article}
%\usepackage[top=0.5in,left=0.5in,right=0.5in, bottom=0.5in]{geometry}
\usepackage[margin=1.00in]{geometry}
\usepackage{amssymb}
\usepackage{algorithm,algorithmic}
\newcommand{\EXCISE}[1]{}

\title{\Large \bf User Guide for Parallel Sampling-Based Motion Planning}
\author{Sam Ade Jacobs}

\begin{document}
\maketitle

This user guide outlines the necessary steps require to compile and run parallel code.
It is a living document and expected to be modified as we make progress with parallel motion planning
research. The primary objective is to streamline the compilation and execution process for users.
A separate file (to be called developer's guide) will outline what developers need to do to write a 
STAPL-based parallel SBMP code. 


%%%%%%%%%%%%%%%%%%%%%%%%%%%%%%%%%%%%%%%%%%%%%%%%%%%%%%%%%%%%%%%%%%%%%%%%%%%%%%%%%%%%%%%%%%%%%%%%%%%
%% COMPILATION
%%%%%%%%%%%%%%%%%%%%%%%%%%%%%%%%%%%%%%%%%%%%%%%%%%%%%%%%%%%%%%%%%%%%%%%%%%%%%%%%%%%%%%%%%%%%%%%%%%%

\section{COMPILATION}
The parallel code heavily depends on STAPL framework for many of its components and STAPL also depends on
Boost and C++ STL. For this reason users need to have both BOOST and STL set up. Since STAPL is usually
ahead of PMPL in terms of compiler, STL and BOOST, it is important that the user have most recent versions of
these libraries and C++ compiler ( with OpenMPI/MPI wrapper). Here are the outlines of steps involved:
\begin{enumerate}
\item Run on the appropriate machine.  If you are developing for initial testing, then use any of Parasol Lab quad-core 64-bits machines (i.e. columbo, agate, newdelhi etc)
\item Load the appropriate compilers and libraries by using the module system as necessary.  Run {\tt module avail} to check available modules.
  \begin{itemize}
    \item {\tt load module gcc-4.6.2}
    \item {\tt load module boost-1.48}
    \item {\tt load module openmpi-x86\_64}
  \end{itemize}
\item Compile STAPL first with appropriate platform and STL specified
  \begin{itemize}
    \item {\tt cd utils/stapl\_release}
    \item {\tt gmake platform=LINUX\_gcc stl=4.6.2}
    \item {\tt cd ../../}
  \end{itemize}
\item Compile PMPL with parallel=1 and the appropriate STL LIB. This can either be set in the Makefile or done on the commandline:
  \begin{itemize}
    \item {\tt cd src}
    \item {\tt gmake platform=LINUX\_64\_gcc parallel=1 STL\_LIB=4.6.2 pmpl}
  \end{itemize}
\end{enumerate}


%%%%%%%%%%%%%%%%%%%%%%%%%%%%%%%%%%%%%%%%%%%%%%%%%%%%%%%%%%%%%%%%%%%%%%%%%%%%%%%%%%%%%%%%%%%%%%%%%%%
%% EXECUTION
%%%%%%%%%%%%%%%%%%%%%%%%%%%%%%%%%%%%%%%%%%%%%%%%%%%%%%%%%%%%%%%%%%%%%%%%%%%%%%%%%%%%%%%%%%%%%%%%%%%

\section{EXECUTION}
Having successfully compiled the code, the next step is to run:
\begin{enumerate}
\item Change to the environment directory you wish to run: {\tt cd TestRigid}
\item Adjust parameters in the xml input file (e.g., {\tt TestRigid/ParallelPMPLExample.xml})  (Note:this will file will be merged with PMPLExamples.xml soon)
\item Run via mpirun, specifing the number of processors p: {\tt mpirun -np p ../pmpl -f ParallelPMPLExample.xml}
\end{enumerate}
NOTE: Provided you compile in debug mode, totalview is a great debugger for parallel code, detail on how
to use this will be provided in developer's guide.

\end{document}
