\chapter{Distance Metrics}

A Distance Metric (\dm) is defined as a meaningful value representing the
feasibility of transitioning between two configurations $c_1$ and $c_2$. A \dm\
$\delta(c_1, c_2)$ should be non-negative. It should be zero when $c_1 \equiv
c_2$.

%%%%%%%%%%%%%%%%%%%%%%%%%%%%%%%%%%%%%%%%%%%%%%%%%%%%%%%%%%%%%%%%%%%%%%%%%%%%%%%%
%% Base Class
%%%%%%%%%%%%%%%%%%%%%%%%%%%%%%%%%%%%%%%%%%%%%%%%%%%%%%%%%%%%%%%%%%%%%%%%%%%%%%%%
\section{\texttt{DistanceMetricMethod}}

The base class, \texttt{DistanceMetricMethod}, has two important methods,
$\CALL{Distance}$ and $\CALL{ScaleCfg}$.

$\CALL{Distance}$ takes as input two configurations $c_1$ and $c_2$ and returns
the computed transition distance between them. It is called like the following:
\begin{lstlisting}
DistanceMetricPointer dm = this->GetMPProblem()->GetDistanceMetric(m_dmLabel);
CfgType c1, c2;
double dist = dm->Distance(c1, c2);
\end{lstlisting}
Where, $m\_dmLabel$ is an XML label, and $c1$ and $c2$ are configurations.

$\CALL{ScaleCfg}$ is purposed to scale a $d$-dimensional ray in $\cspace$ to a
certain magnitude based upon a general \dm. It is called like the following:
\begin{lstlisting}
DistanceMetricPointer dm = this->GetMPProblem()->GetDistanceMetric(m_dmLabel);
CfgType ray, origin;
double length;
dm->ScaleCfg(length, ray, origin);
\end{lstlisting}
Where, $m\_dmLabel$ is an XML label, $ray$ is the ray to be scaled, $length$ is
the desired magnitude, and $origin$ is the origin used to scale upon.

%%%%%%%%%%%%%%%%%%%%%%%%%%%%%%%%%%%%%%%%%%%%%%%%%%%%%%%%%%%%%%%%%%%%%%%%%%%%%%%%
%% Derived Class
%%%%%%%%%%%%%%%%%%%%%%%%%%%%%%%%%%%%%%%%%%%%%%%%%%%%%%%%%%%%%%%%%%%%%%%%%%%%%%%%
\section{Derived Classes}

%% Center Of Mass
\subsection{\texttt{CenterOfMassDistance}}

%% Knot Theory
\subsection{\texttt{KnotTheoryDistance}}

%% LP Swept
\subsection{\texttt{LPSweptDistance}}

%% Binary LP Swept
\subsubsection{\texttt{BinaryLPSweptDistance}}

%% Minkowski
\subsection{\texttt{MinkowskiDistance}}

%% Euclidean
\subsubsection{\texttt{EuclideanDistance}}

%% Scaled Euclidean
\paragraph{\texttt{ScaledEuclideanDistance}}

%% Manhattan
\subsubsection{\texttt{ManhattanDistance}}

%% Reachable
\subsection{\texttt{ReachableDistance}}

%% RMSD
\subsection{\texttt{RMSDDistance}}

