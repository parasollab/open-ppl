\chapter{Core PMPL Classes}

This chapter explains some of the core concepts within PMPL, which are used
throughout the library of planning abstractions and algorithms, but are
consistent through all of them, e.g., $\cspace$, work space, code insights, etc.

%%%%%%%%%%%%%%%%%%%%%%%%%%%%%%%%%%%%%%%%%%%%%%%%%%%%%%%%%%%%%%%%%%%%%%%%%%%%%%%%
%% Cfg Classes
%%%%%%%%%%%%%%%%%%%%%%%%%%%%%%%%%%%%%%%%%%%%%%%%%%%%%%%%%%%%%%%%%%%%%%%%%%%%%%%%
\section{\texttt{Cfg}}

\texttt{Cfg} is the core class which defines a configuration, or a vector of
values representing all the degrees of freedom of a robot. This is the
abstraction of $\cspace$ essentially, and thus \texttt{Cfg} is a point or vector
inside of $\cspace$. Most mathematical operations are defined over this class,
reading, writing, accessing, random sampling, etc.

%% Cfg Reach
\subsection{\texttt{Cfg\_reach\_cc}}

%% Cfg Reach Fixed
\subsubsection{\texttt{Cfg\_reach\_cc\_fixed}}

%% Virtual Link
\subsubsection{\texttt{VirtualLink}}

%% Cfg Surface
\subsection{\texttt{CfgSurface}}

%% SSSurface
\subsubsection{\texttt{SSSurface}}

%% SSSurface Mult
\paragraph{\texttt{SSSurfaceMult}}

%%%%%%%%%%%%%%%%%%%%%%%%%%%%%%%%%%%%%%%%%%%%%%%%%%%%%%%%%%%%%%%%%%%%%%%%%%%%%%%%
%% Weight Classes
%%%%%%%%%%%%%%%%%%%%%%%%%%%%%%%%%%%%%%%%%%%%%%%%%%%%%%%%%%%%%%%%%%%%%%%%%%%%%%%%
\section{\texttt{Weight}}

\texttt{Weight} is the concept for what is stored on the graph edges.
Essentially, edges are defined as polygonal chains through $\cspace$. They have
two essential properties, a weight value representing some idea of distance
between the two end points and a set of intermediate configurations along the
polygonal chain not including the start and goal configurations.

%%%%%%%%%%%%%%%%%%%%%%%%%%%%%%%%%%%%%%%%%%%%%%%%%%%%%%%%%%%%%%%%%%%%%%%%%%%%%%%%
%% MPProblem Classes
%%%%%%%%%%%%%%%%%%%%%%%%%%%%%%%%%%%%%%%%%%%%%%%%%%%%%%%%%%%%%%%%%%%%%%%%%%%%%%%%
\section{\texttt{MPProblem}}

The \texttt{MPProblem} is a central class to PMPL. It ``stores everything''. It
is a hub of accessing all algorithmic abstractions, roadmaps, statistics
classes, the environment, etc.

%% Closed Chain Problem
\subsection{\texttt{ClosedChainProblem}}

%%%%%%%%%%%%%%%%%%%%%%%%%%%%%%%%%%%%%%%%%%%%%%%%%%%%%%%%%%%%%%%%%%%%%%%%%%%%%%%%
%% MPTraits Classes
%%%%%%%%%%%%%%%%%%%%%%%%%%%%%%%%%%%%%%%%%%%%%%%%%%%%%%%%%%%%%%%%%%%%%%%%%%%%%%%%
\section{\texttt{MPTraits}}

\texttt{MPTraits} is a type class which defines the motion planning universe. We
construct our methods through a factory design pattern, and thus this states all
available classes within an abstraction that you can use in the system.
Essentially the important types are, the \texttt{CfgType} or the $\cspace$
abstraction class, the \texttt{WeightType} or the edge type of the graph, and a
type list for each algorithm abstraction --- here you only need to define what
you need, as extraneous methods in the type class imply longer compile times.

%%%%%%%%%%%%%%%%%%%%%%%%%%%%%%%%%%%%%%%%%%%%%%%%%%%%%%%%%%%%%%%%%%%%%%%%%%%%%%%%
%% Environment Classes
%%%%%%%%%%%%%%%%%%%%%%%%%%%%%%%%%%%%%%%%%%%%%%%%%%%%%%%%%%%%%%%%%%%%%%%%%%%%%%%%
\section{\texttt{Environment}}

The \texttt{Environment} is essentially the work space of the motion planning
problem. We define a workspace as a set of \texttt{MultiBody} which are
essentially either robot or obstacle geometries and a \texttt{Boundary} to
define the sampling/exploration region for the planner.

%% Boundary
\subsection{\texttt{Boundary}}

%% Bounding Box
\subsubsection{\texttt{BoundingBox}}

%% Bounding Sphere
\subsubsection{\texttt{BoundingSphere}}

%% Robot
\subsection{\texttt{Robot}}

%% MultiBody
\subsection{\texttt{MultiBody}}

%% Body
\subsubsection{\texttt{Body}}

%% Fixed Body
\paragraph{\texttt{FixedBody}}

%% Free Body
\paragraph{\texttt{FreeBody}}

%% GMS Polyhedron
\paragraph{\texttt{GMSPolyhedron}}

%% Connection
\subsubsection{\texttt{Connection}}

%% DH Parameters
\paragraph{\texttt{DHparameters}}

%% Contact
\subsection{\texttt{Contact}}

%%%%%%%%%%%%%%%%%%%%%%%%%%%%%%%%%%%%%%%%%%%%%%%%%%%%%%%%%%%%%%%%%%%%%%%%%%%%%%%%
%% Roadmap Classes
%%%%%%%%%%%%%%%%%%%%%%%%%%%%%%%%%%%%%%%%%%%%%%%%%%%%%%%%%%%%%%%%%%%%%%%%%%%%%%%%
\section{\texttt{Roadmap}}

The \texttt{Roadmap} is essentially the graph $G=(V, E)$ which is used to
approximate the planning space used in sampling-based motion planning
algorithms. $V$, or the set of vertices, are all of type \texttt{CfgType} and
$E$, or the set of edges, are all of type \texttt{WeightType}.

%% Roadmap Graph
\subsection{\texttt{RoadmapGraph}}

%% Roadmap VCS
\subsubsection{\texttt{RoadmapVCS}}

%% Roadmap Change Event
\subsubsection{\texttt{RoadmapChangeEvent}}

%%%%%%%%%%%%%%%%%%%%%%%%%%%%%%%%%%%%%%%%%%%%%%%%%%%%%%%%%%%%%%%%%%%%%%%%%%%%%%%%
%% Miscellaneous
%%%%%%%%%%%%%%%%%%%%%%%%%%%%%%%%%%%%%%%%%%%%%%%%%%%%%%%%%%%%%%%%%%%%%%%%%%%%%%%%
\section{Miscellaneous Classes}

%% Generate Partitions
\subsection{\texttt{GeneratePartitions}}

%% Is Closed Chain
\subsection{\texttt{IsClosedChain}}

%% Region Graph
\subsection{\texttt{RegionGraph}}

%% SRT Info
\subsection{\texttt{SRTInfo}}

