\chapter{Connectors}

Connectors (\c) orchestrate the creation of edges within a sampling-based
planner. More loosely, the combine a neighbor selection phase to determine
candidate edges and validate these with a local planner. If a local planner
succeeds then the edge is added to the roadmap. These algorithms vary in the
ways they do this.

%%%%%%%%%%%%%%%%%%%%%%%%%%%%%%%%%%%%%%%%%%%%%%%%%%%%%%%%%%%%%%%%%%%%%%%%%%%%%%%%
%% Base Class
%%%%%%%%%%%%%%%%%%%%%%%%%%%%%%%%%%%%%%%%%%%%%%%%%%%%%%%%%%%%%%%%%%%%%%%%%%%%%%%%
\section{\texttt{ConnectorMethod}}

\texttt{ConnectorMethod} essentially has one important function,
$\CALL{Connect}$ which can be called in a multitude of ways. In its basic forms
it takes two sets of configurations and generates edges in the roadmap between
them. Usage:
\begin{lstlisting}
ConnectorPointer c = this->GetMPProblem()->GetConnector(m_cLabel);
ColorMapType cm;
vector<VID> c1, c2;
c->Connect(this->GetMPProblem()->GetRoadmap(),
           this->GetMPProblem()->GetStatClass(),
           cmap, c1.begin(), c1.end(), c2.begin(), c2.end());
\end{lstlisting}
Where ColorMapType is the type of a color map for the underlying STAPL graph.

%%%%%%%%%%%%%%%%%%%%%%%%%%%%%%%%%%%%%%%%%%%%%%%%%%%%%%%%%%%%%%%%%%%%%%%%%%%%%%%%
%% Derived Class
%%%%%%%%%%%%%%%%%%%%%%%%%%%%%%%%%%%%%%%%%%%%%%%%%%%%%%%%%%%%%%%%%%%%%%%%%%%%%%%%
\section{Derived Classes}

%% Adaptive
\subsection{\texttt{AdaptiveConnector}}

%% CC Expansion
\subsection{\texttt{CCExpansion}}

%% CCs
\subsection{\texttt{CCsConnector}}

%% Closest VE
\subsection{\texttt{ClosestVE}}

%% Connect Neighboring Surfaces
\subsection{\texttt{ConnectNeighboringSurfaces}}

%% Neighborhood Connector
\subsection{\texttt{NeighborhoodConnector}}

%% Rewire
\subsection{\texttt{RewireConnector}}

%%%%%%%%%%%%%%%%%%%%%%%%%%%%%%%%%%%%%%%%%%%%%%%%%%%%%%%%%%%%%%%%%%%%%%%%%%%%%%%%
%% Old
%%%%%%%%%%%%%%%%%%%%%%%%%%%%%%%%%%%%%%%%%%%%%%%%%%%%%%%%%%%%%%%%%%%%%%%%%%%%%%%%
\section{Old Classes}

%% Preferential Attachment
\subsection{\texttt{PreferentialAttachment}}

%% RRT Components
\subsection{\texttt{RRTcomponents}}

%% RRT Expand
\subsection{\texttt{RRTexpand}}

%% Ray Tracer
\subsection{\texttt{RayTracer}}

