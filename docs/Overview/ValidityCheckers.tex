\chapter{Validity Checkers}

A Validity Checker (\vc) is defined as a black box unit which returns whether or
not a configurations is within $\cfree$ or $\cobst$. This is most commonly
defined as collision detection, but could be more widely seen as anything
which defines $\cobst$, i.e., energy-feasible.

%%%%%%%%%%%%%%%%%%%%%%%%%%%%%%%%%%%%%%%%%%%%%%%%%%%%%%%%%%%%%%%%%%%%%%%%%%%%%%%%
%% Base Class
%%%%%%%%%%%%%%%%%%%%%%%%%%%%%%%%%%%%%%%%%%%%%%%%%%%%%%%%%%%%%%%%%%%%%%%%%%%%%%%%
\section{\texttt{ValidityCheckerMethod}}

The base class, \texttt{ValidityCheckerMethod}, has two important methods,
$\CALL{IsValid}$ and $\CALL{IsInsideObstacle}$.

$\CALL{IsValid}$ takes as input a configuration $c$ and a \texttt{CDInfo} and
returns whether or not $c$ is within $\cfree$. It is called like the following:
\begin{lstlisting}
ValidityCheckerPointer vc = this->GetMPProblem()->GetValidityChecker(m_vcLabel);
CfgType c;
CDInfo cdInfo;
string callee("StandardsDoc");
bool valid = vc->IsValid(c, cdInfo, callee);
\end{lstlisting}
Where $callee$ is a string representing the calling function of
$\CALL{IsValid}$.

$\CALL{IsInsideObstacle}$ is meant mostly for medial axis related functions, but
it takes as input a configuration $c$ and determines whether the robot
configured at $c$ lies entirely within an obstacle. It is called like the
following:
\begin{lstlisting}
ValidityCheckerPointer vc = this->GetMPProblem()->GetValidityChecker(m_vcLabel);
CfgType c;
bool valid = vc->IsInsideObstacle(c);
\end{lstlisting}

%%%%%%%%%%%%%%%%%%%%%%%%%%%%%%%%%%%%%%%%%%%%%%%%%%%%%%%%%%%%%%%%%%%%%%%%%%%%%%%%
%% Helper Classes
%%%%%%%%%%%%%%%%%%%%%%%%%%%%%%%%%%%%%%%%%%%%%%%%%%%%%%%%%%%%%%%%%%%%%%%%%%%%%%%%
\section{Helper Classes}

\subsection{\texttt{ValidityCheckerFunctor}}

%%%%%%%%%%%%%%%%%%%%%%%%%%%%%%%%%%%%%%%%%%%%%%%%%%%%%%%%%%%%%%%%%%%%%%%%%%%%%%%%
%% Derived Class
%%%%%%%%%%%%%%%%%%%%%%%%%%%%%%%%%%%%%%%%%%%%%%%%%%%%%%%%%%%%%%%%%%%%%%%%%%%%%%%%
\section{Derived Classes}

%% Always True
\subsection{\texttt{AlwaysTrueValidity}}

%% Collision Detection
\subsection{\texttt{CollisionDetectionValidity}}

%% Compose
\subsection{\texttt{ComposeValidity}}

%% Medial Axis Clearance
\subsection{\texttt{MedialAxisClearanceValidity}}

%% Negate
\subsection{\texttt{NegateValidity}}

%% Node Clearance
\subsection{\texttt{NodeClearanceValidity}}

%% Obstacle Clearance
\subsection{\texttt{ObstacleClearanceValidity}}

%% SS Surface
\subsection{\texttt{SSSurfaceValidity}}

%% Surface
\subsection{\texttt{SurfaceValidity}}

%%%%%%%%%%%%%%%%%%%%%%%%%%%%%%%%%%%%%%%%%%%%%%%%%%%%%%%%%%%%%%%%%%%%%%%%%%%%%%%%
%% Base Class
%%%%%%%%%%%%%%%%%%%%%%%%%%%%%%%%%%%%%%%%%%%%%%%%%%%%%%%%%%%%%%%%%%%%%%%%%%%%%%%%
\section{\texttt{CollisionDetectionMethod}}

Collision Detection (\cd) is an important compontent of a robotics library. It
is one of the fundamental validity types in PMPL. It takes an input a
configuration $c$ and returns whether or not the robot configured at $c$ is in
collision with any obstacles. \texttt{CollisionDetectionMethod} is the core
class of these geometric collision detection libraries. Mostly these serve as
middleware to interface with the external libraries. \cd\ methods are not
directly accessed but have two core functions, \CALL{IsInCollision} and
\CALL{IsInsideObstacle}. \CALL{IsInCollision} takes as input a
\texttt{MultiBody} for a robot and an obstacle and returns whether the robot and
obstacle collide. \CALL{IsInsideObstacle} takes a configuration $c$ and
determines whether the robot configured at $c$ lies entirely within a workspace
obstacle.

%%%%%%%%%%%%%%%%%%%%%%%%%%%%%%%%%%%%%%%%%%%%%%%%%%%%%%%%%%%%%%%%%%%%%%%%%%%%%%%%
%% Helper Classes
%%%%%%%%%%%%%%%%%%%%%%%%%%%%%%%%%%%%%%%%%%%%%%%%%%%%%%%%%%%%%%%%%%%%%%%%%%%%%%%%
\section{Helper Classes}

\subsection{\texttt{CDInfo}}

%%%%%%%%%%%%%%%%%%%%%%%%%%%%%%%%%%%%%%%%%%%%%%%%%%%%%%%%%%%%%%%%%%%%%%%%%%%%%%%%
%% Derived Class
%%%%%%%%%%%%%%%%%%%%%%%%%%%%%%%%%%%%%%%%%%%%%%%%%%%%%%%%%%%%%%%%%%%%%%%%%%%%%%%%
\section{Derived Classes}

%% PQP
\subsection{\texttt{PQPCollisionDetection}}

%% RAPID
\subsection{\texttt{RapidCollisionDetection}}

%% SOLID
\subsection{\texttt{SolidCollisionDetection}}

%% Spheres
\subsection{\texttt{SpheresCollisionDetection}}

%% VCLIP
\subsection{\texttt{VClipCollisionDetection}}

