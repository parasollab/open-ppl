\chapter{Path Modifiers}

Path Modifiers (\pam) take an input path, or a series of configurations a
resolution distance part, and attempt to make that path better. This might
involve cutting out back tracking motions, applying local optimization
techniques, or pushing the path to the medial axis.

%%%%%%%%%%%%%%%%%%%%%%%%%%%%%%%%%%%%%%%%%%%%%%%%%%%%%%%%%%%%%%%%%%%%%%%%%%%%%%%%
%% Base Class
%%%%%%%%%%%%%%%%%%%%%%%%%%%%%%%%%%%%%%%%%%%%%%%%%%%%%%%%%%%%%%%%%%%%%%%%%%%%%%%%
\section{\texttt{PathModifierMethod}}

\texttt{PathModifierMethod} has one main method, \CALL{Modify}, which takes an
input path and produces an output path. Usage:
\begin{lstlisting}
PathModifierPointer pm = this->GetMPProblem()->GetPathModifier(m_pmLabel);
vector<CfgType> inputPath, outputPath;
pm->Modify(inputPath, outputPath);
\end{lstlisting}

%%%%%%%%%%%%%%%%%%%%%%%%%%%%%%%%%%%%%%%%%%%%%%%%%%%%%%%%%%%%%%%%%%%%%%%%%%%%%%%%
%% Derived Class
%%%%%%%%%%%%%%%%%%%%%%%%%%%%%%%%%%%%%%%%%%%%%%%%%%%%%%%%%%%%%%%%%%%%%%%%%%%%%%%%
\section{Derived Classes}

%% Combined
\subsection{\texttt{CombinedPathModifier}}

%% Medial Axis
\subsection{\texttt{MedialAxisPathModifier}}

%% Resample
\subsection{\texttt{ResamplePathModifier}}

%% Shortcutting
\subsection{\texttt{ShortcuttingPathModifier}}
