\chapter{Extenders}

An Extender (\exte) is used to create a new configuration $q_{new}$ in a
direction towards $q_{dir}$ from a starting configuration $q$. This is commonly
used for single-query planners such as \rrt.

%%%%%%%%%%%%%%%%%%%%%%%%%%%%%%%%%%%%%%%%%%%%%%%%%%%%%%%%%%%%%%%%%%%%%%%%%%%%%%%%
%% Base Class
%%%%%%%%%%%%%%%%%%%%%%%%%%%%%%%%%%%%%%%%%%%%%%%%%%%%%%%%%%%%%%%%%%%%%%%%%%%%%%%%
\section{\texttt{ExtenderMethod}}

\texttt{ExtenderMethod} has one main method, \CALL{Extend}, to perform this main
operation. Usage:
\begin{lstlisting}
ExtenderPointer e = this->GetMPProblem()->GetExtender(m_eLabel);
CfgType c, cDir, cNew;
vector<CfgType> intermediates;
e->Extend(c, cDir, cNew, intermediates);
\end{lstlisting}
\CALL{Extend} returns a boolean of success/fail, fills cNew with the extended
configuration, and fills intermediates with any intermediate configurations
along the polygon chain of the expansion --- not all expansion methods go in
straight lines through $\cspace$.

%%%%%%%%%%%%%%%%%%%%%%%%%%%%%%%%%%%%%%%%%%%%%%%%%%%%%%%%%%%%%%%%%%%%%%%%%%%%%%%%
%% Derived Class
%%%%%%%%%%%%%%%%%%%%%%%%%%%%%%%%%%%%%%%%%%%%%%%%%%%%%%%%%%%%%%%%%%%%%%%%%%%%%%%%
\section{Derived Classes}

%% Basic
\subsection{\texttt{BasicExtender}}

%% Random Obstacle Vector
\subsubsection{\texttt{RandomObstacleVector}}

%% Rotation Then Translation
\subsubsection{\texttt{RotationThenTranslation}}

%% Trace C-space Obstacle
\subsubsection{\texttt{TraceCSpaceObstacle}}

%% Trace Obstacle
\subsubsection{\texttt{TraceObstacle}}

%% Trace Medial Axis Push
\paragraph{\texttt{TraceMAPush}}

%% Mix
\subsection{\texttt{MixExtender}}

