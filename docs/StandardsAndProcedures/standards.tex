\documentclass[12pt]{article}

\usepackage[top=0.5in,left=0.5in,right=0.5in, bottom=0.5in]{geometry}
\usepackage{amssymb}
\usepackage{algorithm,algorithmic}

\title{\Large \bf Standards and Procedures of PMPL}
\author{Jory Denny}

\begin{document}
\thispagestyle{empty}
\pagestyle{empty}
\maketitle

This document outlines the standards for all procedures within PMPL. This ranges from coding standards, to testing
standards, to documentation standards, etc. Everyone will be expected to adhere to these standards.

\section{Check-in Procedure}
The following procedure will occur anytime a check-in will take place within PMPL.
\begin{algorithmic}[1]
\STATE One person will check over correctness and testing for proposed code.
\STATE A separate person shall check over the coding and documentation standards set forth within this document.
\STATE Back up validation shall occur during the PMPL nightly tests, person who checks in the code is responsible for
fixing warnings and errors caused by the check-in
\end{algorithmic}

\section{Coding Standards}

\subsection{.h/.cpp}
All classes and files will be split between .h/.cpp format when possible. Exceptions are templated functions, and templated
classes (avoided when possible). All .h files should have \#define guards with the name of the file in all caps with
\_H\_ at the end (e.g., for ExampleClass.h -- EXAMPLECLASS\_H\_).

\subsection{Classes}
\subsubsection{Class Names}
All class names will begin with a capitol letter and each successive word will be capitalized. Ex. class MyPMPLClass.

\subsubsection{Class Variables}
Class variables should be indicated with m\_ (e.g., m\_variableName). No type specification is necessary (i.e.,
m\_pVariable for a pointer).

\subsubsection{Function Parameter Names}
Function parameters should be indicated with an \_ (e.g., \_variableName).

\subsection{All variable/function/class names}
All variables will be named in CamelCasing. Class names and function names should begin with a capitol letter (i.e.,
MyClass or MyFunction) while variables should begin with a lower case letter (i.e., m\_classVariable,
\_functionParameter, or tempVar).

\subsection{Spaces/Tabbing}
All tab characters should be represented by 2 spaces (easy to make default in all editors, vim is recommended).

\subsection{Curly Braces}
Curly braces used in control flow, function definitions, or class definitions should be placed on the same line as the
statement (e.g., class MyClass $\{$, void MyFunction()$\{$, while(1)$\{$).

\subsection{Function Call/Declaration}
Function calls or declarations which are "long" should be split between multiple lines when necessary to aid in reading
the code.

\subsection{m\_Debug}
There will be a variable, m\_Debug in the PMPL base class which will be used to determine when output occurs. This means
that there will be no \#Define statements for debug output, all debug output should be surrounded by:

if(m\_debug) cout$<<$"My debug statement";

\subsection{PrintOptions}
In all classes one function PrintOptions should be implemented which displays the values of the class variables at the
time of calling. This function can be called after XML initialization, but should not be called else where (at least in
checked-in code).

\subsection{Example Class}
Example class is located in ExampleClass.h/.cpp within this same folder.

\section{Documentation}
All documentation should be done with DOxygen. DOxygen standards used within PMPL are not decided upon yet.

\section{Testing}
Currently only nightly compilation testing occurs nightly. This should be expanded in the future.

\end{document}
