\documentclass[12pt]{article}

\usepackage[top=0.5in,left=0.5in,right=0.5in, bottom=0.5in]{geometry}
\usepackage{amssymb}
\usepackage{algorithm,algorithmic}
\newcommand{\EXCISE}[1]{}

\usepackage[usenames]{color}
\usepackage{listings}
\definecolor{LightGrey}{rgb}{0.9,0.9,1.0}
\lstdefinestyle{C++}{language=C++,%
  backgroundcolor=\color{LightGrey},
  showstringspaces=false,
  columns=fullflexible,
  escapechar=@,
  basicstyle=\sffamily,
  moredelim=**[is][\color{white}]{~}{~},
  literate={=>}{{$\Rightarrow\;$}}1 {->}{{$\rightarrow{}$}}1 %
  {<-}{{$\leftarrow$}}1 {<:}{{$\subtype{}\ $}}1,
}

\newcommand{\pmpl}{PMPL}
\newcommand{\gforge}{\texttt{GForge}}

\title{\Large \bf Standards and Procedures of \pmpl}
\author{Jory Denny and Shawna Thomas and Nancy M. Amato}

\begin{document}

\lstset{style=C++}

\maketitle
\thispagestyle{empty}
\pagestyle{empty}

This document outlines the standards for all procedures within \pmpl. This
ranges from coding standards, to testing standards, to documentation standards,
etc. Everyone will be expected to adhere to these standards.

\clearpage
\pagestyle{plain}

%%%%%%%%%%%%%%%%%%%%%%%%%%%%%%%%%%%%%%%%%%%%%%%%%%%%%%%%%%%%%%%%%%%%%%%%%%%%%%%%%%%%%%%%%%%%%%%%%%%
%% PMPL Check-in procedure
%%%%%%%%%%%%%%%%%%%%%%%%%%%%%%%%%%%%%%%%%%%%%%%%%%%%%%%%%%%%%%%%%%%%%%%%%%%%%%%%%%%%%%%%%%%%%%%%%%%

\section{Check-in Procedure}
The following procedure will occur anytime a check-in will take place within
\pmpl. No warnings or errors are permitted in check-ins.
\begin{algorithmic}[1]
\STATE The check-in should be documented in a task on \gforge\ (for\
non-simplistic check-ins).
\STATE One person will check over correctness and testing for proposed code.
This person should be an experienced group member in the component of \pmpl\
related to the check-in. The discussion of correctness should be documented on
\gforge.
\STATE A separate person shall check over the coding and documentation standards
set forth within this document. This person can be any member of the motion
planning group and should be as nitpicky as possible. Any discussion of fixes in
standards should be documented on \gforge.
\STATE If the check-in involves a new algorithm, the documentation for \pmpl\
must be augmented to include this new algorithm.
\STATE Once approval is made on correctness and standards, the check-in can be
made and \gforge\ task closed.
\STATE Back up validation shall occur during the \pmpl\ nightly tests, person
who checks in the code is responsible for fixing warnings and errors caused by
the check-in.
\end{algorithmic}

%%%%%%%%%%%%%%%%%%%%%%%%%%%%%%%%%%%%%%%%%%%%%%%%%%%%%%%%%%%%%%%%%%%%%%%%%%%%%%%%%%%%%%%%%%%%%%%%%%%
%% PMPL Coding Standards
%%%%%%%%%%%%%%%%%%%%%%%%%%%%%%%%%%%%%%%%%%%%%%%%%%%%%%%%%%%%%%%%%%%%%%%%%%%%%%%%%%%%%%%%%%%%%%%%%%%

\section{Coding Standards}

\subsection{\texttt{.h/.cpp}}
All classes and files will be split between \texttt{.h/.cpp} format when
possible, i.e., non-template classes/functions. All \texttt{.h} files should
have \#define guards with the name of the file in all caps, with \texttt{\_}s
separating words and \texttt{\_H\_} at the end. For example: for
\texttt{ExampleClass.h} -- \texttt{EXAMPLE\_CLASS\_H\_}.

\begin{lstlisting}
#ifndef EXAMPLE_CLASS_H_
#define EXAMPLE_CLASS_H_

class Class;

#endif
\end{lstlisting}

\subsection{Naming}

\subsubsection{\#define}
All \#define preprocessor commands should be named in all capital letters, with
\_s separating words.

\begin{lstlisting}
#define MY_DEFINE
\end{lstlisting}

When possible, \#defines should be avoided and use constant functions instead.
This is the C++ style.

\subsubsection{Types}
All types, being all classes, structs, typedefs, enums, etc, should begin
with a capital letter and be in camel casing format (i.e., each successive word
will be capitalized).

\begin{lstlisting}
class MyClass;
struct MyStruct;
typedef int MyInt;
enum MyEnum {Value1, Value2, Value3};
\end{lstlisting}

\paragraph{Pointer and Reference Type Specification}

When declaring a variable to be a pointer or reference the * or \& should be
adjacent to the type:

\begin{lstlisting}
Type* pointerName;
Type& referenceName;
\end{lstlisting}

\subsubsection{Enums}
Enum values should be named like types, i.e, camel casing beginning with a
capital letter.

\begin{lstlisting}
enum MyEnum {Value1, Value2, Value3};
\end{lstlisting}

\subsubsection{Functions}
Functions shall be names with camel casing beginning with a capital letter.

\begin{lstlisting}
void MyFunction();
\end{lstlisting}

\subsubsection{Variables}

All variables should be named in camel casing beginning with a lowercase letter.

\begin{lstlisting}
int myInt;
\end{lstlisting}

\paragraph{Member/class variables}

All member variables should begin with an m\_. No type specification is
necessary (i.e., no m\_pVariable for a pointer).


\begin{lstlisting}
class MyClass{
  public:
    int myInt;
}
\end{lstlisting}

\paragraph{Function parameters}

All function parameters should begin with an \_.

\begin{lstlisting}
int MyFunction(int _a, int _b);
\end{lstlisting}

\subsection{XML}
Similarly to the class and variable names, within the XML, tags should be camel
cased beginning with a capital letter and variables should be camel cased
beginning with a lowercase letter.

\begin{lstlisting}
<MyXMLTag label="label" myXMLVarable="false"/>
<MyXMLTagWithChildren label="label2">
  <Child method="label">
  <Child method="label2">
</MyXMLTagWithChildren>
\end{lstlisting}

When loading labels for \pmpl\ methods, e.g., distance metrics, the variable
name should be the \pmpl\ acronym for that component with Label. So for a
distance metric label variable it should be named dmLabel, a validity checker
label should be vcLabel, etc. Acronyms are as follows:

\begin{itemize}
  \item Distance Metric - dm
  \item Validity Checker - vc
  \item NeighborhoodFinder - nf
  \item Sampler - s
  \item Local Planner - lp
  \item Connector - c
  \item Map Evaluator - me
  \item Metric - m
  \item Extender - e
  \item Path Modifier - pm
  \item Motion Planning Strategy - mps
\end{itemize}

\subsection{Spaces/Tabbing}
All tab characters should be represented by 2 spaces (easy to make default in
all editors, vim is recommended). There should be no trailing white space at the
end of lines (this causes unnecessary conflicts with in SVN).

\subsection{Curly Braces}
Curly braces used in control flow, function definitions, or class definitions
should be placed on the same line as the statement. It is recommended that a
space occurs before the curly brace, e.g., if() \{ NOT if()\{.

\begin{lstlisting}
class MyClass {
  void MyFunction() {
    while(1) {
    }
  }
}
\end{lstlisting}

Additionally, if the body of control flow structures is one line, curly braces
are not required, and the body must appeat on a new line.

\begin{lstlisting}
if(true)
  doSomething();
\end{lstlisting}

\subsection{Function Call/Declaration/Definition}
Function calls or declarations which are "long" should be split between multiple
lines when necessary to aid in reading the code.

Definitions of functions should be outside the class, except in one-line
functions. Definitions shall be formatted with the template, return type, class
name, and function name all on separate lines. The function parameters may span
multiple lines if necessary.

\begin{lstlisting}
/* Descriptive comment of function recommended but not enforced at this time
 */
template<class MPTraits>
ReturnType
ClassName::
FunctionName(parameter list which may span multiple lines) {
}
\end{lstlisting}

\subsection{MPBaseObject inherited variables/functions}

\subsubsection{m\_debug}

There will be a variable, m\_debug in the PMPL base class which will be used to
determine when output occurs, e.g., in debugging statements. This means that
there will be no \#Define statements for debug output. All debug output will be
of the form:

\begin{lstlisting}
if(m_debug)
  cout << "My debug statement";
\end{lstlisting}

\subsubsection{GetNameAndLabel()}

This function returns a unique identifier to this class based on XML input.
Returns m\_name + "::" + m\_label. This should be used in output and record
keeping.

\begin{lstlisting}
string callee = this->GetNameAndLabel();
\end{lstlisting}

\subsubsection{Print}

In all classes one function Print should be implemented which displays the
values of the class variables at the time of calling. This is called for every
loaded variable after parsing the XML in the main function of PMPL.

\begin{lstlisting}
void Print(ostream& _os) const {
  _os << GetNameAndLabel() << endl;
  //output rest of class values
}
\end{lstlisting}

\subsubsection{ParseXML}

All derived classes from MPBaseObject have the option to implement ParseXML to
aid in code clarity for construction of objects from XML nodes.

\subsection{Throwing Errors}

All errors, unless otherwise approved should be thrown using a PMPLException, or
other base type. This is to encourage recovering from errors in sister programs,
like Vizmo and GroupBehaviors. Exception classes are located within
\texttt{src/Utilities/PMPLExceptions.h}. Use the WHERE define to give an exact
location of the error. The PMPLException base is nicely formatted.

\begin{lstlisting}
throw PMPLException("MyExceptionType", WHERE, "My error message");
\end{lstlisting}

\subsection{STL and External Libraries}

The STL and External Libraries, e.g, boost and CGAL should be used as much as
possible, except when there is a good reason not too, e.g., efficiency.

\subsection{Statistics Tracking}

All statistics should be tracked and recorded in the StatClass. There are
standard map data structures to associate names with statistics, e.g., sampling
attempts. The name should include the class name and label. There is a separate
map for each piece of the code and the output of such statistics is standardized
to aid in automated data collection.

\subsection{Example Class}
Example class is located in \texttt{ExampleClass.h} within this same folder:

\lstinputlisting{ExampleClass.h}

%%%%%%%%%%%%%%%%%%%%%%%%%%%%%%%%%%%%%%%%%%%%%%%%%%%%%%%%%%%%%%%%%%%%%%%%%%%%%%%%%%%%%%%%%%%%%%%%%%%
%% DOCUMENTATION AND TESTING STANDARDS NOT SET
%%%%%%%%%%%%%%%%%%%%%%%%%%%%%%%%%%%%%%%%%%%%%%%%%%%%%%%%%%%%%%%%%%%%%%%%%%%%%%%%%%%%%%%%%%%%%%%%%%%

\EXCISE{

\section{Documentation}
All documentation should be done with DOxygen. DOxygen standards used within PMPL are not decided upon yet.

\section{Testing}
Currently only nightly compilation testing occurs nightly. This should be expanded in the future.

}

\end{document}
